%============================ MAIN DOCUMENT ================================
% define document class
\PassOptionsToPackage{table}{xcolor}
\documentclass[
    a4paper,
    BCOR=15mm,            % Binding correction
    twoside,
    openany,
% openright,
%  headings=openright,
    bibliography=totoc,   % If enabled add bibliography to TOC
    listof=totoc,         % If enabled add lists to TOC
    monolingual,
% bilingual,
    invert-title,
]{bfhthesis}

\LoadBFHModule{listings,terminal,boxes}
%---------------------------------------------------------------------------
% Documents paths
%---------------------------------------------------------------------------
\makeatletter
\def\input@path{{content/}}
%or: \def\input@path{{/path/to/folder/}{/path/to/other/folder/}}
\makeatother
%-----------------  Base packages     --------------------------------------
% Include Packages
\usepackage[french,ngerman,main=english]{babel}  % https://www.namsu.de/Extra/pakete/Babel.html
%% To disable the french list setting you can add -- see https://gitlab.ti.bfh.ch/bfh-latex/bfh-ci/-/issues/166
%\frenchsetup{StandardLists=true}

\usepackage{amsmath}          % various features to facilitate writing math formulas
\usepackage{amsthm}           % enhanced version of latex's newtheorem
\usepackage{amsfonts}         % set of miscellaneous TeX fonts that augment the standard CM
\usepackage{amssymb}          % mathematical special characters

\usepackage{siunitx}

\usepackage{graphicx}         % integration of images
\usepackage{float}            % floating objects

\usepackage{caption}          % for captions of figures and tables
\usepackage{subcaption}       % for subcaptions in subfigures
\usepackage{cite}             % use bibtex
\usepackage{wrapfig}

\usepackage{exscale}          % mathematical size corresponds to textsize
\usepackage{multirow}         % multirow emables combining rows in tables
\usepackage{multicol}

\usepackage{parskip}
\usepackage{tabularx}
\usepackage{pifont}
\usepackage{array}
\usepackage{longtable}
\usepackage{booktabs}

%---------------------------------------------------------------------------
% Graphics paths
%---------------------------------------------------------------------------
\graphicspath{{pictures/}{figures/}}
%---------------------------------------------------------------------------
% Blind text -> for dummy text
%---------------------------------------------------------------------------
\usepackage{blindtext}
\usepackage{letltxmacro}
\LetLtxMacro{\blindtextblindtext}{\blindtext}

\RenewDocumentCommand{\blindtext}{O{\value{blindtext}}}{
    \begingroup\color{BFH-Gray}\blindtextblindtext[#1]\endgroup
}
%---------------------------------------------------------------------------
% Glossary Package
%---------------------------------------------------------------------------
% the glossaries package uses makeindex
% if you use TeXnicCenter do the following steps:
%  - Goto "Ausgabeprofile definieren" (ctrl + F7)
%  - Select the profile "LaTeX => PDF"
%  - Add in register "Nachbearbeitung" a new "Postprozessoren" point named Glossar
%  - Select makeindex.exe in the field "Anwendung" ( ..\MiKTeX x.x\miktex\bin\makeindex.exe )
%  - Add this [ -s "%tm.ist" -t "%tmx.glg" -o "%tm.gls" "%tm.glo" ] in the field "Argumente"
%
% for futher informations go to http://ewus.de/tipp-1029.html
%---------------------------------------------------------------------------
\usepackage[nonumberlist]{glossaries-extra}
\makeglossaries
\newacronym{arp}{ARP}{Address Resolution Protocol}

\newglossaryentry{BlueTeam}{
    name=Blue Team,
    description={The blue team is responsible for protective measures within an organization; during an exercise, they detect and defend against red team engagements.}
}

\newacronym{cli}{CLI}{Command Line Interface}

\newacronym{dhcp}{DHCP}{Dynamic Host Configuration Protocol}

\newacronym{dns}{DNS}{Domain Name System}

\newglossaryentry{GreenTeam}{
    name=Green Team,
    description={The green team represents the cooperation between the blue and yellow teams to close knowledge gaps in both areas.}
}

\newacronym{ids}{IDS}{Intrusion Detection System}

\newacronym{ips}{IPS}{Intrusion Prevention System}

\newacronym{mitm}{MITM}{Man-in-the-Middle}

\newglossaryentry{ProxyServer}{
  name=Proxy Server,
  description={An intermediary service between a client and a server.}
}

\newacronym{nsak}{NSAK}{Network Swiss Army Knife}

\newacronym{oci}{OCI}{Open Container Initiative}

\newglossaryentry{OrangeTeam}{
    name=Orange Team,
    description={The orange team represents the cooperation between the red and yellow teams, primarily for sharing insights, knowledge, and education.}
}

\newglossaryentry{PurpleTeam}{
    name=purple team,
    description={Purple teaming describes the cooperation between red and blue teams to enhance mutual understanding and defensive capabilities.}
}

\newglossaryentry{RedTeam}{
    name=red team,
    description={The red team consists of ethical hackers who simulate attacks against systems, networks, and software to test defensive measures.}
}

\newacronym{rogueap}{RAP}{Rogue Access Point}

\newacronym{siem}{SIEM}{Security Information and Event Management}

\newacronym{soc}{SOC}{Security Operations Center}

\newacronym{ssid}{SSID}{Service Set Identifier}

\newacronym{tcp}{TCP}{Transmission Control Protocol}

\newacronym{udp}{UDP}{User Datagram Protocol}

\newglossaryentry{WhiteTeam}{
    name=White Team,
    description={The white team consists of non-technical and technical stakeholders who provide oversight, compliance guidance, and organizational requirements during an exercise.}
}

\newglossaryentry{YellowTeam}{
    name=Yellow Team,
    description={The yellow team consists of system, network, and software architects and engineers who design and maintain the infrastructure.}
}

\newacronym{api}{API}{Application Programming Interface}

\newacronym{rest}{REST}{Representational State Transfer}

\newacronym{yaml}{YAML}{YAML Ain't Markup Language}

\newacronym{git}{GIT}{Git is a free and open source distributed version control system designed to handle everything from small to very large projects with speed and efficiency}

\newacronym{mac}{MAC}{Media Access Control}

%---------------------------------------------------------------------------
% Makeindex Package
%---------------------------------------------------------------------------
\usepackage{makeidx}
\makeindex
%\usepackage{imakeidx}          % To produce index
%\makeindex[columns=2,intoc]    % Index-Initialisation
%\makeindex[columns=3,columnseprule,columnsep,intoc]
%---------------------------------------------------------------------------
% Hyperref Package (Create links in a pdf)
%---------------------------------------------------------------------------
\usepackage[
,bookmarks
,plainpages=false
,pdfpagelabels
,pdfusetitle
,backref = {false}          % No index backreference
,colorlinks = {true}        % Color links in a PDF
,hypertexnames = {true}     % no failures "same page(i)"
,bookmarksopen = {true}     % opens the bar on the left side
,bookmarksopenlevel = {0}   % depth of opened bookmarks
,linkcolor=.
,filecolor=.
,urlcolor=.
,citecolor=.
]{hyperref}
\usepackage{xparse}
\usepackage{color}
\usepackage{todonotes}
\usepackage[utf8]{inputenc}
%\usepackage{glossaries}
\usepackage{csquotes}
\usepackage{textcomp}
%---------------------------------------------------------------------------

%% %% Customize Footer and Headers in Document
%% \KOMAoptions{headsepline,plainheadsepline,footsepline,plainfootsepline}%
%% \setkomafont{headsepline}{\color{BFH-DarkBlue}}% BFH-DarkBlue required bfhcolors
%% \setkomafont{footsepline}{\color{BFH-DarkBlue}}%
%% \lehead*{lehead} % the * character does replace the header on the first chapter page as well
%% \cehead*{cehead}
%% \rehead*{rehead}
%% \lohead*{lohead}
%% \cohead*{cohead}
%% \rohead*{rohead}

%% \lefoot*{lefoot}
%% \cefoot*{cefoot}
%% \refoot*{refoot}
%% \lofoot*{lofoot}
%% \cofoot*{cofoot}
%% \rofoot*{rofoot}
%---------------------------------------------------------------------------
\begin{document}
    \raggedbottom  % content does not get strechted verticaly

%% Snippet do redefine German babel translations -- see FAQ latex.ti.bfh.ch for further information or
%% the package documentation https://www.ctan.org/pkg/translations
%% \redefinetranslation{German}{Advisor}{Dozentin}
%% \redefinetranslation{German}{advisor}{Dozentin} %  just for consistency
%% \redefinetranslation{German}{Author}{Autorin}
%% \redefinetranslation{German}{author}{Autorin} %  just for consistency
%% \redefinetranslation{German}{Co-advisor}{Mitbetreuerin}
%% \redefinetranslation{German}{co-advisor}{Mitbetreuerin} %  just for consistency
%% \redefinetranslation{German}{Expert}{Expertin}
%% \redefinetranslation{German}{expert}{Expertin} %  just for consistency
%% \redefinetranslation{German}{Project partner}{Projektpartnerin}
%% \redefinetranslation{German}{project partner}{Projektpartnerin} %  just for consistency

%------------ START FRONT PART ------------
    \frontmatter

    \title{Project 2}
    \subtitle{Network Swiss Army Knife}
    \author{Frank Gauss (gausf1) and Lukas von Allmen (vonal3)}
    \institution{Bern University of Applied Sciences}
    \department{Technik und Informatik}
%    \institute{TBD}
    \version{1.0}
%    \titlegraphic*{\includegraphics{somePicture}}
    \advisor{Wenger Hansjürg}
%    \coadvisor{PhD A. Smart}
%    \projectpartner{proj partner}
%    \expert{Some expert}
    \degreeprogram{Bachelor of Science in Computer Science}
%    \setupSignature{
%        A. Muster={\includegraphics[width=.5\linewidth]{sig_muster}},
%        C. Example={\includegraphics[width=.5\linewidth]{sig_example}}
%    }


%----------------  BFH tile page   -----------------------------------------
    \maketitle
%------------ ABSTRACT        ----------------
    \addchap{Abstract}
    \section{Abstract}

One-paragraph summary of the entire study – typically no more than 250 words in length (and in many cases it is well shorter than that), the Abstract provides an overview of the study.

\pagebreak

Inhalt Abstract
– Hintergrundinformationen wie Ausgangslage, Relevanz, Forschungskontext in ein bis zwei Sätzen zusammenfassen.
– Fragestellung und Ziel explizit formulieren.
– Die wichtigsten Eckpunkte zum methodischen Vorgehen angeben, bei empirischen Studien auch Angaben zu den Daten wie
etwa die Charakteristika der Stichprobe.
– Im Hauptteil des Abstracts die relevanten Ergebnisse und deren Bedeutung mit wichtigen Kennzahlen aufführen (ca. zwei Drittel des Abstracts).
– Mit wichtigen Schlussfolgerungen oder Anwendungsmöglichkeiten das Abstract abrunden.
– Das Abstract enthält keine Quellenverweise


%------------ TABLEOFCONTENTS ----------------
    \tableofcontents

%------------ START MAIN PART ------------
    \mainmatter


    \chapter{Introduction}\label{ch:introduction}
    \section{Introduction}\label{sec:introduction}

According to the World Economic Forum's Global Risk Report 2025, the categories \enquote{Crime and illicit economic activity incl. Cyber} and \enquote{Cyber espionage and warfare} are both ranked among the top 10 global risks in the next two to ten years~\cite{WEF2025GlobalRisks}.
These risks are expected to intensify even further because the economic and operational costs of launching cyberattacks will decrease due to AI automation~\cite{Garg2024AIEconomicsCyberattacks}.
This underlines the need for cost-effective and easy-to-use security tools, methods and frameworks (conceptional and software) to identify and defend against cyberattacks.

\textbf{Information security methods and conceptional frameworks}\\
In practice, the method of combining red team activities and blue team observation techniques is widely adopted within the cybersecurity community and industry~\cite{cremen2018infoseccolourwheel}.
While the red team focuses on emulating adversarial behavior, the primary objective of the blue team is to detect such activities through non-invasive monitoring and analysis of system behavior~\cite{NIST2012RedTeamBlueTeamApproach}.
Further, we can observe approaches from the community and the industry to evolve this approach in to the so-called \enquote{InfoSec color wheel}~\cite{cremen2018infoseccolourwheel}.
In the proposed InfoSec color wheel, the author splits the six colors into primary and secondary colors, where the primary colors are teams on their own and the secondary colors are cooperation between two primary color teams.
The primary colors are represented by the \gls{RedTeam}, the \gls{BlueTeam}, and a newly introduced the \gls{YellowTeam}, which represents the \enquote{builders} of software and systems.
The secondary colors are represented by the \gls{PurpleTeam} (red and blue), the \gls{OrangeTeam} (red and yellow), and the \gls{GreenTeam} (yellow and blue).
Where \enquote{purple teaming} is actually an already established praxis as it evolved naturally from the cooperation between red and blue teams~\cite{NIST2012RedTeamBlueTeamApproach}.


\textbf{Information security tools and software framworks}\\
One approach to reduce operational security costs is to adopt multiple modular frameworks that can be easily extended, configured, and executed continuously in a controlled manner~\cite{Zilberman2020SoKThreatEmulators}.
The threat emulation frameworks analyzed by Zilberman et al.\ evaluate multiple attack phases, including lateral movement, persistence, and attack execution.


\textbf{\acrfull{nsak}}

The Network Swiss Army Knife focuses on containerized, orchestrated scenarios that execute specific attack drills in a controlled environment.
Future extensions will focus on enriching the assessment layer by systematically capturing and evaluating defensive responses of multiple scenarios.

This proof of concept comprises the design and implementation of a modular, isolated open-source security framework that focuses on extensibility and the controlled execution of attack-based scenarios.

The objective of this work is to investigate whether such a framework can provide a flexible, extendable, and safe foundation for modular and automated security testing in a network environment. .

In the summary of their paper, Zilberman et al.\ are highlighting the necessity of the following design requirements~\cite{Zilberman2020SoKThreatEmulators}:
\begin{itemize}
    \item Cleanup and configurability are important in order to repeat and automate the execution of attack scenarios during security tool assessment and what-if analysis.
    \item An emulator should support cleanup after the completion of the attack scenario, like CALDERA, Atomic Red Team, and Infection Monkey do, rather than after each individual procedure.
    \item An API, currently provided by Atomic Red Team, CALDERA, and Metasploit, facilitates integration between the threat emulators and organizational security array, thus enabling periodic and systematic security assessment.
    \item It is important to provide a GUI and ready to execute multi-procedure attacks for novice operators as well as a \gls{cli} to support automation and advanced customization capabilities.
\end{itemize}

We reconsolidate the highlighted design requirements for the implementation of \acrshort{nsak} into the following list of features:
\begin{itemize}
    \item CLI, GUI, and API to manage resources and execute scenarios.
    \item Configurability of the framework and the resources.
    \item Automatic cleanup procedures after the completion of a scenario.
\end{itemize}

Even though we agree with the importance of the highlighted design requirements, we are not able to implement them all in the time constraints of this project.
Because we are planning to build upon this PoC, we will incorporate the design requirements in the chapter architecture and design~\ref{ch:design_and_architecture} of this paper and list them in the future work~\ref{sec:future-work} section.

    \section{Current State of Research}\label{sec:current-state-of-research}


Recent studies highlight the importance of modularity, reproducibility, and automation to reduce the operational overhead of
security assessments.
Methodologies such as red and blue teaming, and their combinations within the InfoSec color wheel, show
the complexity and overlapping disciplines in the security sector.
\cite{NIST2012RedTeamBlueTeamApproach, cremen2018infoseccolourwheel}

On a technical level, threat emulation frameworks such as CALDERA, Atomic Red Team, and Metasploit
implement multi-stage attack scenarios to evaluate the detection of defensive systems.
However, many existing frameworks focus primarily on large-scale enterprise environments and require significant
setup effort, limiting their adaptability in resource-constrained networks.
\cite{Zilberman2020SoKThreatEmulators}

These aspects highlight the need for a lightweight and modular framework to reduce overall cyber thread risks in
network infrastructures.\cite{WEF2025GlobalRisks}



    \chapter{Design and Architecture}\label{ch:design_and_architecture}
    \section{Framework Concepts}\label{sec:framework-concepts}

This section describes the high-level concepts, resources and vocabulary needed to understand and work with the \acrshort{nsak} framework.

Overview of the \acrshort{nsak} resources and concepts:
\begin{itemize}
    \item Devices
    \item Environments
    \item Drills
    \item Scenarios
    \item Operator
    \item Operation
\end{itemize}

\subsection{Devices}\label{subsec:devices}
Under a \textbf{device} we understand a physical or virtual machine, which is capable of running the \acrshort{nsak} framework.
Even though we currently only work and describe the hardware devices evaluated in the section Hardware Selection~\ref{sec:hardware-selection}, other devices or virtual machines could be used with \acrshort{nsak}\@.

The following list vaguely describes the minimum requirements for a device:
\begin{itemize}
    \item Processor architecture: ARM and x86 should work equally well, as the \acrshort{nsak} framework is written in python and the scenarios are \acrshort{oci} images/containers, which are built on the \acrshort{nsak} device.
    \item Capable of running a Linux-based operating system, such as Debian.
    \item Enough memory and compute resources to run multiple \acrshort{oci} containers.
    \item Ideally, multiple physical network ports and Wi-Fi for covering many scenarios and environments.
    \item Optionally, additional bulk storage for data collection, such as PCAPs via T-Shark.
\end{itemize}

\textbf{Provisioning a \acrshort{nsak} device} usually consists of the following tasks:
\begin{enumerate}
    \item Install and configure a Linux-based operating system
    \item Set up a minimal network configuration and SSH access
    \item Install system dependencies required for \acrshort{nsak}
    \item Install and configure \acrshort{nsak}
\end{enumerate}

After a device is provisioned, we refer to it as a \textbf{\acrshort{nsak} device}, which may or may not be prepared for an operation.

\subsection{Environments}\label{subsec:environments}
An \textbf{environment} is representing a specific network topology including infrastructure components, servers, clients and services.
Ideally, an environment describes a part or a subset of a network and system infrastructure like you would encounter in a real organization.

Examples of environments:
\begin{itemize}
    \item WLAN AP: Smartphone, WLAN AccessPoint, Router
    \item Client - server: Client, Server, Switch
    \item Home network: Router, WLAN, One Physical Network (Star Topology), Multiple Devices (Computers, Laptops, SmartPhones, SmartTVs)
    \item Business network: Firewall, Router, DC Server, Intranet, Multiple Subnets, Multiple WLAN Access points, Switches
\end{itemize}

\subsection{Drills}\label{subsec:drills}
A \textbf{drill}, initially called a module, is a sequence of actions with a specific goal.
This goal can be an active or passive attack, network discovery, monitoring, analysis, data extraction, a hook for manual intervention or a device configuration.

Examples of drills:
\begin{itemize}
    \item Network sniffing with TShark with a specific filter (http traffic)
    \item Data extraction on an internal bulk storage or external network file system
    \item Active or passive \acrshort{mitm} (man in the middle) attack with a transparent \acrshort{tcp} proxy
    \item \acrshort{arp} Spoofing
    \item WLAN SSID spoofing
    \item Network discovery with nmap or \acrshort{arp}-scan
    \item Network configuration, such as enabling IP-Forwarding or NAT
\end{itemize}

\subsection{Scenarios}\label{subsec:scenarios}
A \textbf{scenario} is designed for one or multiple environments, consists of a sequence of drills and describes a concrete use case for specific red or blue team activities.

Examples of Scenarios:
\begin{itemize}
    \item WLAN SSID Spoofing:
    \begin{itemize}
        \item Environment: WLAN AP
        \item Drills: Network configuration for DHCP, NAT, SSID Spoofing, Packet Sniffing
    \end{itemize}
    \item \acrshort{tcp} \acrshort{mitm} Attack:
    \begin{itemize}
        \item Environment: \acrshort{tcp} client - server
        \item Drills: Automatic network discovery and configuration, \acrshort{arp} Spoofing, Transparent \acrshort{tcp} Proxy, Packet manipulation
    \end{itemize}
\end{itemize}

\subsection{Operator}\label{subsec:operator}
For simplicity and consistency we use the term \textbf{operator} for the person or team, which is planning and executing operations with \acrshort{nsak}\@.
So an operator can refer to a single IT-specialist, a red, blue or purple team.

Examples of Operators:
\begin{itemize}
    \item A single IT-Specialized or Security researcher
    \item System and network engineering teams
    \item Usually, red, blue and purple teams, but potentially all teams in the InfoSec color wheel~\cite{cremen2018infoseccolourwheel}
\end{itemize}

\subsection{Operation}\label{subsec:operation}
An \textbf{operation} is the deployment of \acrshort{nsak} in a real network.

An operation explicitly excludes the development phase for scenarios, drills and environments, as these resources should be finalized and tested before being used in a real operation, otherwise the following conventions should be used:
\begin{itemize}
    \item \textbf{Simulated Operation:} Simulating an operation in a virtualized environment.
    \item \textbf{Test Operation:} Testing an operation in a physical lab network.
\end{itemize}

Preparing and planning an operation usually has the following sequence of tasks, assessed and executed by an operator:
\begin{enumerate}
    \item Provision a \acrshort{nsak} device.
    \item Select one or multiple environments which are relevant for the target network and system infrastructure.
    \item Configure and build all or a subset of scenarios which can be executed in the selected environments.
    \item Ideally, simulate and test the operation in a virtual or lab network infrastructure.
\end{enumerate}

    \section{Network Environments}\label{sec: Network Environment}

Each environment represents a practical setup in which the NSAK device can be deployed.
For traffic analysis, performance testing or security evaluation.

\textbf{Category I — Inline:}
\textbf{Diagram:} Laptop $\leftrightarrow$ NSAK device $\leftrightarrow$ Router  \\

\begin{figure}[H]
	\centering
	\includegraphics[width=0.7\linewidth]{../assets/figures/diagrams/network-diagrams/LAN-Sniffing-Switch-AccessPoint}
	\caption{The network topology with inline NSAK-device}
	\label{fig:lan-inline}
\end{figure}


\textbf{Description:}
Figure~\ref{fig:lan-inline} shows direct inline bridge between a client or switch and router.
Used for basic LAN capturing, latency, and throughput testing.

\textbf{Category II — Wireless:}
\textbf{Diagram:} Laptop, Smart Devices, Printer $\leftrightarrow$ NSAK device (inline) $\leftrightarrow$ Router  \\
\begin{figure}[H]
	\centering
	\includegraphics[width=0.7\linewidth]{../assets/figures/diagrams/network-diagrams/LAN-WLAN-Sniffing.drawio}
	\caption{The network topology for RAP and Wifi scenarios}
	\label{fig:lan-wlan-sniffing}
\end{figure}

\textbf{Description:}
The NSAK device is inline and lets traffic pass but intercepts as \gls{rogueap} (RAP) and capture data Figure~\ref{fig:lan-wlan-sniffing}.

    \section{Use-Cases}\label{sec:use-cases}

\begin{figure}[H]
    \centering
    \includegraphics[width=0.7\linewidth]{../assets/figures/diagrams/concept-architecture-diagrams/NSAK-use-case-diagram}
    \caption{User-Story UML Diagram}
    \label{fig:NSAK-use-case-diagram}
\end{figure}

\textbf{Figure~\ref{fig:NSAK-use-case-diagram}} illustrates the use case structure of the proposed \gls{nsak} modular framework.
The operator interacts with \gls{nsak} primarily through two high-level commands:
Build Scenario (UC-01) and Run Scenario (UC-06).
During the Build Scenario use case (UC-01), the system builds a scenario container for execution.
In complex network infrastructures, additional configuration parameters, such as network interface mappings may be
required and need to be provided as build-time arguments (UC-02).

The system loads the selected scenario (UC-03).
At this stage, the scenario orchestrates the required drills necessary to perform the intended attack.

The build process concludes with the creation of an \gls{oci} compliant container image (UC-05), which encapsulates
the fully configured scenario.

The Run Scenario use case (UC-06) represents an abstract execution phase.
In this phase, the previously built container image is run, and the configured attack drills are executed within the containerized environment (UC-08).

Finally, the system performs a cleanup procedure which needs to be optimized in future work, in which all scenario-specific resources,
processes, and drills are terminated.
This step minimizes side effects, reduces system noise, and prevents interference with other scenarios that may reuse the same drills.


\newcommand{\uctab}{\par\noindent\hspace{1em}}
\newcommand{\ucbullet}[1]{\uctab\textbullet~#1}

\begin{longtable}{@{} >{\columncolor{blue!15}}l >{\raggedright\arraybackslash}p{0.75\textwidth} @{}}
    \caption{Use Cases Specification (NSAK)}\\
    \toprule
    \textbf{NR} \& \textbf{Details} \\
    \endfirsthead
    \toprule
    \textbf{NR} \& \textbf{Details} \\
    \endhead
    \toprule

    % --- UC-01 ----------------------------------------------------
    \textbf{UC-01} & \textbf{Use-Case:} Build Scenario \par
                     \textbf{Description:} Builds a scenario container based on a selected scenario configuration. \par
                     \textbf{Actor:} Operator \par
                     \textbf{Trigger:} Operator initiates a scenario build via the command-line interface. \par
                     \textbf{Preconditions:} \gls{nsak} initialized; scenarios available. \par
                     \textbf{Main Scenario:} \par
                     \uctab{} 1. Operator selects a scenario to build using the command-line interface. \par
                     2. System validates the selected scenario. \par
                     3. System executes the included use cases: \par
                         \ucbullet Configure Scenario (UC-2) \par
                         \ucbullet Load Scenario (UC-3) \par
                         \ucbullet Load Drills (UC-4) \par
                         \ucbullet Build \gls{oci} Container (UC-5) \par
                     \textbf{Alternative Scenarios:} No scenarios available~\rightarrow~inform the operator. \par
                     \textbf{Error Scenarios:} Conflicting scenario configuration detected~\rightarrow~build aborted. \par
                     \textbf{Result:} Scenario container successfully built. \par
                     \textbf{Postconditions:} Scenario container stored and ready to run. \\
    \midrule
    % --- UC-02 ----------------------------------------------------
    \textbf{UC-02} & \textbf{Use-Case:} Configure Scenario \par
                     \textbf{Description:} Defines scenario-specific build parameters such as network interfaces and execution options. \par
                     \textbf{Actor:} System \par
                     \textbf{Trigger:} Scenario selected for build (UC-01). \par
                     \textbf{Preconditions:} Scenario selection is available. \par
                     \textbf{Main Scenario:}
                     1. System applies scenario-specific configuration parameters. \par
                     \textbf{Result:} Scenario configuration created. \par
                     \textbf{Postconditions:} Scenario configuration available for loading. \\
    \midrule
    % --- UC-03 ----------------------------------------------------
    \textbf{UC-03} & \textbf{Use-Case:} Load Scenario \par
                     \textbf{Description:} Loads and validate the selected scenarios \par
                     \textbf{Actor:} System \par
                     \textbf{Trigger:} Scenario configuration available (UC-02). \par
                     \textbf{Preconditions:} Scenario configuration created. \par
                     \textbf{Main Scenario:} \par
                     1. System retrieves the scenario definition files (scenario.yaml, scenario.py, README.md). \par
                     2. System validates the scenario structure and resolves declared dependencies. \par
                     \textbf{Error Scenarios:} Validation or dependency failure, preparation aborted with Error Log. \par
                     \textbf{Result:} Scenario is successfully loaded. \par
                     \textbf{Postconditions:} Scenario representation available for drill loading. \\
    \midrule
    % --- UC-04 ----------------------------------------------------
    \textbf{UC-04} & \textbf{Use-Case:} Load Drills \par
                     \textbf{Description:} Loads the attack drills required by the selected scenario. \par
                     \textbf{Actor:} System \par
                     \textbf{Trigger:} Scenario loaded (UC-03). \par
                     \textbf{Preconditions:} Scenario representation is available. \par
                     \textbf{Main Scenario:} \par
                     1. System resolves drill references defined in the scenario configuration. \par
                     2. System instantiates drill objects and loads associated metadata. \par
                     \textbf{Error Scenarios:} Invalid drill definition, drill not found, or ambiguous drill reference. \par
                     \textbf{Result:} Required drill objects loaded. \par
                     \textbf{Postconditions:} Drills available for container build. \\
    \midrule
    % --- UC-05 --------------------------------------------------
    \textbf{UC-05} & \textbf{Use-Case:} Build \gls{oci} Container \par
                     \textbf{Description:} Builds an \gls{oci} compliant container image for the loaded scenario. \par
                     \textbf{Actor:} System \par
                     \textbf{Trigger:} Scenario and drills loaded (UC-03, UC-04). \par
                     \textbf{Preconditions:} Scenario representation and drill objects available. \par
                     \textbf{Main Scenario:} \par
                     1. System generates the container build context. \par
                     2. System builds the scenario container image with required privileges and network configuration. \par
                     \textbf{Error Scenarios:} Container build failure — build aborted with an error message. \par
                     \textbf{Result:} \gls{oci} compliant scenario container image built. \par
                     \textbf{Postconditions:} Scenario container image stored and ready for execution. \\
    \midrule
    %--- UC-06 --------------------------------------------------
    \textbf{UC-06} & \textbf{Use-Case:} Run Scenario \par
                     \textbf{Description:} Executes a previously built scenario container. \par
                     Specific scenarios such as Rogue AP or ARP MITM represent specialized configurations of this use case.
                     \textbf{Actor:} Operator \par
                     \textbf{Trigger:} Operator initiates scenario execution via the command-line interface. \par
                     \textbf{Preconditions:} Scenario container image available (UC-05). \par
                     \textbf{Main Scenario:} \par
                     1. System starts the scenario container with the required execution parameters. \par
                     2. System executes the included use cases: \par
                         \ucbullet Execute Scenario (UC-07) \par
                         \ucbullet Cleanup Scenario (UC-09) \par
                     \textbf{Result:} Scenario container execution started. \par
                     \textbf{Postconditions:} Scenario execution context active. \\
    \midrule
    %--- UC-07 --------------------------------------------------
    \textbf{UC-07} & \textbf{Use-Case:} Execute Scenario \par
                     \textbf{Description:} Orchestrates the execution of a previously built scenario container and coordinates
                     the execution of the associated attack drills. \par
                     \textbf{Actor:} System \par
                     \textbf{Trigger:} Run Scenario (UC-6) \par
                     \textbf{Preconditions:} Scenario Image available and started \par
                     \textbf{Main Scenario:} \par
                     1. System Scenario Manager executes for the selected scenario \par
                     2. System Drill Manager executes drill UC-8 include use-case
                     \textbf{Error Scenarios:} Scenario not found or scenario container was not available. \par
                     \textbf{Result:} Scenario execution initiated and drill execution orchestrated. \par
                     \textbf{Postconditions:} Scenario container is running and drills are being executed.\\
    \midrule
    % --- UC-08 --------------------------------------------------
    \textbf{UC-08} & \textbf{Use-Case:} Execute Drills \par
                     \textbf{Description:} Executes the attack drills defined in the scenario configuration within the running scenario
                     container. \par
                     \textbf{Actor:} System \par
                     \textbf{Preconditions:} Scenario execution context initialized. \par
                     \textbf{Main Scenario:} \par
                     1. System Drill Manager retrieves the list of configured drills. \par
                     2. System Drill Manager executes the drills according to the defined order and parameters. \par
                     \textbf{Error Scenarios:} Drill execution failure or missing drill definition. \par
                     \textbf{Result:} Configured attack drills executed. \\
    \midrule
    % --- UC-09 --------------------------------------------------
    \textbf{UC-09} & \textbf{Use-Case:} Clean Up Scenario \par
                     \textbf{Description:} Terminates the running scenario container and restores the system to a defined baseline state. \par
                     \textbf{Actor:} System \par
                     \textbf{Trigger:} Stop Scenario (UC-06) \par
                     \textbf{Preconditions:} Scenario container is running. \par
                     \textbf{Main Scenario:} \par
                     1. System stops the running scenario container. \par
                     2. System invokes the included use case Clean Up Drills (UC-10). \par
                     \textbf{Error Scenarios:} Scenario container cannot be terminated. \par
                     \textbf{Result:} Scenario execution terminated. \par
                     \textbf{Postconditions:} Scenario container stopped and removed.\\
    \midrule
    % --- UC-10 --------------------------------------------------
    \textbf{UC-10} & \textbf{Use-Case:} Clean Up Drills \par
                     \textbf{Description:} Cleans up artifacts and state changes introduced by executed attack drills. \par
                     \textbf{Actor:} System \par
                     \textbf{Preconditions:} Drill execution completed or aborted. \par
                     \textbf{Main Scenario:} \par
                     1. System Drill Manager terminates active drill processes. \par
                     2. System Drill Manager removes temporary artifacts and resets modified parameters. \par
                     \textbf{Error Scenarios:} Incomplete cleanup due to failed drill termination. \par
                     \textbf{Result:} Drill-related artifacts removed and state reset. \\
    \bottomrule
\end{longtable}

    \section{Component-diagram}

\begin{figure}[H]
	\centering
	\includegraphics[width=1\linewidth]{../assets/figures/diagrams/concept-architecture-diagrams/project_2_nsak_component_diagram}
	\caption{}
	\label{fig:project2nsakcomponentdiagram}
\end{figure}

    \section{Sequence-Diagram}\label{sec:sequence-diagram}

\begin{figure}[H]
    \centering
    \includegraphics[width=0.9\linewidth]{../assets/figures/diagrams/concept-architecture-diagrams/sequence-diagram}
    \caption{Sequence of a Scenario loading and drill execution}
    \label{fig:sequence-diagram.drawio}
\end{figure}

\textbf{Figure~\ref{fig:sequence-diagram.drawio}} shows the interaction sequence for building,
loading, and executing a scenario within NSAK.
The diagram focuses on the main modularity concept and describes the orchestration flow between scenarios
and drills without interface details, error-handling, and drill or scenario clean-up mechanisms.

The process begins with the operator triggering the build command for a scenario container.
During this phase, the selected scenario is loaded and returned as a containerized representation.
In the execution phase, the scenario manager runs the container image and orchestrates the Drills order.
The Drill Manager executes the required drills.

A scenario may contain multiple drills; therefore, the * signalize various drills can be executed from a Drill Manager
in a one scenario.
Each drill is resolved and executed individually, while the scenario manager maintains complete control
over the scenario lifecycle.

Finally, each drill is intended to provide a clean-up procedure that can be triggered from the scenario manager.
All drill-specific artifacts should be removed from the host system to enable the sequential execution of multiple
scenarios.
The clean-up process is a planned extension of the framework.



    \chapter{Hardware Evaluation}\label{ch:system_evaluation}
    \input{content/08-hardware-evaluation}


    \chapter{Methods}\label{ch:methods}
    \input{content/09-method}


    \chapter{Implementation}\label{ch:implementation}
    In this chapter we will describe what and how we implemented the \acrshort{nsak} framework, the first two scenarios, their drills and environments.
To efficiently describe the implementation, we reference paths and files relative to the repositories root directory~\cite{nsakRepository2026}.

\section{\acrshort{nsak} framework}\label{sec:nsak_framework}

In this section we describe which technologies we choose to implement the \acrshort{nsak} framework and the resource library.

As this project is a so called monorepo, the logical primary driver for the technology stack and dependencies is the core component.
The following system components such as the resource library and the CLI will inherit the technology stack, the python and system dependencies.
For this reason, the components following the core will only describe what they introduce additionally.

\textbf{Core}\\

\todo{Describe why we choose this technology, maybe we find references which underline the ease of use and advantages for modularity which are comming with python}

\begin{itemize}
    \item Programming language: Python
    \item Dependency manager: uv
    \item Virtual environment manager: uv
    \item Package build tool: uv
    \item Linter: ruff
    \item Formatter: ruff
    \item Type checker: mypy
    \item Testing framework: pytest
\end{itemize}

Python dependencies:
\begin{itemize}
    \item pyyaml: Library for loading, validating and reading yaml files
    \item scapy: Library for red team operations
    \item pre-commit: Package to enforce code quality tools for each commit
\end{itemize}

\textbf{System Dependencies}\\
As we leverage the abstraction of \acrshort{oci} containers to run scenarios in an encapsulated environment, we have only a minimal set of system dependencies.
All system dependencies that are required for running a drill or a scenario are installed into the scenario image, during the build process.

\begin{itemize}
    \item Version control: git
    \item Network tooling: iptables (we should switch to nf\_tables)
    \item \acrshort{oci} container manager: podman
    \item \acrshort{oci} container orchestrator: podman-compose
    \item Programming languages: python3, python3-pip, uv
    \item Utilities: curl, sudo
\end{itemize}

\textbf{Resource library}\\
\\

\textbf{CLI}\\

Python dependencies:
\begin{itemize}
    \item click: Library for building CLIs
\end{itemize}

\section{\acrshort{mitm} (ARP-spoofing / transparent TCP Proxy)}\label{sec:mitm_arp_spoofing_transparent_tcp_proxy}

\textbf{Network Topology / Environment}

\begin{figure}[H]
    \centering
    \includegraphics[width=0.7\linewidth]{../assets/figures/diagrams/concept-architecture-diagrams/mitm_arp_spoofing_transparent_tcp_proxy}
    \caption{}
    \label{fig:mitm-arp-spoofing-transparent-tcp-proxy}
\end{figure}

Figure~\ref{fig:mitm-arp-spoofing-transparent-tcp-proxy} illustrates the network topology used for the \acrfull{mitm} scenario.
It consists of two dotted circles separating the network boundaries of two networks.
The left circle is describing a simple \acrshort{tcp} client - server environment consisting of Alice (client), Bob (server), a network switch, and the \acrshort{nsak} device which was placed in the target network.
The right circle is sketching the management network, which is used by a red team operator to remotely execute the \acrshort{mitm} scenario on the \acrshort{nsak} device.

\textbf{\acrshort{mitm} scenario implementation}\\
This section describes the technical realization of the \acrshort{mitm} scenario within the \acrshort{nsak} framework.

\todo{Lücku: Complete this section}

\textbf{Simple TCP client - server environment}\\
For development and testing purposes the simple \acrshort{tcp} client - server environment was added as a resource to the \acrshort{nsak} repository under~\texttt{lib/environments/simple\_tcp\_client\_server}.
The resource includes a virtualized lab setup with podman/docker compose file (\texttt{lib/environments/simple\_tcp\_client\_server/compose.yaml}) and provides an extensive readme file (\texttt{lib/environments/simple\_tcp\_client\_server/README.md}), describing the setup of this environment as a physical lab environment.
This physical lab setup builds the basis for the experiment, verifying that the \acrshort{mitm} scenario works in a real world environment.
The compose file provides a possible representation of the target network with containers acting as Alice and Bob.
The containers entrypoints are two Python scripts, which are implementing the behavior as a \acrshort{tcp} client for Alice and as a \acrshort{tcp} server for Bob.
Because the default network driver used by podman and docker compose is abstracting away the data link of the OSI model,
the configuration of the macvlan network driver is required to simulate the networks behavior on layer 2 correctly.
The \acrshort{mitm} scenario resource under~\texttt{lib/scenarios/mitm/} contains a subfolder containing another readme file (\texttt{lib/scenarios/mitm/environments/simple\_tcp\_client\_server/README.md}), describing its integration in this physical lab environment.
This subfolder also provides a podman/docker compose file (\texttt{lib/scenarios/mitm/environments/simple\_tcp\_client\_server/compose.yaml}) which extends the compose file of this resource, providing a complete setup for running the scenario in a simulation.
The \acrshort{nsak} framework was then extended with the ability to simulate a scenario in a compatible environment which provides a podman/docker compose implementation.


\section{Rogue Access Point}\label{sec: Rogue Access Point}

\textbf{Network Topology}

\begin{figure}[H]
    \centering
    \includegraphics[width=0.7\linewidth]{../assets/figures/diagrams/concept-architecture-diagrams/rogue-ap}
    \caption{}
    \label{fig:rogue-ap}
\end{figure}

Figure~\ref{fig:rogue-ap} illustrates the network topology used for the Rogue Access Point scenario.
The NSAK device is positioned between the wireless clients and an upstream host system acting as an internet gateway.
Two network interfaces are used on the NSAK device: a wireless interface (wlan0) operating in access point mode,
and a wired uplink interface (eth1) connected to the host system.

\textbf{Rogue Access Point Implementation}
This section describes the technical realization of the Rogue Access Point scenario within the NSAK framework.
The focus lies on the integration of wireless access point functionality, traffic forwarding,
and packet capture.


Within the NSAK framework, the Rogue Access Point scenario is implemented as a composition of multiple drills.
Each drill encapsulates a single operational responsibility, allowing the scenario to orchestrate the drills separately,
ant to remain modular and extensible.

\textbf{Scenario}
The Rogue Access Point scenario consists of several drills that are executed sequentially to establish a functional
wireless access point capable of intercepting and forwarding network traffic.

\begin{itemize}
    \item Network interface preparation
    \item Wireless access point initialization
    \item Traffic forwarding and network address translation
    \item Packet capture and monitoring
\end{itemize}

\textbf{Drills}


\textbf{The Hostapd Drill} is responsible for configuring the wireless network interface of the NSAK device in access
mode.
This includes assigning network parameters to the interfaces and enabling beacon transmission to allow the client
devices to connect to the rogue access point


The access point functionality is implemented using standard Linux networking services running in an isolated subprocesses.
The controlled interaction with the operating system allows reliable startup and shutdown behavior.
Furthermore, the current process state can be tracked.


\textbf{Traffic Forwarding and Network Integration drills} are providing network connectivity for clients.
The NSAK device establishes an uplink connection to an external network interface.
Traffic forwarding is enabled between the wireless and uplink interfaces, enabling transparent internet access.


Network address translation and packet forwarding are configured dynamically during scenario execution.
This enables the NSAK device to operate as an intermediary between wireless clients and the upstream network.


In parallel, a \textbf{traffic capture drill} on the connected interface captures traffic passing through.
The pcap files can be used for later analysis, enabling the evaluation of client behavior and the network.
interactions.

By separating packet capture into an independent drill, the framework allows traffic monitoring to
be reused across different scenarios without modification.

The scenario manager orchestrates the execution of all drills involved in the Rogue Access Point scenario.
Drills are executed in a predefined order, ensuring that the required network services are available before
dependent components are started.


\textbf{Error handling and cleanup}\
To prevent persistent system modifications, each drill defines a cleanup routine that can be executed after scenario
completion or upon failure.
This ensures that network interfaces and system services are restored to their original state.
In the current state of the POC the cleanup functionality need to be adjusted for the broad diversity of the drills
and covers momentarily not all possible edge cases.

But as mentioned in, a centralized cleanup mechanism ensures that partial execution states do not persist in the system
in an inconsistent configuration.
And helps to prevent uncontrolled behaviors of drills.

\textbf{}
This section focused on the technical realization of the Rogue Access Point scenario.
The effectiveness and behavior of the scenario under real network conditions are evaluated in the subsequent evaluation
chapter.



    \chapter{Evaluation and Discussion}\label{ch:evaluation_and_discussion}
    \section{Evaluation}\label{sec: Evaluation}

\subsection{Generel Functionality of a Rogue Access Point}\label{subsec: General Functionality}
The Rogue Access Point scenario was executed with different combinations of drills to test the modularity and the
independent drill execution.
The expected outcome is the creation of an open Wi-fi access point broadcasting the configured SSID,
providing DHCP leases and forwarding traffic to the uplink interface.
Additionally, traffic will be captured in PCAP format.


During execution, the access point was successfully created, and client devices could connect using the assigned IP address.
via DHCP.
The Network traffic was captured in a PCAP file.

\subsection{Research Questions}\label{subsec: Research Questions}
\textbf{RQ1 – Modularity}\label{rq1}
RQ1: Can network security scenarios be composed of independent drills without modifying the core framework?


To ensure modularity, every drill should be able to work independently.
When setting the DISABLE-DRILL environment variable, remaining drills should execute independently without failure.
For example, if everything is disabled beside the ap-mode drill, the network should still be discoverable from clients.
In T4~\ref{tab:modular-execution}, IP forwarding, uplink, and capture should not be executed and therefore are not working.

During the process, each drill has a specific attack goal or functionality.
No implicit dependencies between drills were observed during execution.


\begin{table}[H]
    \centering
    \begin{tabularx}{\textwidth}{l X X}
        \toprule
        \textbf{Test} & \textbf{Description} & \textbf{Expectation} \\
        \midrule
        T1 & Execute Rogue AP scenario with full drill set & Scenario executes successfully \\
        T2 & Deactivate packet capture drill & AP still functions \\
        T3 & Deactivate dnsmasq drill & All drills are working, tshark cannot capture traffic \\
        T4 & Only the AP-mode drill is enabled & Client can see the SSID and connect \\
        \bottomrule
    \end{tabularx}
    \caption{Modular execution test cases}
    \label{tab:modular-execution}
\end{table}

\textbf{RQ2 – Reusability}\label{rq2}
RQ2: Can individual drills be reused across different scenarios with minimal configuration effort?


The hostapd drill was reused in two scenarios: a standalone access point and a rogue access point.
access point scenario.
In both cases, the drill implementation remained unchanged.
Scenario-specific behavior was controlled through environment variables such as SSID, channel, and country code.


These results confirm that drills are scenario-orchestrated and can be composed flexibly with minimal modification.

\textbf{RQ3 – Containerization}\label{rq3}
RQ3: Does container-based execution provide sufficient isolation while maintaining required network functionality?


The evaluation focuses on whether container-based execution restricts or enables low-level network operations,
required for the functionality of \acrshort{nsak}.


The Rogue Access Point scenario was executed inside a containerized \acrshort{nsak} environment on an embedded Linux device.
The container was granted only the necessary capabilities for network configuration and packet capture.


During scenario execution, all network services were started in privileged mode within the container environment.
During container-based execution, network interfaces, routing rules, and packet capture were successfully initialized.
The drills ap-mode, network setup, dnsmasq, t-shark left certain config files and rules in iptables and filepaths.

\textbf{RQ4 – Reproducibility}\label{rq4}
RQ4: Can scenarios be executed repeatedly without changing the system behavior?


With the same configuration, the scenario execution was triggered multiple times.
Devices could establish a connection with the access point.
Network connectivity was enabled in all runs.
And the \acrshort{nsak} framework was able to capture the traffic.


No deviations in system behavior were observed across repeated executions.

    \section{Discussion}\label{sec: Discussion}


This section discusses the results of the evaluation and the research questions defined in Section~\ref{subsec: Research Questions}


\textbf{Modularity of Drills (RQ1)}~\ref{rq1}


The evaluation demonstrates that network-composed scenarios can be built from different combinations of drills without
requiring modifications to the core framework.
All tested drill combinations executed successfully.
Disabling individual or multiple drills did not cause unintended side effects and unwanted system behaviors.
This indicates the loose coupling between drills.


However, the observed modularity is currently limited to the tested scenarios.
More complex scenarios that target a more demanding network may introduce unintended dependencies that lead to unwanted
behaviors, which should be investigated more thoroughly in future work.


\textbf{Reusability Across Scenarios (RQ2)}~\ref{rq2}


The reuse of different drills in multiple scenarios suggests that drills can be reused with minimal configuration effort.
Scenario-specific behavior was controlled exclusively through environment variables and build arguments.
Drill implementation remained unchanged.
This demonstrates the assumption that the \acrshort{nsak} framework is flexible and be orchestrated from a scenario perspective.
The minimal configuration helps reduce complexity for network security teams.
The flexibility between the inline and Wi-Fi attack scenarios demonstrates the spectrum and potential of network attacks.


Reusability was evaluated across a limited number of scenarios and drills.
Further evaluation across a wider range of scenarios is required to validate in-depth.


\textbf{Containerization and Isolation (RQ3)}~\ref{rq3}


The isolated scenario container can be pre-built and run independently with an individual drill set.
The container execution in the test scenario was triggered solely from the command line,
but could also be called similarly from a system process.
This enables the \acrshort{nsak} device to remain latent in a network until the attack is most likely to succeed or to have the greatest impact.


At the same time, the evaluation revealed the necessity to manipulate the host-network configuration, such as
Iptables rules and temporary files were not fully cleaned up after scenario execution.
This highlights the importance of an improved cleanup mechanism to ensure isolation and reproducibility in repeated execution.
The running containers are constantly running, increasing the likelihood of detection.
In a more sophisticated approach, the \acrshort{nsak} device should be as stealthy as possible to remain hidden from blue teams.


\textbf{Reproducibility or Scenarios (RQ4)}~\ref{rq4}
Repeated execution of identical scenarios resulted in consistent system behavior.
This suggests that the current framework and implementation provide sufficient security reproducibility.
experiments.


However, the reproducibility was assessed over a limited number of runs and devices.
Long-term testing would be required to evaluate stability.



    \chapter{Future Work and Conclusion}\label{ch:future_work_and_conclusion}
    \section{Future work}\label{sec:future-work}
While the presented proof of concept demonstrates the modularity and scenario-based network security framework, several aspects require extension and investigation.

\textbf{\acrshort{rest} \acrshort{api}}\\
A key area for future work is the introduction of a \acrshort{rest}ful \acrshort{api} that provides parity with the existing command-line interface.
Such an HTTP-based \acrshort{api} would enable the interoperability with many other systems and the addition of alternative frontends to the \acrshort{cli}.
A good option which would integrate nicely into the current stack would be FastAPI~\cite{fastapi}.

\textbf{\acrfull{gui}}\\
Based on the proposed \acrshort{rest} \acrshort{api}, a browser-based \acrshort{gui} would provide a lower entry barrier and improved accessibility for a wider range of users and could also help in educational settings, security training and research labs.
We thought of~\cite{vuejs} as a possible library to implement this interface, but other technologies would be equally valid.

\textbf{Cleanup Management}\\
The evaluation revealed limitations in the current cleanup procedure.
Temporary configuration artifacts and network rules were not removed, which affects the successive usage of multiple scenarios.
Furthermore, the device would be more likely to be detected.

\textbf{Configuration Management}\\
We identified the possible brittleness of scenarios in different environments, as we have to make a lot of assumptions when designing a scenario or writing a drill.
Implementing a well-designed configuration management for scenarios and drills could enable the operator to set up a scenario in a much flexible way and maybe also allow autoconfiguration for adoption during runtime.
It would also render the drills much better suited for implementation in a wider range of scenarios.

\textbf{Test Coverage}\\
Increasing test coverage is essential to validate the correctness and stability of the framework as it evolves.
Automated testing would improve reliability and support long-term maintainability.

\textbf{Advanced Scenario Management}\\
As already proposed in the component diagram~\ref{fig:project2nsakcomponentdiagram} the integration of SystemD capabilities for scenarios via unit files, would enable more complex management and scheduling of scenarios.

\textbf{Automated Reporting and Analysis}\\
Finally, integrating automated reporting and analysis of defensive responses would enable more comprehensive blue- and purple-team assessments.

    \section{Conclusion}\label{sec:conclusion}

This thesis presents a conceptual design and implementation of the Network Swiss Army Knife (NSAK).
The NSAk-device is a modular and scenario-based security framework for controlled network security and
experimentation.


The proposed architecture combines container-based isolation with reusable drills.
These drills can be composed into high-level scenarios.
The implementation demonstrated the core design objectives of the framework, and the evaluation emphasizes the
modularity, reproducibility, and container-based isolation.
Independent drills were successfully orchestrated in two differential environments, in line, on the data link layer,
with a ARP spoofing MITM scenario and over a Wi-Fi network interface to capture traffic as rogue access point.

The results indicate that the NSAK-device can reduce setup complexity for network security experiments while maintaining
a high degree of flexibility.

At the same time, this work is merely a proof of concept with clear limitations.
Supporting more sophisticated attack scenarios in complex networks require further refinement and extension.
Further improvements include advanced cleanup mechanisms, richer configuration models, and automated assessment of
defensive responses.

In conclusion, the NSAK-device provides a promising foundation stone for future extension
and open source contribution.
The independent core backend and the isolated framework structure helps to develop encapsuled drills for embedded devices
without compromising other devices.



%------------ Authorship declaration translated to main language ------------

\setupSignature{
	L. von Allmen={\includegraphics[width=.4\linewidth]{../assets/pictures/lukas-von-allmen-sign}\vskip-0.5em},
	F. Gauss={\includegraphics[width=.4\linewidth]{../assets/pictures/frank-gauss-sign}}
}
\declarationOfAuthorship


%----------- Bibliography ----------------
    \clearpage
    \bibliographystyle{unsrt}
    \bibliography{references}      % the references.bib file gets loaded

%------------ List of Figures ------------
    \listoffigures

%------------ List of Tables -------------
    \listoftables

%------------ List of Listings -----------
    \lstlistoflistings

%------------ Glossary -------------------
    \printglossary

%------------ Index ----------------------
    \clearpage
    \printindex
%------------ Appendix ----------------
    \appendix
%\include{content/appendix}
\end{document}
