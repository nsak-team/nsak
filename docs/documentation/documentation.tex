%============================ MAIN DOCUMENT ================================
% define document class
\PassOptionsToPackage{table}{xcolor}
\documentclass[
    a4paper,
    BCOR=15mm,            % Binding correction
    twoside,
    openany,
% openright,
%  headings=openright,
    bibliography=totoc,   % If enabled add bibliography to TOC
    listof=totoc,         % If enabled add lists to TOC
    monolingual,
% bilingual,
    invert-title,
]{bfhthesis}

\LoadBFHModule{listings,terminal,boxes}
%---------------------------------------------------------------------------
% Documents paths
%---------------------------------------------------------------------------
\makeatletter
\def\input@path{{content/}}
%or: \def\input@path{{/path/to/folder/}{/path/to/other/folder/}}
\makeatother
%-----------------  Base packages     --------------------------------------
% Include Packages
\usepackage[french,ngerman,main=english]{babel}  % https://www.namsu.de/Extra/pakete/Babel.html
%% To disable the french list setting you can add -- see https://gitlab.ti.bfh.ch/bfh-latex/bfh-ci/-/issues/166
%\frenchsetup{StandardLists=true}

\usepackage{amsmath}          % various features to facilitate writing math formulas
\usepackage{amsthm}           % enhanced version of latex's newtheorem
\usepackage{amsfonts}         % set of miscellaneous TeX fonts that augment the standard CM
\usepackage{amssymb}          % mathematical special characters

\usepackage{siunitx}

\usepackage{graphicx}         % integration of images
\usepackage{float}            % floating objects

\usepackage{caption}          % for captions of figures and tables
\usepackage{subcaption}       % for subcaptions in subfigures
\usepackage{cite}             % use bibtex
\usepackage{wrapfig}

\usepackage{exscale}          % mathematical size corresponds to textsize
\usepackage{multirow}         % multirow emables combining rows in tables
\usepackage{multicol}

\usepackage{parskip}
\usepackage{tabularx}
\usepackage{graphicx}
\usepackage{pifont}
\usepackage{array}
\usepackage{longtable}
\usepackage{booktabs}

%---------------------------------------------------------------------------
% Graphics paths
%---------------------------------------------------------------------------
\graphicspath{{pictures/}{figures/}}
%---------------------------------------------------------------------------
% Blind text -> for dummy text
%---------------------------------------------------------------------------
\usepackage{blindtext}
\usepackage{letltxmacro}
\LetLtxMacro{\blindtextblindtext}{\blindtext}

\RenewDocumentCommand{\blindtext}{O{\value{blindtext}}}{
    \begingroup\color{BFH-Gray}\blindtextblindtext[#1]\endgroup
}
%---------------------------------------------------------------------------
% Glossary Package
%---------------------------------------------------------------------------
% the glossaries package uses makeindex
% if you use TeXnicCenter do the following steps:
%  - Goto "Ausgabeprofile definieren" (ctrl + F7)
%  - Select the profile "LaTeX => PDF"
%  - Add in register "Nachbearbeitung" a new "Postprozessoren" point named Glossar
%  - Select makeindex.exe in the field "Anwendung" ( ..\MiKTeX x.x\miktex\bin\makeindex.exe )
%  - Add this [ -s "%tm.ist" -t "%tmx.glg" -o "%tm.gls" "%tm.glo" ] in the field "Argumente"
%
% for futher informations go to http://ewus.de/tipp-1029.html
%---------------------------------------------------------------------------
\usepackage[nonumberlist]{glossaries-extra}
\makeglossaries
\newacronym{arp}{ARP}{Address Resolution Protocol}

\newglossaryentry{BlueTeam}{
    name=Blue Team,
    description={The blue team is responsible for protective measures within an organization; during an exercise, they detect and defend against red team engagements.}
}

\newacronym{cli}{CLI}{Command Line Interface}

\newacronym{dhcp}{DHCP}{Dynamic Host Configuration Protocol}

\newacronym{dns}{DNS}{Domain Name System}

\newglossaryentry{GreenTeam}{
    name=Green Team,
    description={The green team represents the cooperation between the blue and yellow teams to close knowledge gaps in both areas.}
}

\newacronym{ids}{IDS}{Intrusion Detection System}

\newacronym{ips}{IPS}{Intrusion Prevention System}

\newacronym{mitm}{MITM}{Man-in-the-Middle}

\newglossaryentry{ProxyServer}{
  name=Proxy Server,
  description={An intermediary service between a client and a server.}
}

\newacronym{nsak}{NSAK}{Network Swiss Army Knife}

\newacronym{oci}{OCI}{Open Container Initiative}

\newglossaryentry{OrangeTeam}{
    name=Orange Team,
    description={The orange team represents the cooperation between the red and yellow teams, primarily for sharing insights, knowledge, and education.}
}

\newglossaryentry{PurpleTeam}{
    name=purple team,
    description={Purple teaming describes the cooperation between red and blue teams to enhance mutual understanding and defensive capabilities.}
}

\newglossaryentry{RedTeam}{
    name=red team,
    description={The red team consists of ethical hackers who simulate attacks against systems, networks, and software to test defensive measures.}
}

\newacronym{rogueap}{RAP}{Rogue Access Point}

\newacronym{siem}{SIEM}{Security Information and Event Management}

\newacronym{soc}{SOC}{Security Operations Center}

\newacronym{ssid}{SSID}{Service Set Identifier}

\newacronym{tcp}{TCP}{Transmission Control Protocol}

\newacronym{udp}{UDP}{User Datagram Protocol}

\newglossaryentry{WhiteTeam}{
    name=White Team,
    description={The white team consists of non-technical and technical stakeholders who provide oversight, compliance guidance, and organizational requirements during an exercise.}
}

\newglossaryentry{YellowTeam}{
    name=Yellow Team,
    description={The yellow team consists of system, network, and software architects and engineers who design and maintain the infrastructure.}
}

\newacronym{api}{API}{Application Programming Interface}

\newacronym{rest}{REST}{Representational State Transfer}

\newacronym{yaml}{YAML}{YAML Ain't Markup Language}

\newacronym{git}{GIT}{Git is a free and open source distributed version control system designed to handle everything from small to very large projects with speed and efficiency}

\newacronym{mac}{MAC}{Media Access Control}

%---------------------------------------------------------------------------
% Makeindex Package
%---------------------------------------------------------------------------
\usepackage{makeidx}
\makeindex
%\usepackage{imakeidx}          % To produce index
%\makeindex[columns=2,intoc]    % Index-Initialisation
%\makeindex[columns=3,columnseprule,columnsep,intoc]
%---------------------------------------------------------------------------
% Hyperref Package (Create links in a pdf)
%---------------------------------------------------------------------------
\usepackage[
,bookmarks
,plainpages=false
,pdfpagelabels
,pdfusetitle
,backref = {false}          % No index backreference
,colorlinks = {true}        % Color links in a PDF
,hypertexnames = {true}     % no failures "same page(i)"
,bookmarksopen = {true}     % opens the bar on the left side
,bookmarksopenlevel = {0}   % depth of opened bookmarks
,linkcolor=.
,filecolor=.
,urlcolor=.
,citecolor=.
]{hyperref}
\usepackage{xparse}
\usepackage{color}
\usepackage{todonotes}
%---------------------------------------------------------------------------

%% %% Customize Footer and Headers in Document
%% \KOMAoptions{headsepline,plainheadsepline,footsepline,plainfootsepline}%
%% \setkomafont{headsepline}{\color{BFH-DarkBlue}}% BFH-DarkBlue required bfhcolors
%% \setkomafont{footsepline}{\color{BFH-DarkBlue}}%
%% \lehead*{lehead} % the * character does replace the header on the first chapter page as well
%% \cehead*{cehead}
%% \rehead*{rehead}
%% \lohead*{lohead}
%% \cohead*{cohead}
%% \rohead*{rohead}

%% \lefoot*{lefoot}
%% \cefoot*{cefoot}
%% \refoot*{refoot}
%% \lofoot*{lofoot}
%% \cofoot*{cofoot}
%% \rofoot*{rofoot}
%---------------------------------------------------------------------------
\begin{document}
\raggedbottom  % content does not get strechted verticaly


%% Snippet do redefine German babel translations -- see FAQ latex.ti.bfh.ch for further information or
%% the package documentation https://www.ctan.org/pkg/translations
%% \redefinetranslation{German}{Advisor}{Dozentin}
%% \redefinetranslation{German}{advisor}{Dozentin} %  just for consistency
%% \redefinetranslation{German}{Author}{Autorin}
%% \redefinetranslation{German}{author}{Autorin} %  just for consistency
%% \redefinetranslation{German}{Co-advisor}{Mitbetreuerin}
%% \redefinetranslation{German}{co-advisor}{Mitbetreuerin} %  just for consistency
%% \redefinetranslation{German}{Expert}{Expertin}
%% \redefinetranslation{German}{expert}{Expertin} %  just for consistency
%% \redefinetranslation{German}{Project partner}{Projektpartnerin}
%% \redefinetranslation{German}{project partner}{Projektpartnerin} %  just for consistency

%------------ START FRONT PART ------------
    \frontmatter

    \title{Project 2}
    \subtitle{Swiss Army Knife Network Sniffer}
    \author{Frank Gauss (gausf1) and Lukas von Allmen (vonal3)}
    \institution{Bern University of Applied Sciences}
    \department{Technik und Informatik}
%    \institute{TBD}
    \version{1.0}
%    \titlegraphic*{\includegraphics{somePicture}}
    \advisor{Wenger Hansjürg}
%    \coadvisor{PhD A. Smart}
%    \projectpartner{proj partner}
%    \expert{Some expert}
    \degreeprogram{Bachelor of Science in Computer Science}
%    \setupSignature{
%        A. Muster={\includegraphics[width=.5\linewidth]{sig_muster}},
%        C. Example={\includegraphics[width=.5\linewidth]{sig_example}}
%    }


%----------------  BFH tile page   -----------------------------------------
    \maketitle
%------------ ABSTRACT        ----------------
    \addchap{Abstract}
    \section{Abstract}

One-paragraph summary of the entire study – typically no more than 250 words in length (and in many cases it is well shorter than that), the Abstract provides an overview of the study.

\pagebreak

Inhalt Abstract
– Hintergrundinformationen wie Ausgangslage, Relevanz, Forschungskontext in ein bis zwei Sätzen zusammenfassen.
– Fragestellung und Ziel explizit formulieren.
– Die wichtigsten Eckpunkte zum methodischen Vorgehen angeben, bei empirischen Studien auch Angaben zu den Daten wie
etwa die Charakteristika der Stichprobe.
– Im Hauptteil des Abstracts die relevanten Ergebnisse und deren Bedeutung mit wichtigen Kennzahlen aufführen (ca. zwei Drittel des Abstracts).
– Mit wichtigen Schlussfolgerungen oder Anwendungsmöglichkeiten das Abstract abrunden.
– Das Abstract enthält keine Quellenverweise


%------------ TABLEOFCONTENTS ----------------
    \tableofcontents

%------------ START MAIN PART ------------
    \mainmatter

    \chapter{First Part Thesis}\label{ch:first-thesis-chapter}
    \section{Introduction}\label{sec:introduction}

According to the World Economic Forum's Global Risk Report 2025, the categories \enquote{Crime and illicit economic activity incl. Cyber} and \enquote{Cyber espionage and warfare} are both ranked among the top 10 global risks in the next two to ten years~\cite{WEF2025GlobalRisks}.
These risks are expected to intensify even further because the economic and operational costs of launching cyberattacks will decrease due to AI automation~\cite{Garg2024AIEconomicsCyberattacks}.
This underlines the need for cost-effective and easy-to-use security tools, methods and frameworks (conceptional and software) to identify and defend against cyberattacks.

\textbf{Information security methods and conceptional frameworks}\\
In practice, the method of combining red team activities and blue team observation techniques is widely adopted within the cybersecurity community and industry~\cite{cremen2018infoseccolourwheel}.
While the red team focuses on emulating adversarial behavior, the primary objective of the blue team is to detect such activities through non-invasive monitoring and analysis of system behavior~\cite{NIST2012RedTeamBlueTeamApproach}.
Further, we can observe approaches from the community and the industry to evolve this approach in to the so-called \enquote{InfoSec color wheel}~\cite{cremen2018infoseccolourwheel}.
In the proposed InfoSec color wheel, the author splits the six colors into primary and secondary colors, where the primary colors are teams on their own and the secondary colors are cooperation between two primary color teams.
The primary colors are represented by the \gls{RedTeam}, the \gls{BlueTeam}, and a newly introduced the \gls{YellowTeam}, which represents the \enquote{builders} of software and systems.
The secondary colors are represented by the \gls{PurpleTeam} (red and blue), the \gls{OrangeTeam} (red and yellow), and the \gls{GreenTeam} (yellow and blue).
Where \enquote{purple teaming} is actually an already established praxis as it evolved naturally from the cooperation between red and blue teams~\cite{NIST2012RedTeamBlueTeamApproach}.


\textbf{Information security tools and software framworks}\\
One approach to reduce operational security costs is to adopt multiple modular frameworks that can be easily extended, configured, and executed continuously in a controlled manner~\cite{Zilberman2020SoKThreatEmulators}.
The threat emulation frameworks analyzed by Zilberman et al.\ evaluate multiple attack phases, including lateral movement, persistence, and attack execution.


\textbf{\acrfull{nsak}}

The Network Swiss Army Knife focuses on containerized, orchestrated scenarios that execute specific attack drills in a controlled environment.
Future extensions will focus on enriching the assessment layer by systematically capturing and evaluating defensive responses of multiple scenarios.

This proof of concept comprises the design and implementation of a modular, isolated open-source security framework that focuses on extensibility and the controlled execution of attack-based scenarios.

The objective of this work is to investigate whether such a framework can provide a flexible, extendable, and safe foundation for modular and automated security testing in a network environment. .

In the summary of their paper, Zilberman et al.\ are highlighting the necessity of the following design requirements~\cite{Zilberman2020SoKThreatEmulators}:
\begin{itemize}
    \item Cleanup and configurability are important in order to repeat and automate the execution of attack scenarios during security tool assessment and what-if analysis.
    \item An emulator should support cleanup after the completion of the attack scenario, like CALDERA, Atomic Red Team, and Infection Monkey do, rather than after each individual procedure.
    \item An API, currently provided by Atomic Red Team, CALDERA, and Metasploit, facilitates integration between the threat emulators and organizational security array, thus enabling periodic and systematic security assessment.
    \item It is important to provide a GUI and ready to execute multi-procedure attacks for novice operators as well as a \gls{cli} to support automation and advanced customization capabilities.
\end{itemize}

We reconsolidate the highlighted design requirements for the implementation of \acrshort{nsak} into the following list of features:
\begin{itemize}
    \item CLI, GUI, and API to manage resources and execute scenarios.
    \item Configurability of the framework and the resources.
    \item Automatic cleanup procedures after the completion of a scenario.
\end{itemize}

Even though we agree with the importance of the highlighted design requirements, we are not able to implement them all in the time constraints of this project.
Because we are planning to build upon this PoC, we will incorporate the design requirements in the chapter architecture and design~\ref{ch:design_and_architecture} of this paper and list them in the future work~\ref{sec:future-work} section.

    \section{Current State of Research}\label{sec:current-state-of-research}


Recent studies highlight the importance of modularity, reproducibility, and automation to reduce the operational overhead of
security assessments.
Methodologies such as red and blue teaming, and their combinations within the InfoSec color wheel, show
the complexity and overlapping disciplines in the security sector.
\cite{NIST2012RedTeamBlueTeamApproach, cremen2018infoseccolourwheel}

On a technical level, threat emulation frameworks such as CALDERA, Atomic Red Team, and Metasploit
implement multi-stage attack scenarios to evaluate the detection of defensive systems.
However, many existing frameworks focus primarily on large-scale enterprise environments and require significant
setup effort, limiting their adaptability in resource-constrained networks.
\cite{Zilberman2020SoKThreatEmulators}

These aspects highlight the need for a lightweight and modular framework to reduce overall cyber thread risks in
network infrastructures.\cite{WEF2025GlobalRisks}


	\chapter{Hardware Selection}
    \input{content/04-evaluation}
    \section{Network Environments}\label{sec:environments}

Each environment represents a practical setup in which the Network Swiss Army Knife (nsak) can be deployed. For traffic analysis, performance testing or security evaluation.

\subsection{Category I - Inline:}
\textbf{Diagram:} Laptop $\leftrightarrow$ nsak $\leftrightarrow$ Router  \\

\begin{figure}[H]
	\centering
	\includegraphics[width=0.7\linewidth]{../assets/figures/diagrams/network-diagrams/LAN-Sniffing-Switch-AccessPoint}
	\caption{}
	\label{fig:lan-sniffing-switch-accesspoint}
\end{figure}


\textbf{Description:}
Direct inline bridge between client or switch and router.
Used for basic LAN capturing, latency, and throughput testing.

\subsection{Category II Wireless:}
\textbf{Diagram:} Laptop, Smart Devices, Printer $\leftrightarrow$ nsak (inline) $\leftrightarrow$ Router  \\
\begin{figure}[H]
	\centering
	\includegraphics[width=0.7\linewidth]{../assets/figures/diagrams/network-diagrams/LAN-WLAN-Sniffing.drawio}
	\caption{}
	\label{fig:lan-wlan-sniffing}
\end{figure}

\textbf{Description:}
The nsak is inline and let traffic pass but intercepts as Rouge AP and capture data


	\chapter{Architecture and Design}
	\section{Use-Cases}\label{sec:use-cases}

\begin{figure}[H]
    \centering
    \includegraphics[width=0.7\linewidth]{../assets/figures/diagrams/concept-architecture-diagrams/NSAK-use-case-diagram}
    \caption{User-Story UML Diagram}
    \label{fig:NSAK-use-case-diagram}
\end{figure}

\textbf{Figure~\ref{fig:NSAK-use-case-diagram}} illustrates the use case structure of the proposed \gls{nsak} modular framework.
The operator interacts with \gls{nsak} primarily through two high-level commands:
Build Scenario (UC-01) and Run Scenario (UC-06).
During the Build Scenario use case (UC-01), the system builds a scenario container for execution.
In complex network infrastructures, additional configuration parameters, such as network interface mappings may be
required and need to be provided as build-time arguments (UC-02).

The system loads the selected scenario (UC-03).
At this stage, the scenario orchestrates the required drills necessary to perform the intended attack.

The build process concludes with the creation of an \gls{oci} compliant container image (UC-05), which encapsulates
the fully configured scenario.

The Run Scenario use case (UC-06) represents an abstract execution phase.
In this phase, the previously built container image is run, and the configured attack drills are executed within the containerized environment (UC-08).

Finally, the system performs a cleanup procedure which needs to be optimized in future work, in which all scenario-specific resources,
processes, and drills are terminated.
This step minimizes side effects, reduces system noise, and prevents interference with other scenarios that may reuse the same drills.


\newcommand{\uctab}{\par\noindent\hspace{1em}}
\newcommand{\ucbullet}[1]{\uctab\textbullet~#1}

\begin{longtable}{@{} >{\columncolor{blue!15}}l >{\raggedright\arraybackslash}p{0.75\textwidth} @{}}
    \caption{Use Cases Specification (NSAK)}\\
    \toprule
    \textbf{NR} \& \textbf{Details} \\
    \endfirsthead
    \toprule
    \textbf{NR} \& \textbf{Details} \\
    \endhead
    \toprule

    % --- UC-01 ----------------------------------------------------
    \textbf{UC-01} & \textbf{Use-Case:} Build Scenario \par
                     \textbf{Description:} Builds a scenario container based on a selected scenario configuration. \par
                     \textbf{Actor:} Operator \par
                     \textbf{Trigger:} Operator initiates a scenario build via the command-line interface. \par
                     \textbf{Preconditions:} \gls{nsak} initialized; scenarios available. \par
                     \textbf{Main Scenario:} \par
                     \uctab{} 1. Operator selects a scenario to build using the command-line interface. \par
                     2. System validates the selected scenario. \par
                     3. System executes the included use cases: \par
                         \ucbullet Configure Scenario (UC-2) \par
                         \ucbullet Load Scenario (UC-3) \par
                         \ucbullet Load Drills (UC-4) \par
                         \ucbullet Build \gls{oci} Container (UC-5) \par
                     \textbf{Alternative Scenarios:} No scenarios available~\rightarrow~inform the operator. \par
                     \textbf{Error Scenarios:} Conflicting scenario configuration detected~\rightarrow~build aborted. \par
                     \textbf{Result:} Scenario container successfully built. \par
                     \textbf{Postconditions:} Scenario container stored and ready to run. \\
    \midrule
    % --- UC-02 ----------------------------------------------------
    \textbf{UC-02} & \textbf{Use-Case:} Configure Scenario \par
                     \textbf{Description:} Defines scenario-specific build parameters such as network interfaces and execution options. \par
                     \textbf{Actor:} System \par
                     \textbf{Trigger:} Scenario selected for build (UC-01). \par
                     \textbf{Preconditions:} Scenario selection is available. \par
                     \textbf{Main Scenario:}
                     1. System applies scenario-specific configuration parameters. \par
                     \textbf{Result:} Scenario configuration created. \par
                     \textbf{Postconditions:} Scenario configuration available for loading. \\
    \midrule
    % --- UC-03 ----------------------------------------------------
    \textbf{UC-03} & \textbf{Use-Case:} Load Scenario \par
                     \textbf{Description:} Loads and validate the selected scenarios \par
                     \textbf{Actor:} System \par
                     \textbf{Trigger:} Scenario configuration available (UC-02). \par
                     \textbf{Preconditions:} Scenario configuration created. \par
                     \textbf{Main Scenario:} \par
                     1. System retrieves the scenario definition files (scenario.yaml, scenario.py, README.md). \par
                     2. System validates the scenario structure and resolves declared dependencies. \par
                     \textbf{Error Scenarios:} Validation or dependency failure, preparation aborted with Error Log. \par
                     \textbf{Result:} Scenario is successfully loaded. \par
                     \textbf{Postconditions:} Scenario representation available for drill loading. \\
    \midrule
    % --- UC-04 ----------------------------------------------------
    \textbf{UC-04} & \textbf{Use-Case:} Load Drills \par
                     \textbf{Description:} Loads the attack drills required by the selected scenario. \par
                     \textbf{Actor:} System \par
                     \textbf{Trigger:} Scenario loaded (UC-03). \par
                     \textbf{Preconditions:} Scenario representation is available. \par
                     \textbf{Main Scenario:} \par
                     1. System resolves drill references defined in the scenario configuration. \par
                     2. System instantiates drill objects and loads associated metadata. \par
                     \textbf{Error Scenarios:} Invalid drill definition, drill not found, or ambiguous drill reference. \par
                     \textbf{Result:} Required drill objects loaded. \par
                     \textbf{Postconditions:} Drills available for container build. \\
    \midrule
    % --- UC-05 --------------------------------------------------
    \textbf{UC-05} & \textbf{Use-Case:} Build \gls{oci} Container \par
                     \textbf{Description:} Builds an \gls{oci} compliant container image for the loaded scenario. \par
                     \textbf{Actor:} System \par
                     \textbf{Trigger:} Scenario and drills loaded (UC-03, UC-04). \par
                     \textbf{Preconditions:} Scenario representation and drill objects available. \par
                     \textbf{Main Scenario:} \par
                     1. System generates the container build context. \par
                     2. System builds the scenario container image with required privileges and network configuration. \par
                     \textbf{Error Scenarios:} Container build failure — build aborted with an error message. \par
                     \textbf{Result:} \gls{oci} compliant scenario container image built. \par
                     \textbf{Postconditions:} Scenario container image stored and ready for execution. \\
    \midrule
    %--- UC-06 --------------------------------------------------
    \textbf{UC-06} & \textbf{Use-Case:} Run Scenario \par
                     \textbf{Description:} Executes a previously built scenario container. \par
                     Specific scenarios such as Rogue AP or ARP MITM represent specialized configurations of this use case.
                     \textbf{Actor:} Operator \par
                     \textbf{Trigger:} Operator initiates scenario execution via the command-line interface. \par
                     \textbf{Preconditions:} Scenario container image available (UC-05). \par
                     \textbf{Main Scenario:} \par
                     1. System starts the scenario container with the required execution parameters. \par
                     2. System executes the included use cases: \par
                         \ucbullet Execute Scenario (UC-07) \par
                         \ucbullet Cleanup Scenario (UC-09) \par
                     \textbf{Result:} Scenario container execution started. \par
                     \textbf{Postconditions:} Scenario execution context active. \\
    \midrule
    %--- UC-07 --------------------------------------------------
    \textbf{UC-07} & \textbf{Use-Case:} Execute Scenario \par
                     \textbf{Description:} Orchestrates the execution of a previously built scenario container and coordinates
                     the execution of the associated attack drills. \par
                     \textbf{Actor:} System \par
                     \textbf{Trigger:} Run Scenario (UC-6) \par
                     \textbf{Preconditions:} Scenario Image available and started \par
                     \textbf{Main Scenario:} \par
                     1. System Scenario Manager executes for the selected scenario \par
                     2. System Drill Manager executes drill UC-8 include use-case
                     \textbf{Error Scenarios:} Scenario not found or scenario container was not available. \par
                     \textbf{Result:} Scenario execution initiated and drill execution orchestrated. \par
                     \textbf{Postconditions:} Scenario container is running and drills are being executed.\\
    \midrule
    % --- UC-08 --------------------------------------------------
    \textbf{UC-08} & \textbf{Use-Case:} Execute Drills \par
                     \textbf{Description:} Executes the attack drills defined in the scenario configuration within the running scenario
                     container. \par
                     \textbf{Actor:} System \par
                     \textbf{Preconditions:} Scenario execution context initialized. \par
                     \textbf{Main Scenario:} \par
                     1. System Drill Manager retrieves the list of configured drills. \par
                     2. System Drill Manager executes the drills according to the defined order and parameters. \par
                     \textbf{Error Scenarios:} Drill execution failure or missing drill definition. \par
                     \textbf{Result:} Configured attack drills executed. \\
    \midrule
    % --- UC-09 --------------------------------------------------
    \textbf{UC-09} & \textbf{Use-Case:} Clean Up Scenario \par
                     \textbf{Description:} Terminates the running scenario container and restores the system to a defined baseline state. \par
                     \textbf{Actor:} System \par
                     \textbf{Trigger:} Stop Scenario (UC-06) \par
                     \textbf{Preconditions:} Scenario container is running. \par
                     \textbf{Main Scenario:} \par
                     1. System stops the running scenario container. \par
                     2. System invokes the included use case Clean Up Drills (UC-10). \par
                     \textbf{Error Scenarios:} Scenario container cannot be terminated. \par
                     \textbf{Result:} Scenario execution terminated. \par
                     \textbf{Postconditions:} Scenario container stopped and removed.\\
    \midrule
    % --- UC-10 --------------------------------------------------
    \textbf{UC-10} & \textbf{Use-Case:} Clean Up Drills \par
                     \textbf{Description:} Cleans up artifacts and state changes introduced by executed attack drills. \par
                     \textbf{Actor:} System \par
                     \textbf{Preconditions:} Drill execution completed or aborted. \par
                     \textbf{Main Scenario:} \par
                     1. System Drill Manager terminates active drill processes. \par
                     2. System Drill Manager removes temporary artifacts and resets modified parameters. \par
                     \textbf{Error Scenarios:} Incomplete cleanup due to failed drill termination. \par
                     \textbf{Result:} Drill-related artifacts removed and state reset. \\
    \bottomrule
\end{longtable}

	\section{Component-diagram}

\begin{figure}[H]
	\centering
	\includegraphics[width=1\linewidth]{../assets/figures/diagrams/concept-architecture-diagrams/project_2_nsak_component_diagram}
	\caption{}
	\label{fig:project2nsakcomponentdiagram}
\end{figure}


    \chapter{Second Part Thesis: Method}\label{ch:second-thesis-chapter}
    \input{content/08-method}
    \input{content/09-implementation}
    \input{content/10-conclusion}


%------------ Authorship declaration translated to main language ------------
    \declarationOfAuthorship

%----------- Bibliography ----------------
    \clearpage
    \bibliographystyle{unsrt}
    \bibliography{references}      % the references.bib file gets loaded

%------------ List of Figures ------------
    \listoffigures

%------------ List of Tables -------------
    \listoftables

%------------ List of Listings -----------
    \lstlistoflistings

%------------ Glossary -------------------
    \printglossary

%------------ Index ----------------------
    \clearpage
    \printindex
%------------ Appendix ----------------
    \appendix
    \section{First Appendix Chapter}\label{sec:appendix}

\subsection{Project 2 Proposal}\label{subsec:project-2-proposal}


\end{document}
