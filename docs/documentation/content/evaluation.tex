\chapter{Hardware Selection}\label{sec:evaluation}

\section{Requirements}
The following requirements were defined for the hardware platform used in this project:
\begin{itemize}
	\item At least two native Ethernet interfaces for inline packet sniffing
	\item Support for 2.5~GbE or higher
	\item Onboard Wi-Fi with access point (AP) and monitor mode support
	\item Low power consumption suitable for 24/7 operation
	\item Compact form factor for laboratory and prototype setups
	\item Strong community and software support 
	\item Affordable cost (below 150 CHF)
\end{itemize}
\newpage
\section{Evaluated Boards}

Several boards were considered as potential variants. Their main specifications relevant to the project are listed in Table~\ref{tab:boardspecs}.

\begin{table}[H]
	\centering
	\caption{Comparison of Board Variants}
	\label{tab:boardspecs}
	\scriptsize % kleinere Schrift für den Inhalt
	\begin{tabularx}{\textwidth}{|X|X|X|X|X|X|}
		\hline
		\textbf{Board} & \textbf{SoC / CPU} & \textbf{RAM / Storage} & \textbf{Ethernet Ports} & \textbf{Power (typ.)} & \textbf{Wireless (onboard)} \\
		\hline
		Banana Pi R3 Mini & MT7986A, Quad-core ARM Cortex-A53 @ 1.3 GHz & 2 GB DDR4, 8 GB eMMC, microSD & 2 × 2.5 GbE & 5–7 W & MT7976C, Wi-Fi 6 (AP/Client/Monitor) \\
		\hline
		Banana Pi R3 & MT7986A, Quad-core ARM Cortex-A53 @ 1.3 GHz & 2–4 GB DDR4, eMMC, microSD & 1 × 1 GbE, 2 × 2.5 GbE, 4 × 1 GbE & 7–10 W & MT7976C, Wi-Fi 6 \\
		\hline
		Banana Pi R4 & MT7988A, Quad-core ARM Cortex-A73 & 4 GB DDR4, NVMe option & 4 × 2.5 GbE, 2 × 10 GbE (SFP+) & 10–15 W & None (M.2 Wi-Fi module required) \\
		\hline
		Banana Pi R5 & MT7988B, Quad-core ARM Cortex-A73 & 4 GB DDR4, NVMe option & 2 × 10 GbE, 2 × 2.5 GbE & 12–18 W & None (M.2 Wi-Fi module required) \\
		\hline
		Raspberry Pi 4 & BCM2711, Quad-core ARM Cortex-A72 @ 1.5 GHz & 2–8 GB LPDDR4, microSD & 1 × 1 GbE (second via USB dongle) & 6–8 W & Wi-Fi 5 (AP/Client only) \\
		\hline
		Raspberry Pi 5 & BCM2712, Quad-core ARM Cortex-A76 @ 2.4 GHz & 4–8 GB LPDDR4X, microSD & 1 × 1 GbE (second via PCIe card) & 8–12 W & Wi-Fi 5 (AP/Client only) \\
		\hline
		NanoPi R76S & Rockchip RK3588S, Octa-core (4× Cortex-A76 @ 2.4 GHz + 4× Cortex-A55 @ 1.8 GHz) & 16 GB LPDDR4X / LPDDR5, NVMe (option via M.2) & 3 × 2.5 GbE (RJ45) & 10–15 W & None (M.2 Wi-Fi 6E module recommended) \\
		\hline
		
	\end{tabularx}
\end{table}



\begin{table}[H]
	\centering
	\caption{Requirements Fulfillment by Candidate Boards}
	\label{tab:reqcheck}
	\resizebox{\textwidth}{!}{%
		\begin{tabular}{|p{6cm}|c|c|c|c|c|c|c}
			\hline
			\textbf{\large Requirement} & \textbf{\large R3 Mini} & \textbf{\large R3} & \textbf{\large R4} & \textbf{\large R5} & \textbf{\large RPi 4} & \textbf{\large RPi 5} & \textbf{NanoPi R76S}\\
			\hline
			\scriptsize ≥ 2 native Ethernet interfaces & \ding{51} & \ding{51} & \ding{51} & \ding{51} & \ding{55} & \ding{55} & \ding{51} \\
			\hline
			\scriptsize RAM > 4GB & \ding{55} & \ding{55} & \ding{51} &\ding{51} &
			\ding{55} & \ding{51} & \ding{51} \\
			\hline
			\scriptsize ≥ 2.5 GbE support & \ding{51} (2×) & \ding{51} (2×) & \ding{51} \ding{51} (4×) & \ding{51} (2×) & \ding{55} & \ding{55} & \ding{51} \\
			\hline
			\scriptsize Onboard Wi-Fi with AP \& Monitor mode & \ding{51} & \ding{51} & \ding{55} & \ding{55} & \ding{55} & \ding{55} & \ding{55} \\
			\hline
			\scriptsize Low power consumption (<10 W) & \ding{51} & \ding{51}/\ding{115} & \ding{55} & \ding{55} & \ding{51} & \ding{115} & \ding{51}\\
			\hline
			\scriptsize Compact form factor & \ding{51} & \ding{55} & \ding{55} & \ding{55} & \ding{51} & \ding{51} & \ding{51}\\
			\hline
			\scriptsize Strong community \& software support & \ding{51} & \ding{51} & \ding{115} & \ding{115} & \ding{51} (general) & \ding{51} (general) & \ding{51}\\
			\hline
			\scriptsize Suitable for inline packet sniffing & \ding{51} & \ding{51} (overkill) & \ding{115} (overkill) & \ding{115} (expensive) & \ding{55} & \ding{55} & \ding{51 }\\
			\hline
		\end{tabular}
	}
	\vspace{0.5em}
	\scriptsize
	\textbf{Legend:} \ding{51} = Requirement fulfilled, \ding{55} = Requirement not fulfilled, \ding{115} = Partially fulfilled / limited
\end{table}



\section{Decision}

Based on the defined requirements and the evaluation of alternatives, the \textbf{Banana Pi R4} and the\textbf{NanoPI R76S} are most suitable hardware platform for this prototype implementation.

The Banana Pi R4 offers two native 2.5~GbE interfaces for inline sniffing the board is compact, affordable, and supported by a strong community. In Addition, the two 10 GbE SFP+ ports provide flexibility for extensions as fiber-based packet capturing. A drawback of the R4 is the weaker CPU and a larger size compared to the NanoPI R76S

The NanoPi R76S is more compact and provides up to 16GB of RAM, which is advantageous for memory intensive processing and buffering tasks. While it lacks built-in Wi-fi, it can be expanded via the M.2 Wi-Fi 6E module. It can not host both a Wi-Fi card and NVMe SSD simultaneously. Consequently, data storage must be provided via microSD card or external USB SSD

Alternative boards such as the Banana Pi R3 Mini, R3 are limited overall performance. Rasberry PI 4 or 5 offer higher single core performance but were ultimately discarded because they provide only a single native Ethernet interface, requiring external adapters that reduce performance for inline sniffing scenarios. 

\section{Module}

\begin{table}[H]
	\centering
	\renewcommand{\arraystretch}{1.2}
	\begin{tabular}{|c|p{4cm}|p{8cm}|}
		\hline
		\textbf{ID} & \textbf{Module Name} & \textbf{Description} \\
		\hline
		M1 & Hardware Evaluation \& Setup & Evaluation and selection of suitable embedded hardware (e.g., Banana Pi R4, NanoPi R6S). Setup of OS, drivers, and base configuration. \\
		\hline
		M2 & System Configuration & Installation and configuration of the operating system, networking stack, and essential system packages. \\
		\hline
		M3 & Network Capture Engine & Implementation of inline network sniffing using pcap/tcpdump, bridge configuration, and throughput validation. \\
		\hline
		M4 & Traffic Filtering \& Manipulation & Integration of firewall, packet filtering, and traffic injection capabilities (iptables/nftables). \\
		\hline
		M5 & Performance Measurement & Development of measurement tools for throughput, latency, and jitter (iperf3, netperf, ping tests). \\
		\hline
		M6 & Attack Simulation & Controlled execution of network attacks such as MITM, downgrade, or ARP spoofing for testing purposes. \\
		\hline
		M7 & Switching \& Routing (SDN) & Implementation of dynamic routing or bridging functions, optional integration with SDN controllers. \\
		\hline
		M8 & VPN Endpoint & Configuration of a secure VPN endpoint (WireGuard or OpenVPN) supporting host-to-network and network-to-network connections. \\
		\hline
		M9 & Wi-Fi Client Mode & Setup of Wi-Fi module as a client endpoint for wireless measurements and connectivity. \\
		\hline
		M10 & Wi-Fi Access Point Mode & Setup of hostapd to run the board as a wireless access point, including channel and signal analysis. \\
		\hline
		M11 & Mobile Network (Tethering) & Integration of LTE/5G USB modem for mobile uplink or hotspot functionality. \\
		\hline
		M12 & Backend Communication & Secure communication with a remote backend or control server via REST API or MQTT. \\
		\hline
		M13 & User Interface (CLI / Web) & Development of a local command-line and/or web-based configuration interface for module control. \\
		\hline
		M14 & System Security \& Hardening & Implementation of SSH key management, firewall, certificates, and user authentication for secure operation. \\
		\hline
		M15 & Monitoring \& Logging & Collection of system logs, traffic statistics, and implementation of rotation and data persistence. \\
		\hline
		M16 & Documentation \& Reporting & Continuous documentation of development progress, project report, and final presentation materials. \\
		\hline
	\end{tabular}
	\caption{Overview of project modules for the \textit{Network Swiss Army Knife} prototype.}
	\label{tab:modules_overview}
\end{table}



\section{Diagram}


\subsection{Variant 1 Inline-Sniffing}
\begin{figure}[H]
	\centering
	\includegraphics[width=0.7\linewidth]{../../../Media/LAN-Sniffing}
	\caption{}
	\label{fig:lan-sniffing}
\end{figure}
\subsection{Variant 2}
\begin{figure}[H]
	\centering
	\includegraphics[width=0.7\linewidth]{../../../Media/LAN-Sniffing-Switch-AccessPoint}
	\caption{}
	\label{fig:lan-sniffing-switch-accesspoint}
\end{figure}
\subsection{Variant 3}
\subsection{Variant 4}
\subsection{Variant 5}

