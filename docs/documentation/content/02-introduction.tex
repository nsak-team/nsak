2.2 Einleitung
Die Einleitung führt einerseits zum Thema hin (Ausgangslage), und informiert andererseits darüber, warum (Fragestellung/Problem) und
wozu (Ziel/Zweck) es die Arbeit gibt sowie ggf. wie sie zustande gekommen ist (methodisches Vorgehen). Die Einleitung kann je nach
Umfang und Thema mit oder ohne Unterkapitel verfasst werden. Sie umfasst ca. 5–10 % der Arbeit. Sie nimmt keine Ergebnisse vorweg.
Einen Überblick über die Kapitel der Arbeit gibt es höchstens bei sehr langen Arbeiten (beispielsweise Masterarbeit).
Die Ausgangslage beschreiben
– Relevanz: Warum ist dieses Thema überhaupt bedeutsam? Schon hier gilt es, nicht einfach etwas zu behaupten, sondern Fakten
und Aussagen mit Fachliteratur zu belegen.
– Aktualität: Gibt es einen aktuellen Bezug?
– Zusammenhang: Wie lässt sich das Thema einordnen? In welchem fachlichen Kontext steht die Arbeit?
– Forschungsstand: Was ist schon erforscht? Gibt es bereits Untersuchungen? (Ist-Zustand und Zusammenhang mit dem eigenen
Thema.) Je nach Art und Umfang der Arbeit gibt es zum Wissensstand ein separates Kapitel (siehe 2.3).
– Wenn vorhanden externe Auftraggeber, Auftraggeberinnen: Wer sind die beteiligten Partnerinnen oder Stakeholder?
– Den Kurs bzw. das Modul, in dem die Arbeit entsteht, erwähnt man auf dem Titelblatt (siehe 4.1).
8
Das Problem und das Ziel darstellen
– Zweck der Arbeit: Welche Aufgabe, Herausforderung, welches Problem soll gelöst werden? Warum sollte man die Arbeit lesen?
– Fragestellung: Auf welche konkrete Hauptfrage (evtl. mit konkretisierenden Unterfragen) soll im Schlusskapitel eine fundierte
Antwort gegeben werden? Ist die Fragestellung genügend eingegrenzt? Gibt es Hypothesen?
– Abgrenzung: Wo liegen die Grenzen der Untersuchung (zeitlich, geografisch, thematisch, methodisch, in der Auswahl der Hilfsmittel usw.)? Was wird nicht untersucht? Was kann die Arbeit nicht leisten?
– Ziel: Was soll die Untersuchung genau bewirken? Was ist die Absicht hinter der Arbeit?
– Erwartung: Was möchte die Arbeit leisten (Nutzen der Untersuchung, Soll-Zustand)? Für welche konkrete Zielgruppe sind die
Ergebnisse der Arbeit von Nutzen? Was ist zu erwarten? Was ist nicht zu erwarten?
Das methodische Vorgehen andeuten
Wie man methodisch vorgeht, wird ausführlich im Methodenkapitel beschrieben (siehe 2.4). Oft erwähnt man aber in der Einleitung bereits in ein bis zwei Sätzen, mit welcher Methode man arbeitet (Literaturarbeit, Umfrage, Entwickeln eines Prototyps,
Variantenstudium usw.).
