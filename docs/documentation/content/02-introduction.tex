\section{Introduction}\label{sec:introduction}

According to the World Economic Forum's Global Risk Report 2025, the categories \enquote{Crime and illicit economic activity incl. Cyber} and \enquote{Cyber espionage and warfare} are both ranked among the top 10 global risks in the next two to ten years~\cite{WEF2025GlobalRisks}.
These risks are expected to intensify even further because the economic and operational costs of launching cyberattacks will decrease due to AI automation~\cite{Garg2024AIEconomicsCyberattacks}.
This underlines the need for cost-effective and easy-to-use security tools, methods and frameworks (conceptional and software) to identify and defend against cyberattacks.

\textbf{Information security methods and conceptional frameworks}\\
In practice, the method of combining red team activities and blue team observation techniques is widely adopted within the cybersecurity community and industry~\cite{cremen2018infoseccolourwheel}.
While the red team focuses on emulating adversarial behavior, the primary objective of the blue team is to detect such activities through non-invasive monitoring and analysis of system behavior~\cite{NIST2012RedTeamBlueTeamApproach}.
Further, we can observe approaches from the community and the industry to evolve this approach in to the so-called \enquote{InfoSec color wheel}~\cite{cremen2018infoseccolourwheel}.
In the proposed InfoSec color wheel, the author splits the six colors into primary and secondary colors, where the primary colors are teams on their own and the secondary colors are cooperation between two primary color teams.
The primary colors are represented by the \gls{RedTeam}, the \gls{BlueTeam}, and a newly introduced the \gls{YellowTeam}, which represents the \enquote{builders} of software and systems.
The secondary colors are represented by the \gls{PurpleTeam} (red and blue), the \gls{OrangeTeam} (red and yellow), and the \gls{GreenTeam} (yellow and blue).
Where \enquote{purple teaming} is actually an already established praxis as it evolved naturally from the cooperation between red and blue teams~\cite{NIST2012RedTeamBlueTeamApproach}.


\textbf{Information security tools and software framworks}\\
One approach to reduce operational security costs is to adopt multiple modular frameworks that can be easily extended, configured, and executed continuously in a controlled manner~\cite{Zilberman2020SoKThreatEmulators}.
The threat emulation frameworks analyzed by Zilberman et al.\ evaluate multiple attack phases, including lateral movement, persistence, and attack execution.


\textbf{\acrfull{nsak}}

The Network Swiss Army Knife focuses on containerized, orchestrated scenarios that execute specific attack drills in a controlled environment.
Future extensions will focus on enriching the assessment layer by systematically capturing and evaluating defensive responses of multiple scenarios.

This proof of concept comprises the design and implementation of a modular, isolated open-source security framework that focuses on extensibility and the controlled execution of attack-based scenarios.

The objective of this work is to investigate whether such a framework can provide a flexible, extendable, and safe foundation for modular and automated security testing in a network environment. .

In the summary of their paper, Zilberman et al.\ are highlighting the necessity of the following design requirements~\cite{Zilberman2020SoKThreatEmulators}:
\begin{itemize}
    \item Cleanup and configurability are important in order to repeat and automate the execution of attack scenarios during security tool assessment and what-if analysis.
    \item An emulator should support cleanup after the completion of the attack scenario, like CALDERA, Atomic Red Team, and Infection Monkey do, rather than after each individual procedure.
    \item An API, currently provided by Atomic Red Team, CALDERA, and Metasploit, facilitates integration between the threat emulators and organizational security array, thus enabling periodic and systematic security assessment.
    \item It is important to provide a GUI and ready to execute multi-procedure attacks for novice operators as well as a \gls{cli} to support automation and advanced customization capabilities.
\end{itemize}

We reconsolidate the highlighted design requirements for the implementation of \acrshort{nsak} into the following list of features:
\begin{itemize}
    \item CLI, GUI, and API to manage resources and execute scenarios.
    \item Configurability of the framework and the resources.
    \item Automatic cleanup procedures after the completion of a scenario.
\end{itemize}

Even though we agree with the importance of the highlighted design requirements, we are not able to implement them all in the time constraints of this project.
Because we are planning to build upon this PoC, we will incorporate the design requirements in the chapter architecture and design~\ref{ch:design_and_architecture} of this paper and list them in the future work~\ref{sec:future-work} section.
