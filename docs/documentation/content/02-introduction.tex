\section{Introduction}\label{sec:introduction}

\todo{Braucht es mehr quellen? sind alle Punkte erfüllt}

According to the World Economic Forum’s Global Risk Report 2025, the categories \enquote{Crime and illicit economic activity incl. Cyber} and \enquote{Cyber espionage and warfare} are both ranked among the top 10 global risks in the next two to ten years~\cite{WEF2025GlobalRisks}.
These risks are expected to intensify even further because the economic and operational costs of launching cyberattacks will decrease due to AI automation~\cite{Garg2024AIEconomicsCyberattacks}.
This underlines the need for cost-effective and easy-to-use security tools, methods and frameworks (conceptional and software) to identify and defend against cyberattacks.

\subsection{Information security methods and conceptional frameworks}\label{subsec:information-security-methods-and-conceptional-frameworks}
In practice, the method of combining red team activities and blue team observation techniques is widely adopted within the cybersecurity community and industry~\cite{}.
While the red team focuses on emulating adversarial behavior, the primary objective of the blue team is to detect such activities through non-invasive monitoring and analysis of system behavior~\cite{NIST2012RedTeamBlueTeamApproach}.
Further, we can observe approaches from the community and the industry to evolve this approach in to the so-called \enquote{InfoSec color wheel}~\cite{cremen2018infoseccolourwheel}.
In the proposed InfoSec color wheel, the author splits the six colors into primary and secondary colors, where the primary colors are teams on their own and the secondary colors are cooperation between two primary color teams.
The primary colors are represented by the red team, the blue team, and a newly introduced the yellow team, which represents the \enquote{builders} of software and systems.
The secondary colors are represented by the purple team (red and blue), the orange team (red and yellow), and the green team (yellow and blue).
Where \enquote{purple teaming} is actually an already established praxis as it evolved naturally from the cooperation between red and blue teams~\cite{}.

\todo{Add security methods and conceptional framworks such as MITRE attack, NIST, OWASP etc.}
\todo{Add Zero Trust Architecture, sorry Franky for removing it xD}

\subsection{Information security tools and software framworks}\label{subsec:information-security-tools-and-software-framworks}
One approach to reduce operational security costs is to adopt multiple modular frameworks that can be easily extended, configured, and executed continuously in a controlled manner~\cite{Zilberman2020SoKThreatEmulators}.
The threat emulation frameworks analyzed by Zilberman et al.\ evaluate multiple attack phases, including lateral movement, persistence, and attack execution.

\todo{Add security tools and software framworks such as MITRE Caledra, Atomic Red Team Metasploit, Infection Monkey, etc.}
\todo{Describe gaps and issues with the existing tools}

\subsection{Lower the bar with modularity and automation}\label{subsec:lower-the-bar-with-modularity-and-automation}


\subsection{\acrfull{NSAK}}\label{subsec:nsak-network-swiss-army-knife}

The Network Swiss Army Knife focuses on containerized, orchestrated scenarios that execute specific attack drills in a controlled environment.
Future extensions will focus on enriching the assessment layer by systematically capturing and evaluating defensive responses of multiple scenarios.

This proof of concept comprises the design and implementation of a modular, isolated open-source security framework that focuses on extensibility and the controlled execution of attack-based scenarios.

The objective of this work is to investigate whether such a framework can provide a flexible, extendable, and safe foundation for modular and automated security testing in a network environment. .

In the summary of their paper, Zilberman et al.\ are highlighting the necessity of the following design requirements~\cite{Zilberman2020SoKThreatEmulators}:
\begin{itemize}
    \item Cleanup and configurability are important in order to repeat and automate the execution of attack scenarios during security tool assessment and what-if analysis.
    \item An emulator should support cleanup after the completion of the attack scenario, like CALDERA, Atomic Red Team, and Infection Monkey do, rather than after each individual procedure.
    \item An API, currently provided by Atomic Red Team, CALDERA, and Metasploit, facilitates integration between the threat emulators and organizational security array, thus enabling periodic and systematic security assessment.
    \item It is important to provide a GUI and ready to execute multi-procedure attacks for novice operators as well as a CLI to support automation and advanced customization capabilities.
\end{itemize}

We reconsolidate the highlighted design requirements for the implementation of NSAK into the following list of features:
\begin{itemize}
    \item CLI, GUI, and API to manage resources and execute scenarios.
    \item Configurability of the framework and the resources.
    \item Automatic cleanup procedures after the completion of a scenario.
\end{itemize}

Even though we agree with the importance of the highlighted design requirements, we are not able to include them all in the PoC of the NSAK framework.
In the context of this project, our focus is to verify the feasibility of building a framework which provides modularity and automation and show which advantages it can provide over existing solutions.
But because we are planning to build upon this PoC, we try to incorporate the design requirements in the architecture and design of the framework.

\todo{Which gap are we closing, how we plan to distinguish NSAK from MITRE Caledra and other Tools, how can it be incoperated into existing security methods and conceptional frameworks}

\hrule

- Hypothese: Mit einem modularen framework kan das erstellen von automatisierten und reproduzierbaren Red und Blue Team Szenarien ermöglicht werden
- Experimente:
    - Das implementieren von Szenarien mithilfe von modularen und wiederverwendbaren drills funktioniert
    - Szenarien können in simulierten und realen Netzwerkumgebungen ausgeführt werden

2.2 Einleitung
Die Einleitung führt einerseits zum Thema hin (Ausgangslage), und informiert andererseits darüber, warum (Fragestellung/Problem) und
wozu (Ziel/Zweck) es die Arbeit gibt sowie ggf. wie sie zustande gekommen ist (methodisches Vorgehen). Die Einleitung kann je nach
Umfang und Thema mit oder ohne Unterkapitel verfasst werden. Sie umfasst ca. 5–10 % der Arbeit. Sie nimmt keine Ergebnisse vorweg.
Einen Überblick über die Kapitel der Arbeit gibt es höchstens bei sehr langen Arbeiten (beispielsweise Masterarbeit).
Die Ausgangslage beschreiben
– Relevanz: Warum ist dieses Thema überhaupt bedeutsam? Schon hier gilt es, nicht einfach etwas zu behaupten, sondern Fakten
und Aussagen mit Fachliteratur zu belegen.
– Aktualität: Gibt es einen aktuellen Bezug?
– Zusammenhang: Wie lässt sich das Thema einordnen? In welchem fachlichen Kontext steht die Arbeit?
– Forschungsstand: Was ist schon erforscht? Gibt es bereits Untersuchungen? (Ist-Zustand und Zusammenhang mit dem eigenen
Thema.) Je nach Art und Umfang der Arbeit gibt es zum Wissensstand ein separates Kapitel (siehe 2.3).
– Wenn vorhanden externe Auftraggeber, Auftraggeberinnen: Wer sind die beteiligten Partnerinnen oder Stakeholder?
– Den Kurs bzw. das Modul, in dem die Arbeit entsteht, erwähnt man auf dem Titelblatt (siehe 4.1).
8
Das Problem und das Ziel darstellen
– Zweck der Arbeit: Welche Aufgabe, Herausforderung, welches Problem soll gelöst werden? Warum sollte man die Arbeit lesen?
– Fragestellung: Auf welche konkrete Hauptfrage (evtl. mit konkretisierenden Unterfragen) soll im Schlusskapitel eine fundierte
Antwort gegeben werden? Ist die Fragestellung genügend eingegrenzt? Gibt es Hypothesen?
– Abgrenzung: Wo liegen die Grenzen der Untersuchung (zeitlich, geografisch, thematisch, methodisch, in der Auswahl der Hilfsmittel usw.)? Was wird nicht untersucht? Was kann die Arbeit nicht leisten?
– Ziel: Was soll die Untersuchung genau bewirken? Was ist die Absicht hinter der Arbeit?
– Erwartung: Was möchte die Arbeit leisten (Nutzen der Untersuchung, Soll-Zustand)? Für welche konkrete Zielgruppe sind die
Ergebnisse der Arbeit von Nutzen? Was ist zu erwarten? Was ist nicht zu erwarten?
Das methodische Vorgehen andeuten
Wie man methodisch vorgeht, wird ausführlich im Methodenkapitel beschrieben (siehe 2.4). Oft erwähnt man aber in der Einleitung bereits in ein bis zwei Sätzen, mit welcher Methode man arbeitet (Literaturarbeit, Umfrage, Entwickeln eines Prototyps,
Variantenstudium usw.).
