\section{Use-Cases}\label{sec:use-cases}

\begin{figure}[H]
    \centering
    \includegraphics[width=0.7\linewidth]{../assets/figures/diagrams/concept-architecture-diagrams/NSAK-use-case-diagram}
    \caption{}
    \label{fig:NSAK-use-case-diagram}
\end{figure}
Figure~\ref{fig:NSAK-use-case-diagram} illustrates the use case structure of the proposed NSAK modular framework.
The operator interacts with the NSAK primarily through two high-level commands:
Build Scenario (UC-01) and Run Scenario (UC-06).
During the Build Scenario use case (UC-01), the system builds a scenario container for execution.
In complex network infrastructures, additional configuration parameters—such as network interface mappings may be
required and need to be provided as build-time arguments (UC-02).

The system loads the selected scenario (UC-03).
At this stage, the scenario orchestrates the required drills necessary to perform the intended attack.

The build process concludes with the creation of an OCI-compliant container image (UC-05), which encapsulates
the fully configured scenario.

The Run Scenario use case (UC-06) represents an abstract execution phase.
In this phase, the previously built container image is runed, and the configured attack drills are been executed within
the containerized environment (UC-08).

Finally, the system performs a cleanup procedure in which all scenario-specific resources, processes and drills are
terminated.
This step minimizes side effects, reduces system noise, and prevents interference other scenarios that may
reuse the same drills.



\todo{drüber lesen und mit UC final ableichen}
\begin{longtable}{@{} >{\columncolor{blue!15}}l >{\raggedright\arraybackslash}p{0.75\textwidth} @{}}
    \caption{Use Cases Specification (NSAK)}\\
    \toprule
    \rowcolor{blue!30}\textbf{NR} \& \textbf{Details} \\
    \endfirsthead
    \toprule
    \rowcolor{blue!30}\textbf{NR} \& \textbf{Details} \\
    \endhead

    % --- UC-01 ----------------------------------------------------
    \textbf{UC-01} &
    \textbf{Use-Case:} Build Scenario \newline
    \textbf{Description:} Builds a scenario container based on a selected scenario configuration. \newline
    \textbf{Actor:} Operator \newline
    \textbf{Trigger:} Operator initiates scenario build via the command-line interface. \newline
    \textbf{Preconditions:} NSAK initialized; scenarios available. \newline
    \textbf{Main Scenario:} \newline
    1. Operator selects a scenario to build using the command-line interface. \newline
    2. System validates the selected scenario. \newline
    3. System executes the included use cases:
    \begin{itemize}
        \item Configure Scenario (UC-2)
        \item Load Scenario (UC-3)
        \item Load Drills (UC-4)
        \item Build OCI Container (UC-5)
    \end{itemize}
    \textbf{Alternative Scenarios:} No scenarios available → inform operator. \newline
    \textbf{Error Scenarios:} Conflicting scenario configuration detected → build aborted. \newline
    \textbf{Result:} Scenario container successfully built. \newline
    \textbf{Postconditions:} Scenario container stored and ready to run. \\
    \midrule

    % --- UC-02 ----------------------------------------------------

    \textbf{UC-02} &
    \textbf{Use-Case:} Configure Scenario \newline
    \textbf{Description:} Defines scenario-specific build parameters such as network interfaces
    and execution options. \newline
    \textbf{Actor:} System \newline
    \textbf{Trigger:} Scenario selected for build (UC-01). \newline
    \textbf{Preconditions:} Scenario selection available. \newline
    \textbf{Main Scenario:}
    1. System applies scenario-specific configuration parameters. \newline
    \textbf{Result:} Scenario configuration created. \newline
    \textbf{Postconditions:} Scenario configuration available for loading. \\
    \midrule

    % --- UC-03 ----------------------------------------------------

    \textbf{UC-03} &
    \textbf{Use-Case:} Load Scenario \newline
    \textbf{Description:} Loads and validate the selected scenarios \newline
    \textbf{Actor:} System \newline
    \textbf{Trigger:} Scenario configuration available (UC-02). \newline
    \textbf{Preconditions:} Scenario configuration created. \newline
    \textbf{Main Scenario:} \newline
    1. System retrieves the scenario definition files (scenario.yaml, scenario.py, README.md). \newline
    2. System validates the scenario structure and resolves declared dependencies. \newline
    \textbf{Error Scenarios:} Validation or dependency failure - preparation aborted with Error Log. \newline
    \textbf{Result:} Scenario is successfully loaded. \newline
    \textbf{Postconditions:} Scenario representation available for drill loading. \\
    \midrule

    % --- UC-04 ----------------------------------------------------
    \textbf{UC-04} &
    \textbf{Use-Case:} Load Drills \newline
    \textbf{Description:} Loads the attack drills required by the selected scenario. \newline
    \textbf{Actor:} System \newline
    \textbf{Trigger:} Scenario loaded (UC-03). \newline
    \textbf{Preconditions:} Scenario representation available. \newline
    \textbf{Main Scenario:} \newline
    1. System resolves drill references defined in the scenario configuration. \newline
    2. System instantiates drill objects and loads associated metadata. \newline
    \textbf{Error Scenarios:} Invalid drill definition, drill not found, or ambiguous drill reference. \newline
    \textbf{Result:} Required drill objects loaded. \newline
    \textbf{Postconditions:} Drills available for container build. \\
    \midrule

    % --- UC-05 --------------------------------------------------
    \textbf{UC-05} &
    \textbf{Use-Case:} Build OCI Container \newline
    \textbf{Description:} Builds an OCI-compliant container image for the loaded scenario. \newline
    \textbf{Actor:} System \newline
    \textbf{Trigger:} Scenario and drills loaded (UC-03, UC-04). \newline
    \textbf{Preconditions:} Scenario representation and drill objects available. \newline
    \textbf{Main Scenario:} \newline
    1. System generates the container build context. \newline
    2. System builds the scenario container image with required privileges and network configuration. \newline
    \textbf{Error Scenarios:} Container build failure - build aborted with error message. \newline
    \textbf{Result:} OCI-compliant scenario container image built. \newline
    \textbf{Postconditions:} Scenario container image stored and ready for execution. \\
    \midrule


    %--- UC-06 --------------------------------------------------
    \textbf{UC-06} &
    \textbf{Use-Case:} Run Scenario \newline
    \textbf{Description:} Executes a previously built scenario container. \newline
    Specific scenarios such as Rogue AP or ARP MITM represent specialized configurations of this use case.
    \textbf{Actor:} Operator \newline
    \textbf{Trigger:} Operator initiates scenario execution via the command-line interface. \newline
    \textbf{Preconditions:} Scenario container image available (UC-05). \newline
    \textbf{Main Scenario:} \newline
    1. System starts the scenario container with the required execution parameters. \newline
    2. System executes the included use cases:
    \begin{itemize}
        \item Execute Scenario (UC-07)
        \item Clean Up Scenario (UC-09)
    \end{itemize}
    \textbf{Result:} Scenario container execution started. \newline
    \textbf{Postconditions:} Scenario execution context active. \\
    \midrule


    %--- UC-07 --------------------------------------------------

    \textbf{UC-07} &
    \textbf{Use-Case:} Execute Scenario \newline
    \textbf{Description:} Orchestrates the execution of a previously built scenario container and coordinates
    the execution of the associated attack drills. \newline
    \textbf{Actor:} System \newline
    \textbf{Trigger:} Run Scenario (UC-6) \newline
    \textbf{Preconditions:} Scenario Image available and started \newline
    \textbf{Main Scenario:} \newline
    1. System Scenario Manager executes for the selected scenario \newline
    2. System Drill Manager execute drill UC-8 include use-case
    \textbf{Error Scenarios:} Scenario not found or scenario container not available. \newline
    \textbf{Result:} Scenario execution initiated and drill execution orchestrated. \newline
    \textbf{Postconditions:} Scenario container is running and drills are being executed.\\
    \midrule

    % --- UC-08 --------------------------------------------------
    \textbf{UC-08} &
    \textbf{Use-Case:} Execute Drills \newline
    \textbf{Description:} Executes the attack drills defined in the scenario configuration within the running scenario
    container. \newline
    \textbf{Actor:} System \newline
    \textbf{Preconditions:} Scenario execution context initialized. \newline
    \textbf{Main Scenario:} \newline
    1. System Drill Manager retrieves the list of configured drills. \newline
    2. System Drill Manager executes the drills according to the defined order and parameters. \newline
    \textbf{Error Scenarios:} Drill execution failure or missing drill definition. \newline
    \textbf{Result:} Configured attack drills executed. \newline
    \midrule

    % --- UC-09 --------------------------------------------------
    \textbf{UC-09} &
    \textbf{Use-Case:} Clean Up Scenario \newline
    \textbf{Description:} Terminates the running scenario container and restores the system to a defined baseline
    state. \newline
    \textbf{Actor:} System \newline
    \textbf{Trigger:} Stop Scenario (UC-06) \newline
    \textbf{Preconditions:} Scenario container is running. \newline
    \textbf{Main Scenario:} \newline
    1. System stops the running scenario container. \newline
    2. System invokes the included use case Clean Up Drills (UC-10). \newline
    \textbf{Error Scenarios:} Scenario container cannot be terminated. \newline
    \textbf{Result:} Scenario execution terminated. \newline
    \textbf{Postconditions:} Scenario container stopped and removed.\\
    \midrule
    % --- UC-10 --------------------------------------------------
    \textbf{UC-10} &
    \textbf{Use-Case:} Clean Up Drills \newline
    \textbf{Description:} Cleans up artifacts and state changes introduced by executed attack drills. \newline
    \textbf{Actor:} System \newline
    \textbf{Preconditions:} Drill execution completed or aborted. \newline
    \textbf{Main Scenario:} \newline
    1. System Drill Manager terminates active drill processes. \newline
    2. System Drill Manager removes temporary artifacts and resets modified parameters. \newline
    \textbf{Error Scenarios:} Incomplete cleanup due to failed drill termination. \newline
    \textbf{Result:} Drill-related artifacts removed and state reset. \newline
    \bottomrule


\end{longtable}
