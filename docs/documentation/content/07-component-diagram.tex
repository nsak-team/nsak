\section{Component-diagram}\label{sec:component-diagram}

\begin{figure}[H]
	\centering
	\includegraphics[width=1\linewidth]{../assets/figures/diagrams/concept-architecture-diagrams/project_2_nsak_component_diagram.drawio}
	\caption{}
	\label{fig:project2nsakcomponentdiagram}
\end{figure}

The component diagram~\textbf{\ref{fig:project2nsakcomponentdiagram}} illustrates how the different subsystems and their respective components should interface each other.
The \textbf{Network Stack} subsystem abstracts away the \textbf{hardware} layer and provides a \textbf{ManagementNetwork} and \textbf{TargetNetwork} component.
The components \textbf{WIFIConnection}, \textbf{EthernetConnection} and \textbf{MobileConnection} are physical interfaces in the \textbf{Hardware} subsystem.
Operators would connect to the \acrshort{nsak} device via \textbf{SSH} over the \textbf{management network} to open a remote shell.
Afterward, they use the \textbf{NSAK CLI}, which exposes \textbf{DrillCommands}, \textbf{ScenarioCommands} and \textbf{EnvironmentCommands}.
These commands interact with the API provided by the \textbf{NSAK core}, which is structured into a management component per resource type: \textbf{Drills}, \textbf{Scenarios} and \textbf{Environments}.
In the \textbf{NSAK core} subsystem we can see how the management components are responsible for their respective resources and interfacing with each other.
On the right side of we can see how the resources are interacting with the broader system context:
\begin{itemize}
	\item \textbf{Drills:} Ability to access the \textbf{TargetNetwork} via the \textbf{NetworkStack} subsystem.
	\item \textbf{Scenarios:} Responsible for interacting with the \textbf{Container Engine} and \textbf{SystemD} subsystems.
	\item \textbf{Environments:} Interfaces with other subsystems on the right side, but provides important information for the scenarios.
\end{itemize}

As scenarios are built as \acrshort{oci} images, we need an \textbf{OCI Container Engine} subsystem consisting of the \textbf{OCIImage} and \textbf{OCIContainer} components.
While planned and already displayed in the diagram, the \textbf{SystemD} subsystem is not yet integrated into the framework.
As most Linux distributions already ship with SystemD, it would make sense to implement the possibility to provide unit files to schedule and manage scenarios with the \textbf{Timer} and \textbf{Service} components respectively.
