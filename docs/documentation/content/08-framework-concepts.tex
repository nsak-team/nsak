\subsection{Framework Concepts}\label{subsec:framework-concepts}

This section describes the high-level concepts, resources and vocabulary needed to understand and work with the NSAK framework.

\todo{Optional: Add a diagram which shows the relation between the concepts}

Overview of the NSAK resources and concepts:
\begin{itemize}
    \item Devices
    \item Environments
    \item Drills
    \item Scenarios
    \item Operator
    \item Operation
\end{itemize}

\subsubsection{Devices}\label{subsubsec:devices}
Under a \textbf{device} we understand a physical or virtual machine, which is capable of running the NSAK framework.
Even though we currently only work and describe the hardware devices evaluated in~\ref{sec:hardware-selection}, other devices or virtual machines could be used with NSAK\@.

The following list vaguely describes the minimum requirements for a device:
\begin{itemize}
    \item Processor architecture: ARM and x86 should work equally well, as the NSAK framework is written in python and the scenarios are OCI images/containers, which are built on the NSAK device.
    \item Capable of running a Linux-based operating system, such as Debian.
    \item Enough memory and compute resources to run multiple OCI containers.
    \item Ideally, multiple physical network ports and Wi-Fi for covering many scenarios and environments.
    \item Optionally, additional bulk storage for data collection, such as PCAPs via T-Shark.
\end{itemize}

\textbf{Provisioning a NSAK device} usually consists of the following tasks:
\begin{enumerate}
    \item Install and configure a Linux-based operating system
    \item Set up a minimal network configuration and SSH access
    \item Install system dependencies required for NSAK
    \item Install and configure NSAK
\end{enumerate}

After a device is provisioned, we refer to it as a \textbf{NSAK device}, which may or may not be prepared for an operation.

\subsubsection{Environments}\label{subsubsec:environments}
An \textbf{environment} is representing a specific network topology including infrastructure components, servers, clients and services.
Ideally, an environment describes a part or a subset of a network and system infrastructure like you would encounter in a real organization.

Examples of environments:
\begin{itemize}
    \item WLAN AP: Smartphone, WLAN AccessPoint, Router
    \item Client - server: Client, Server, Switch
    \item Home network: Router, WLAN, One Physical Network (Star Topology), Multiple Devices (Computers, Laptops, SmartPhones, SmartTVs)
    \item Business network: Firewall, Router, DC Server, Intranet, Multiple Subnets, Multiple WLAN Access points, Switches
\end{itemize}

\subsubsection{Drills}\label{subsubsec:drills}
A \textbf{drill}, initially called a module, is a sequence of actions with a specific goal.
This goal can be an active or passive attack, network discovery, monitoring, analysis, data extraction, a hook for manual intervention or a device configuration.

Examples of drills:
\begin{itemize}
    \item Network sniffing with TShark with a specific filter (http traffic)
    \item Data extraction on an internal bulk storage or external network file system
    \item Active or passive MITM (man in the middle) attack with a transparent TCP proxy
    \item ARP Spoofing
    \item WLAN SSID spoofing
    \item Network discovery with nmap or arp-scan
    \item Network configuration, such as enabling IP-Forwarding or NAT
\end{itemize}

\subsubsection{Scenarios}\label{subsubsec:scenarios}
A \textbf{scenario} is designed for one or multiple environments, consists of a sequence of drills and describes a concrete use case for specific red or blue team activities.

Examples of Scenarios:
\begin{itemize}
    \item WLAN SSID Spoofing:
    \begin{itemize}
        \item Environment: WLAN AP
        \item Drills: Network configuration for DHCP, NAT, SSID Spoofing, Packet Sniffing
    \end{itemize}
    \item TCP MITM Attack:
    \begin{itemize}
        \item Environment: TCP client - server
        \item Drills: Automatic network discovery and configuration, ARP Spoofing, Transparent TCP Proxy, Packet manipulation
    \end{itemize}
\end{itemize}

\subsubsection{Operator}
For simplicity and consistency we use the term \textbf{operator} for the person or team, which is planning and executing operations with NSAK\@.
So an operator can refer to a single IT-specialist, a red, blue or purple team.

Examples of Operators:
\begin{itemize}
    \item A single IT-Specialized or Security researcher
    \item System and network engineering teams
    \item Red, blue, purple teams
\end{itemize}

\subsubsection{Operation}
An \textbf{operation} is the deployment of NSAK in a real network.

Preparing an operation usually has the following sequence of tasks, assessed and executed by an operator:
\begin{enumerate}
    \item Provision a NSAK device.
    \item Choose one or multiple environments which are relevant for the operation.
    \item Configure and build all or a subset of scenarios which can be executed in those environments.
    \item Scenarios
\end{enumerate}

An operation explicitly excludes the development phase for scenarios, drills and environments, as these resources should be finalized and tested before being used in a real operation, otherwise the following conventions should be used:
\begin{itemize}
    \item \textbf{Operation Simulation:} Simulating an operation in a virtualized environment.
    \item \textbf{Operation Test:} Testing an operation in a physical lab network.
\end{itemize}

\todo{Optional: Describe the relation between an operation and red, blue team phases (https://mindgard.ai/blog/red-team-operations-phases-of-engagement)}
