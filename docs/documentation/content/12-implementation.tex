\section{Implementation}\label{sec:implementation}

\subsection{MITM ARP-spoofing}\label{subsec:MITM ARP-spoofing}

\subsection{Rogue Access Point}\label{subsec: Rogue Access Point}

\subsubsection{Network Topology}{subsubsec: Network Topology}

\begin{figure}[H]
    \centering
    \includegraphics[width=0.7\linewidth]{../assets/figures/diagrams/concept-architecture-diagrams/rogue-ap}
    \caption{}
    \label{fig:rogue-ap}
\end{figure}

Figure\ref {fig:rogue-ap} illustrates the network topology used for the Rogue Access Point scenario.
The NSAK device is positioned between the wireless clients and an upstream host system acting as an internet gateway.
Two network interfaces are used on the NSAK device: a wireless interface (wlan0) operating in access point mode,
and a wired uplink interface (eth1) connected to the host system.

\subsubsection{Rogue Access Point Implementation}\label{subsubsec: RAP Implementation}
This section describes the technical realization of the Rogue Access Point scenario within the NSAK framework.
The focus lies on the integration of wireless access point functionality, traffic forwarding,
and packet capture.


Within the NSAK framework, the Rogue Access Point scenario is implemented as a composition of multiple drills.
Each drill encapsulates a single operational responsibility, allowing the scenario to orchestrate the drills separately,
ant to remain modular and extensible.

\subsubsection{Scenario}\label{subsubsec: Scenario structure and involved components}
The Rogue Access Point scenario consists of several drills that are executed sequentially to establish a functional
wireless access point capable of intercepting and forwarding network traffic.

\begin{itemize}
    \item Network interface preparation
    \item Wireless access point initialization
    \item Traffic forwarding and network address translation
    \item Packet capture and monitoring
\end{itemize}

\subsubsection{Drills}\label{subsubsec: RAP Drills}


\textbf{The Hostapd Drill} is responsible for configuring the wireless network interface of the NSAK device in access
mode.
This includes assigning network parameters to the interfaces and enabling beacon transmission to allow the client
devices to connect to the rogue access point


The access point functionality is implemented using standard Linux networking services running in an isolated subprocesses.
The controlled interaction with the operating system allows reliable startup and shutdown behavior.
Furthermore, the current process state can be tracked.


\textbf{Traffic Forwarding and Network Integration drills} are providing network connectivity for clients.
The NSAK device establishes an uplink connection to an external network interface.
Traffic forwarding is enabled between the wireless and uplink interfaces, enabling transparent internet access.


Network address translation and packet forwarding are configured dynamically during scenario execution.
This enables the NSAK device to operate as an intermediary between wireless clients and the upstream network.


In parallel, a \textbf{traffic capture drill} on the connected interface captures traffic passing through.
The pcap files can be used for later analysis, enabling the evaluation of client behavior and the network.
interactions.

By separating packet capture into an independent drill, the framework allows traffic monitoring to
be reused across different scenarios without modification.

The scenario manager orchestrates the execution of all drills involved in the Rogue Access Point scenario.
Drills are executed in a predefined order, ensuring that the required network services are available before
dependent components are started.


\textbf{Error handling and cleanup}\
To prevent persistent system modifications, each drill defines a cleanup routine that can be executed after scenario
completion or upon failure.
This ensures that network interfaces and system services are restored to their original state.
In the current state of the POC the cleanup functionality need to be adjusted for the broad diversity of the drills
and covers momentarily not all possible edge cases.

But as mentioned in, a centralized cleanup mechanism ensures that partial execution states do not persist in the system
in an inconsistent configuration.
And helps to prevent uncontrolled behaviors of drills.

\textbf{}
This section focused on the technical realization of the Rogue Access Point scenario.
The effectiveness and behavior of the scenario under real network conditions are evaluated in the subsequent evaluation
chapter.
