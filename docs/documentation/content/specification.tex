\section{Network Environments}\label{sec:environments}

Each environment represents a practical setup in which the Network Swiss Army Knife (nsak) can be deployed
for traffic analysis, performance testing, or security evaluation.

\subsection{Category I: Basic / Local}\label{subsec:category-i:-basic-/-local}

\subsubsection{E 1: Point-to-Point}
\textbf{Diagram:} Laptop $\leftrightarrow$ nsak $\leftrightarrow$ Router  \\
\begin{figure}[H]
	\centering
	\includegraphics[width=0.7\linewidth]{../assets/figures/diagrams/network-diagrams/LAN-Sniffing}
	\caption{}
	\label{fig:lan-sniffing}
\end{figure}

\textbf{Description:}  
Direct inline bridge between client and router.
Used for basic LAN capturing, latency, and throughput testing.

\subsubsection{E 2: Home Network}
\textbf{Diagram:} Laptop, Smart Devices, Printer $\leftrightarrow$ nsak (inline) $\leftrightarrow$ Router  \\
\begin{figure}[H]
	\centering
	\includegraphics[width=0.5\linewidth]{../assets/figures/diagrams/network-diagrams/LAN-4-ports-Sniffing.drawio}
	\caption{}
	\label{fig:lan-4-ports-sniffing}
\end{figure}

\textbf{Description:}  
Captures typical home traffic.
The WLAN interface is required for Wi-Fi analysis.
Useful for IoT discovery and local broadcast observation.

\subsubsection{E 3: Business Network}
\textbf{Diagram:} Devices $\leftrightarrow$ Switch $\leftrightarrow$ nsak (inline) $\leftrightarrow$ Server / Router  \\

\begin{figure} [H]
	\centering
	\includegraphics[width=0.6\linewidth]{../assets/figures/diagrams/network-diagrams/LAN-Sniffing-Switch-AccessPoint}
	\caption{}
	\label{fig:lan-sniffing-switch-accesspoint}
\end{figure}

\textbf{Description:}  
Represents a small office LAN. nsak placed at uplink or server edge to monitor internal traffic, VLANs, and broadcast domains.

\subsection{Category II: Mobile / Wireless}\label{subsec:category-ii:-mobile-/-wireless}

\subsubsection{E 4: Access Point Mode}
\textbf{Diagram:} Wi-Fi Device $\leftrightarrow$ AP-nsak $\leftrightarrow$ Internet  \\
\begin{figure}[H]
	\centering
	\includegraphics[width=0.5\linewidth]{../assets/figures/diagrams/network-diagrams/LAN-WLAN-Sniffing.drawio}
	\caption{}
	\label{fig:lan-wlan-sniffing}
\end{figure}


\textbf{Description:}  
nsak acts as Wi-Fi AP, providing connectivity and packet capture.
Captures management and data frames for WLAN analysis.

\subsubsection{E 5: Mobile Hotspot}
\textbf{Diagram:} Smartphone $\leftrightarrow$ nsak $\leftrightarrow$ LTE/5G  \\
\textbf{Description:}  
Mobile tethering or hotspot scenario.
Focus on NAT behavior, encryption overhead, and power constraints.

\subsection{Category III: Secure / Advanced}\label{subsec:category-iii:-secure-/-advanced}

\subsubsection{E 6: Data Center / Server Rack}
\textbf{Diagram:} Servers $\leftrightarrow$ Switch $\leftrightarrow$ Firewall $\leftrightarrow$ nsak   \\
\textbf{Description:}  
High-performance setup for 2.5G\textendash10G traffic capture.
Focus on throughput, buffering, and VLAN-tagged traffic.

\subsubsection{E 7: VPN Gateway}
\textbf{Diagram:} Router $\leftrightarrow$ nsak (VPN Endpoint) $\leftrightarrow$ Remote Peer  \\
\textbf{Description:}  
nsak configured as WireGuard gateway.
Measures encrypted vs.
unencrypted traffic, tunnel stability, and CPU load.

\subsection{Category IV: IoT / Special Purpose}\label{subsec:category-iv:-iot-/-special-purpose}

\subsubsection{E 8: VLAN-Segmented Enterprise Network}
\textbf{Diagram:} Switch + Router + Multiple VLANs $\leftrightarrow$ nsak  \\
\textbf{Description:}  
Used to verify VLAN isolation and detect inter-segment leaks or misconfigurations.

\subsubsection{E 9: IoT Sensor Network}
\textbf{Diagram:} IoT Devices $\leftrightarrow$ Local Gateway (Broadcast) $\leftrightarrow$ nsak   \\
\textbf{Description:}  
Passive capture of local IoT or broadcast-based communication.
Identifies timing, protocol use, and unsecured traffic.

\subsection{Category V: Virtual / Simulation}\label{subsec:category-v:-virtual-/-simulation}

\subsubsection{E 10: Attack Simulation Network}
\textbf{Diagram:} Virtual Attacker $\leftrightarrow$ Target VM $\leftrightarrow$ nsak  
\textbf{Description:}  
Virtual lab for testing detection systems, malware traffic, and replay scenarios.

\subsubsection{E 11: Remote Management Environment}
\textbf{Diagram:} Admin $\leftrightarrow$ VPN/SSH $\leftrightarrow$ nsak (headless)  
\textbf{Description:}  
Remote-controlled nsak for automated capture and monitoring in unattended operation.

\subsubsection{E 12: Dual-Sniffer Setup}
\textbf{Diagram:} nsak-A $\parallel$ nsak-B (same link)  
\textbf{Description:}  
Two synchronized devices capture the same traffic path.
Used for timestamp comparison and hardware validation.

\section{Software Modules}\label{sec:modules}

We define a software module as a sequence of actions with a predefined goal.
Actions can be but are not limited to the execution of scripts, cli tools or steps in a program, but also human interaction.
Such an action may appear in multiple software modules, with different configurations and in interplay with other actions.

We generally differentiate between two software module types:
\begin{itemize}
	\item Active Software Module: Does actively intervene with other devices or alter data.
	Can be an active attack, but also testing for behavior of another device, depending on the scenario.
	\item Passive Software Module: Does not actively intervene in communication and does not alter data.
	Can be a passive attack, but also monitoring or analysis, depending on the scenario.
\end{itemize}

To group similar software modules together, we defined the following categories:
\begin{enumerate}
	\item Network Traffic Collection
	\item Network Mapping and Port Mapping
	\item Man\-in\-the\-Middle (MITM)
	\item Exploits
	\item Brute Force
	\item Denial of Service (DoS)
	\item Physical (TBD?)
\end{enumerate}

\subsection{Category 1: Network Traffic Collection}\label{subsec:module_1_network_traffic_collection}
In this category are software modules to collect network traffic with the help of network sniffers like WireShark/TShark or tcpdump.

\subsubsection{Module 1.1: Live Network Traffic Monitoring}\label{subsubsec:module_1_1_general_network_traffic_collection}
The idea of this module is to allow an operator to use WireShark and connect a remote interface to it,
which will then show all packets sent and received on nsack.
The operator can then add filters and analyze the traffic in real time.
This module requires a live connection from a remote device to nsak,
in a real world attack scenario the attacker must be nearby for WI-FI access or could be identified via the cellular connection.

\begin{itemize}
	\item Type: Passive
	\item Goal: Monitor live network traffic
\end{itemize}

Action sequence:
\begin{enumerate}
	\item nsak: Enable remote access via Wi-Fi or cellular connection.
	\item nsak: Enable layer 2 bridge between interfaces.
	\item nsak: Start tshark, listening on all interfaces, no filters.
	\item Remote Device: Start WireShark and connect to a remote interface.
	\item Remote Device: Analyze live network traffic, apply filters
\end{enumerate}

\subsubsection{Module 1.2: Network Traffic Collection (Local)}\label{subsubsec:module_1_2_general_network_traffic_collection}
This module requires physical access to nsak to extract the collected network traffic,
which might be challenging or dangerous in a real-world attack scenario.
The amount of data which can be extracted this way depends heavily on the amount of network traffic,
how specific the filters are set, and the disk space available.

\begin{itemize}
	\item Type: Passive
	\item Goal: Collect network traffic
\end{itemize}

Action sequence:
\begin{enumerate}
	\item nsak: Enable layer 2 bridge between interfaces.
	\item nsak: Start tshark, listening on all interfaces, set filters, write to a local file.
\end{enumerate}

\subsubsection{Module 1.3: Network Traffic Collection (Remote)}\label{subsubsec:module_1_3_http_network_traffic_collection}
This module requires a live connection from a remote device to nsak,
in a real world attack scenario the attacker must be nearby for WI-FI access or could be identified via the cellular connection.

\begin{itemize}
	\item Type: Passive
	\item Goal: Collect http network traffic
\end{itemize}

Action sequence:
\begin{enumerate}
	\item nsak: Enable remote access via Wi-Fi or cellular connection.
	\item nsak: Enable layer 2 bridge between interfaces.
	\item nsak: Start tshark, listening on all interfaces, set filters, write to a remote file.
\end{enumerate}

\subsection{Category 2: Network Mapping and Port Mapping}\label{subsec:category-2:-network-mapping}
This category is concerned about mapping out the environment nsak is currently located in.

\subsubsection{Module 2.1: Network Discovery}\label{subsubsec:-layer-2-discovery}

\begin{itemize}
	\item Type: Active
	\item Goal: Discover Network Participants
\end{itemize}

Action sequence:
\begin{enumerate}
	\item nsak: Run arp-scan, on interface in a specific subnet
\end{enumerate}

\subsubsection{Module 2.2: Host Service/Version Detection}\label{subsubsec:-host-service/version-detection}

\begin{itemize}
	\item Type: Active
	\item Goal: Detect which OS and/or services are participating in the network
\end{itemize}

Action sequence:
\begin{enumerate}
	\item nsak: Run nmap, against a whole subnet or a specific host
\end{enumerate}

\subsection{Category 3: Man\-in\-the\-Middle (MITM)}\label{subsec:category-3:-man-in-the-middle-(mitm)}
Modules in this category will try to convince other devices or services in the network that nsak is a legitimate participant.

\subsubsection{Module 3.1 SSL/TLS MITM}\label{subsubsec:module-3.1-ssl/tls-mitm}
This module tries to intercept SSL/TLS handshakes and convince Alice and Bob that nsak is the respective counterpart.

\begin{itemize}
	\item Type: Active
	\item Goal: Intercept encrypted traffic
\end{itemize}

Action sequence:
\begin{enumerate}
	\item nsak: Execute SSL/TLS MITM attack (TBD)
\end{enumerate}

\subsubsection{Module 3.2 IP Spoofing}\label{subsec:module-3.2-ip-spoofing}

\begin{itemize}
	\item Type: Active
	\item Goal: Traffic interception
\end{itemize}

Action sequence:
\begin{enumerate}
	\item nsak: Set nsaks IP to the one of another device
\end{enumerate}

\subsubsection{Module 3.3: WLAN SSID Spoofing}\label{subsubsec:-wlan-ssid-spoofing}

\begin{itemize}
	\item Type: Active
	\item Goal: Traffic interception
\end{itemize}

Action sequence:
\begin{enumerate}
	\item nsak: Set nsaks WLAN SSID to the one of an actual WI-FI access point.
\end{enumerate}

\subsection{Category 4: Exploits}\label{subsec:category-4:-exploits}
This category contains modules which execute known exploits.

\subsubsection{Module 4.1: Join Network via WPS}\label{subsubsec:-join-network-via-wps}

\begin{itemize}
	\item Type: Active
	\item Goal: Wireless Network Access (intrusion)
\end{itemize}

Action sequence:
\begin{enumerate}
	\item nsak: TBD
\end{enumerate}

\subsection{Category 5: Brute force}\label{subsec:category-5:-brute-force}
With this module we try to gain access to a device or network with brute force.

\subsubsection{Module 5.1: SSH Login}\label{subsubsec:-ssh-dictonary-attack}

\begin{itemize}
	\item Type: Active
	\item Goal: Server Remote Access
\end{itemize}

Action sequence:
\begin{enumerate}
	\item nsak: Load dictionary for the current locale or better data set if available
	\item nsak: Brute Force SSH authentication
\end{enumerate}

\subsubsection{Module 5.2: WLAN Login}\label{subsubsec:-wlan-login}

\begin{itemize}
	\item Type: Active
	\item Goal: Wireless Network Access (intrusion)
\end{itemize}

Action sequence:
\begin{enumerate}
	\item nsak: Load dictionary for the current locale or better data set if available
	\item nsak: Brute Force WI-FI authentication
\end{enumerate}

\subsection{Category 6: Denial of Service (DoS)}\label{subsec:category-6:-denial-of-service-(dos)}

\subsubsection{Module 6.1: Classic Denial of Service}\label{subsubsec:-classic-denial-of-service}

\begin{itemize}
	\item Type: Active
	\item Goal: DoS
\end{itemize}

Action sequence:
\begin{enumerate}
	\item nsak: Send as many requests as possible until the service is not available anymore
\end{enumerate}

\section{Scenarios}\label{sec:scenarios}

A scenario is described as a combination of an environment, with a defined location of the sniffer and one or multiple software modules.


