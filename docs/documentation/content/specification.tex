\chapter{Network Environments}\label{sec:environment}


Each environment represents a practical setup in which the Network Swiss Army Knife (nsak) can be deployed for traffic analysis, performance testing, or security evaluation.

\subsection{Category I: Basic / Local}

\subsubsection{E 1: Point-to-Point}
\textbf{Diagram:} Laptop $\leftrightarrow$ nsak $\leftrightarrow$ Router  \\
\begin{figure}[H]
	\centering
	\includegraphics[width=0.7\linewidth]{../assets/LAN-Sniffing}
	\caption{}
	\label{fig:lan-sniffing}
\end{figure}

\textbf{Description:}  
Direct inline bridge between client and router. Used for basic LAN capturing, latency and throughput testing.

\subsubsection{E 2: Home Network}
\textbf{Diagram:} Laptop, Smart Devices, Printer $\leftrightarrow$ nsak (inline) $\leftrightarrow$ Router  \\
\begin{figure}[H]
	\centering
	\includegraphics[width=0.5\linewidth]{../assets/LAN-4-ports-Sniffing.drawio}
	\caption{}
	\label{fig:lan-4-ports-sniffing}
\end{figure}

\textbf{Description:}  
Captures typical home traffic. WLAN interface required for Wi-Fi analysis. Useful for IoT discovery and local broadcast observation.

\subsubsection{E 3: Business Network}
\textbf{Diagram:} Devices $\leftrightarrow$ Switch $\leftrightarrow$ nsak (inline) $\leftrightarrow$ Server / Router  \\

\begin{figure} [H]
	\centering
	\includegraphics[width=0.6\linewidth]{../assets/LAN-Sniffing-Switch-AccessPoint}
	\caption{}
	\label{fig:lan-sniffing-switch-accesspoint}
\end{figure}

\textbf{Description:}  
Represents a small office LAN. nsak placed at uplink or server edge to monitor internal traffic, VLANs, and broadcast domains.

\subsection{Category II: Mobile / Wireless}

\subsubsection{E 4: Access Point Mode}
\textbf{Diagram:} Wi-Fi Device $\leftrightarrow$ AP-nsak $\leftrightarrow$ Internet  \\
\begin{figure}[H]
	\centering
	\includegraphics[width=0.5\linewidth]{../assets/LAN-WLAN-Sniffing.drawio}
	\caption{}
	\label{fig:lan-wlan-sniffing}
\end{figure}


\textbf{Description:}  
nsak acts as Wi-Fi AP, providing connectivity and packet capture. Captures management and data frames for WLAN analysis.

\subsubsection{E 5: Mobile Hotspot}
\textbf{Diagram:} Smartphone $\leftrightarrow$ nsak $\leftrightarrow$ LTE/5G  \\
\textbf{Description:}  
Mobile tethering or hotspot scenario. Focus on NAT behavior, encryption overhead, and power constraints.

\subsection{Category III: Secure / Advanced}

\subsubsection{E 6: Data Center / Server Rack}
\textbf{Diagram:} Servers $\leftrightarrow$ Switch $\leftrightarrow$ Firewall $\leftrightarrow$ nsak   \\
\textbf{Description:}  
High-performance setup for 2.5G–10G traffic capture. Focus on throughput, buffering, and VLAN-tagged traffic.

\subsubsection{E 7: VPN Gateway}
\textbf{Diagram:} Router $\leftrightarrow$ nsak (VPN Endpoint) $\leftrightarrow$ Remote Peer  \\
\textbf{Description:}  
nsak configured as WireGuard gateway. Measures encrypted vs. unencrypted traffic, tunnel stability, and CPU load.

\subsection{Category IV: IoT / Special Purpose}

\subsubsection{E 8: VLAN-Segmented Enterprise Network}
\textbf{Diagram:} Switch + Router + Multiple VLANs $\leftrightarrow$ nsak  \\
\textbf{Description:}  
Used to verify VLAN isolation and detect inter-segment leaks or misconfigurations.

\subsubsection{E 9: IoT Sensor Network}
\textbf{Diagram:} IoT Devices $\leftrightarrow$ Local Gateway (Broadcast) $\leftrightarrow$ nsak   \\
\textbf{Description:}  
Passive capture of local IoT or broadcast-based communication. Identifies timing, protocol use, and unsecured traffic.

\subsection{Category V: Virtual / Simulation}

\subsubsection{E 10: Attack Simulation Network}
\textbf{Diagram:} Virtual Attacker $\leftrightarrow$ Target VM $\leftrightarrow$ nsak  
\textbf{Description:}  
Virtual lab for testing detection systems, malware traffic, and replay scenarios.

\subsubsection{E 11: Remote Management Environment}
\textbf{Diagram:} Admin $\leftrightarrow$ VPN/SSH $\leftrightarrow$ nsak (headless)  
\textbf{Description:}  
Remote-controlled nsak for automated capture and monitoring in unattended operation.

\subsubsection{E 12: Dual-Sniffer Setup}
\textbf{Diagram:} nsak-A $\parallel$ nsak-B (same link)  
\textbf{Description:}  
Two synchronized devices capture the same traffic path. Used for timestamp comparison and hardware validation.
