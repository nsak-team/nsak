\section{Discussion}\label{sec: Discussion}


This section discusses the results of the evaluation and the research questions defined in Section~\ref{subsec: Research Questions}


\textbf{Modularity of Drills (RQ1)}~\ref{rq1}


The evaluation demonstrates that network-composed scenarios can be built from different combinations of drills without
requiring modifications to the core framework.
All tested drill combinations executed successfully.
Disabling individual or multiple drills did not cause unintended side effects and unwanted system behaviors.
This indicates the loose coupling between drills.


However, the observed modularity is currently limited to the tested scenarios.
More complex scenarios that target a more demanding network may introduce unintended dependencies that lead to unwanted
behaviors, which should be investigated more thoroughly in future work.


\textbf{Reusability Across Scenarios (RQ2)}~\ref{rq2}


The reuse of different drills in multiple scenarios suggests that drills can be reused with minimal configuration effort.
Scenario-specific behavior was controlled exclusively through environment variables and build arguments.
Drill implementation remained unchanged.
This demonstrates the assumption that the \acrshort{nsak} framework is flexible and be orchestrated from a scenario perspective.
The minimal configuration helps reduce complexity for network security teams.
The flexibility between the inline and Wi-Fi attack scenarios demonstrates the spectrum and potential of network attacks.


Reusability was evaluated across a limited number of scenarios and drills.
Further evaluation across a wider range of scenarios is required to validate in-depth.


\textbf{Containerization and Isolation (RQ3)}~\ref{rq3}


The isolated scenario container can be pre-built and run independently with an individual drill set.
The container execution in the test scenario was triggered solely from the command line,
but could also be called similarly from a system process.
This enables the \acrshort{nsak} device to remain latent in a network until the attack is most likely to succeed or to have the greatest impact.


At the same time, the evaluation revealed the necessity to manipulate the host-network configuration, such as
Iptables rules and temporary files were not fully cleaned up after scenario execution.
This highlights the importance of an improved cleanup mechanism to ensure isolation and reproducibility in repeated execution.
The running containers are constantly running, increasing the likelihood of detection.
In a more sophisticated approach, the \acrshort{nsak} device should be as stealthy as possible to remain hidden from blue teams.


\textbf{Reproducibility or Scenarios (RQ4)}~\ref{rq4}
Repeated execution of identical scenarios resulted in consistent system behavior.
This suggests that the current framework and implementation provide sufficient security reproducibility.
experiments.


However, the reproducibility was assessed over a limited number of runs and devices.
Long-term testing would be required to evaluate stability.
