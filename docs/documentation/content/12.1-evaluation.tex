\section{Evaluation}\label{sec: Evaluation}

\subsection{Generel Functionality of a Rogue Access Point}\label{subsec: General Functionality}
The Rogue Access Point scenario was executed with different combinations of drills to test the modularity and the
independent drill execution.
The expected outcome is the creation of an open Wi-fi access point broadcasting the configured SSID,
providing DHCP leases and forwarding traffic to the uplink interface.
Additionally, traffic will be captured in PCAP format.


During execution, the access point was successfully created, and client devices could connect using the assigned IP address.
via DHCP.
The Network traffic was captured in a PCAP file.

\subsection{Research Questions}\label{subsec: Research Questions}
\textbf{RQ1 – Modularity}\label{RQ1 - Modularity}
RQ1: Can network security scenarios be composed from independent drills without modifying the core framework?


To ensure modularity, every drill should be able to work independently.
When setting the DISABLE-DRILL environment variable, remaining drills should execute independently without failure.
For example, if everything is disabled besides the ap-mode drill, the network should still be discoverable from clients.
In T4\ref{tab:modular-execution}, IP forwarding, uplink, and capture should not be executed and therefore are not working.


During the process, each drill has a specific attack goal or functionality.
No implicit dependencies between drills were observed during execution.


\begin{table}[H]
    \centering
    \begin{tabular}{lll}
        \toprule
        \textbf{Test}  & \textbf{description} & \textbf{expectation} \\
        \midrule
        T1 & Execute Rogue AP scenario with full drill set  & Scenario executes successfully \\
        T2 & Deactivate packet capture drill  & AP still functions \\
        T3 & Deactivate dnsmasq drill & All other drills are working, but in this case.
        t-shark can't capture traffic \\
        T4 & Just the ap-mode drill is enabled  & Client can see the SSID and can connect  \\
        \bottomrule
    \end{tabular}
    \caption{}
    \label{tab:modular-execution}
\end{table}

\textbf{RQ2 – Reusability}\label{RQ2 - Reusability}
RQ2: Can individual drills be reused across different scenarios with minimal configuration effort?


The hostapd drill was reused in two scenarios: a standalone access point and a rogue access point.
access point scenario.
In both cases, the drill implementation remained unchanged.
Scenario-specific behavior was controlled through environment variables such as SSID, channel, and country code.


These results confirm that drills are scenario-orchestrated and can be composed flexibly with minimal modification.

\textbf{RQ3 – Containerization}\label{RQ3 - Containerization}
RQ3: Does container-based execution provide sufficient isolation while maintaining required network functionality?


The evaluation focuses on whether container-based execution restricts or enables low-level network operations.
required for nsak-device functionality.


The Rogue Access Point scenario was executed inside a containerized NSAK environment on an embedded Linux device.
The container was granted only the necessary capabilities for network configuration and packet capture.


During scenario execution, all network services were started in privileged mode within the container environment.
During container-based execution, network interfaces, routing rules, and packet capture were successfully initialized.
The drills ap-mode, network setup, dnsmasq, thsark left certain config files and rules in iptables and filepaths.

\textbf{RQ4 – Reproducibility}\label{RQ4 - Reproducibility}
RQ4: Can scenarios be executed repeatedly without changing the system behavior?


With the same configuration, the scenario execution was triggered multiple times.
Devices could establish a connection with the access point.
Network connectivity was enabled in all runs. And the nsak-device was able to capture the traffic


No deviations in system behavior were observed across repeated executions.


Die Ergebnisse sind das, was man gerechnet, beobachtet, gemessen, entworfen, gelesen usw. hat. Sämtliche im Ergebniskapitel
dargestellten Informationen tragen zum Beantworten der Fragestellung bei. Das Grundlagenwissen zum Thema dagegen gehört nicht
in den Ergebnisteil. In naturwissenschaftlich-technischen Arbeiten werden das Präsentieren der Ergebnisse und deren Interpretation
(Diskussion) strikt voneinander getrennt. Die Ergebnisse werden objektiv und sachlich wiedergegeben (ohne Bewertung). Der Ergebnisteil enthält keine Deutung der Ergebnisse und keine persönliche Meinung. Bei Arbeiten in anderen Disziplinen wird diese Trennung im
Hauptteil der Arbeit weniger stark vorgenommen.
Hinweise
– Der Ergebnisteil ist in der Regel der längste Teil der Arbeit und muss darum einen besonders klaren und nachvollziehbaren
Aufbau und eine schlüssige, lückenlose Argumentationslinie aufweisen.
– Die Gliederung und die Terminologie des Ergebnisteils folgen aus der Fragestellung. Wird beispielsweise nach Varianten gefragt,
sind die einzelnen Varianten Unterkapitel des Ergebnisteils (ähnliches gilt für Massnahmen, Einflussfaktoren, Analysen usw.).
– Werden Begriffe aus der Fragestellung für die Gliederung bzw. Überschriften wiederverwendet, wird der rote Faden durch die
Arbeit klarer.
– Eine gute Abbildung oder Tabelle ist oft hilfreicher als lange Erklärungen im Text. Der Text verweist aber auf die Abbildungen
sowie Tabellen und gibt deren Kernaussagen wieder (siehe 3.2).
– Eine wissenschaftliche Arbeit darf Widersprüche nicht unterdrücken, sondern nennt sie explizit und diskutiert sie im Diskussionsteil.
– Wichtige Ergebnisse sollen Raum bekommen. Ergänzende Details und umfangreiche Daten dagegen gehören in den Anhang der
Arbeit.
