In this chapter we will describe what and how we implemented the \acrshort{nsak} framework, the first two scenarios, their drills and environments.
To efficiently describe the implementation, we reference paths and files relative to the repositories root directory~\cite{nsakRepository2026}.

\section{\acrshort{nsak} framework}\label{sec:nsak_framework}

In this section we describe which technologies we choose to implement the \acrshort{nsak} framework and the resource library.

As this project is a so called monorepo, the logical primary driver for the technology stack and dependencies is the core component.
The following system components such as the resource library and the CLI will inherit the technology stack, the python and system dependencies.
For this reason, the components following the core will only describe what they introduce additionally.

\textbf{Core}\\

\todo{Describe why we choose this technology, maybe we find references which underline the ease of use and advantages for modularity which are comming with python}

\begin{itemize}
    \item Programming language: Python
    \item Dependency manager: uv
    \item Virtual environment manager: uv
    \item Package build tool: uv
    \item Linter: ruff
    \item Formatter: ruff
    \item Type checker: mypy
    \item Testing framework: pytest
\end{itemize}

Python dependencies:
\begin{itemize}
    \item pyyaml: Library for loading, validating and reading yaml files
    \item scapy: Library for red team operations
    \item pre-commit: Package to enforce code quality tools for each commit
\end{itemize}

\textbf{System Dependencies}\\
As we leverage the abstraction of \acrshort{oci} containers to run scenarios in an encapsulated environment, we have only a minimal set of system dependencies.
All system dependencies that are required for running a drill or a scenario are installed into the scenario image, during the build process.

\begin{itemize}
    \item Version control: git
    \item Network tooling: iptables (we should switch to nf\_tables)
    \item \acrshort{oci} container manager: podman
    \item \acrshort{oci} container orchestrator: podman-compose
    \item Programming languages: python3, python3-pip, uv
    \item Utilities: curl, sudo
\end{itemize}

\textbf{Resource library}\\
\\

\textbf{CLI}\\

Python dependencies:
\begin{itemize}
    \item click: Library for building CLIs
\end{itemize}

\section{\acrshort{mitm} (ARP-spoofing / transparent TCP Proxy)}\label{sec:mitm_arp_spoofing_transparent_tcp_proxy}

\textbf{Network Topology / Environment}

\begin{figure}[H]
    \centering
    \includegraphics[width=0.7\linewidth]{../assets/figures/diagrams/concept-architecture-diagrams/mitm_arp_spoofing_transparent_tcp_proxy}
    \caption{}
    \label{fig:mitm-arp-spoofing-transparent-tcp-proxy}
\end{figure}

Figure~\ref{fig:mitm-arp-spoofing-transparent-tcp-proxy} illustrates the network topology used for the \acrfull{mitm} scenario.
It consists of two dotted circles separating the network boundaries of two networks.
The left circle is describing a simple \acrshort{tcp} client - server environment consisting of Alice (client), Bob (server), a network switch, and the \acrshort{nsak} device which was placed in the target network.
The right circle is sketching the management network, which is used by a red team operator to remotely execute the \acrshort{mitm} scenario on the \acrshort{nsak} device.

\textbf{\acrshort{mitm} scenario implementation}\\
This section describes the technical realization of the \acrshort{mitm} scenario within the \acrshort{nsak} framework.

\todo{Lücku: Complete this section}

\textbf{Simple TCP client - server environment}\\
For development and testing purposes the simple \acrshort{tcp} client - server environment was added as a resource to the \acrshort{nsak} repository under~\texttt{lib/environments/simple\_tcp\_client\_server}.
The resource includes a virtualized lab setup with podman/docker compose file (\texttt{lib/environments/simple\_tcp\_client\_server/compose.yaml}) and provides an extensive readme file (\texttt{lib/environments/simple\_tcp\_client\_server/README.md}), describing the setup of this environment as a physical lab environment.
This physical lab setup builds the basis for the experiment, verifying that the \acrshort{mitm} scenario works in a real world environment.
The compose file provides a possible representation of the target network with containers acting as Alice and Bob.
The containers entrypoints are two Python scripts, which are implementing the behavior as a \acrshort{tcp} client for Alice and as a \acrshort{tcp} server for Bob.
Because the default network driver used by podman and docker compose is abstracting away the data link of the OSI model,
the configuration of the macvlan network driver is required to simulate the networks behavior on layer 2 correctly.
The \acrshort{mitm} scenario resource under~\texttt{lib/scenarios/mitm/} contains a subfolder containing another readme file (\texttt{lib/scenarios/mitm/environments/simple\_tcp\_client\_server/README.md}), describing its integration in this physical lab environment.
This subfolder also provides a podman/docker compose file (\texttt{lib/scenarios/mitm/environments/simple\_tcp\_client\_server/compose.yaml}) which extends the compose file of this resource, providing a complete setup for running the scenario in a simulation.
The \acrshort{nsak} framework was then extended with the ability to simulate a scenario in a compatible environment which provides a podman/docker compose implementation.


\section{Rogue Access Point}\label{sec: Rogue Access Point}

\textbf{Network Topology}

\begin{figure}[H]
    \centering
    \includegraphics[width=0.7\linewidth]{../assets/figures/diagrams/concept-architecture-diagrams/rogue-ap}
    \caption{}
    \label{fig:rogue-ap}
\end{figure}

Figure~\ref{fig:rogue-ap} illustrates the network topology used for the Rogue Access Point scenario.
The NSAK device is positioned between the wireless clients and an upstream host system acting as an internet gateway.
Two network interfaces are used on the NSAK device: a wireless interface (wlan0) operating in access point mode,
and a wired uplink interface (eth1) connected to the host system.

\textbf{Rogue Access Point Implementation}
This section describes the technical realization of the Rogue Access Point scenario within the NSAK framework.
The focus lies on the integration of wireless access point functionality, traffic forwarding,
and packet capture.


Within the NSAK framework, the Rogue Access Point scenario is implemented as a composition of multiple drills.
Each drill encapsulates a single operational responsibility, allowing the scenario to orchestrate the drills separately,
ant to remain modular and extensible.

\textbf{Scenario}
The Rogue Access Point scenario consists of several drills that are executed sequentially to establish a functional
wireless access point capable of intercepting and forwarding network traffic.

\begin{itemize}
    \item Network interface preparation
    \item Wireless access point initialization
    \item Traffic forwarding and network address translation
    \item Packet capture and monitoring
\end{itemize}

\textbf{Drills}


\textbf{The Hostapd Drill} is responsible for configuring the wireless network interface of the NSAK device in access
mode.
This includes assigning network parameters to the interfaces and enabling beacon transmission to allow the client
devices to connect to the rogue access point


The access point functionality is implemented using standard Linux networking services running in an isolated subprocesses.
The controlled interaction with the operating system allows reliable startup and shutdown behavior.
Furthermore, the current process state can be tracked.


\textbf{Traffic Forwarding and Network Integration drills} are providing network connectivity for clients.
The NSAK device establishes an uplink connection to an external network interface.
Traffic forwarding is enabled between the wireless and uplink interfaces, enabling transparent internet access.


Network address translation and packet forwarding are configured dynamically during scenario execution.
This enables the NSAK device to operate as an intermediary between wireless clients and the upstream network.


In parallel, a \textbf{traffic capture drill} on the connected interface captures traffic passing through.
The pcap files can be used for later analysis, enabling the evaluation of client behavior and the network.
interactions.

By separating packet capture into an independent drill, the framework allows traffic monitoring to
be reused across different scenarios without modification.

The scenario manager orchestrates the execution of all drills involved in the Rogue Access Point scenario.
Drills are executed in a predefined order, ensuring that the required network services are available before
dependent components are started.


\textbf{Error handling and cleanup}\
To prevent persistent system modifications, each drill defines a cleanup routine that can be executed after scenario
completion or upon failure.
This ensures that network interfaces and system services are restored to their original state.
In the current state of the POC the cleanup functionality need to be adjusted for the broad diversity of the drills
and covers momentarily not all possible edge cases.

But as mentioned in, a centralized cleanup mechanism ensures that partial execution states do not persist in the system
in an inconsistent configuration.
And helps to prevent uncontrolled behaviors of drills.

\textbf{}
This section focused on the technical realization of the Rogue Access Point scenario.
The effectiveness and behavior of the scenario under real network conditions are evaluated in the subsequent evaluation
chapter.
