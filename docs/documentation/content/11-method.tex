2.4 Methoden
Im Methodenkapitel wird aufgezeigt, was (Material, Daten) wie (Methode) untersucht wurde, um die Fragestellung zu beantworten. Das
Ziel des Kapitels ist dabei, die Nachvollziehbarkeit und Überprüfbarkeit der Untersuchung für spätere oder weiterführende Arbeiten zu
gewährleisten. Dafür muss das methodische Vorgehen wiederholbar festgehalten werden. Das Methodenkapitel hat dabei einen ebenso
hohen Stellenwert wie das Ergebniskapitel, denn nur nachvollziehbar erhobene Ergebnisse sind verlässliche Ergebnisse. Was genau im
Methodenkapitel festgehalten wird, hängt stark von der methodischen Ausrichtung der Arbeit ab.
Je nach gewählter Methodik sollten folgende Inhalte festgehalten werden:


– Welches Material wurde untersucht? (Proben, Prüfkörper, Objekte, Bauteile, Elemente, Software usw.)
– Mit welchen Geräten, Vorrichtungen, Werkzeugen usw. wurde gearbeitet?
– Welcher Probenumfang wurde untersucht? (Stichprobe, Gewährspersonen usw.)
– Wie wurden die Daten erhoben?
– Wie wurden die Daten ausgewertet? (statistische Testverfahren, Software usw.)
– Wo und wie wurde Literatur gesucht, welche wichtigen Quellen wurden verwendet, wie und warum wurden sie ausgewählt?
