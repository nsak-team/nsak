\section{Framework Concepts}\label{sec:framework-concepts}

This section describes the high-level concepts, resources and vocabulary needed to understand and work with the \acrshort{nsak} framework.

Overview of the \acrshort{nsak} resources and concepts:
\begin{itemize}
    \item Devices
    \item Environments
    \item Drills
    \item Scenarios
    \item Operator
    \item Operation
\end{itemize}

\subsection{Devices}\label{subsec:devices}
Under a \textbf{device} we understand a physical or virtual machine, which is capable of running the \acrshort{nsak} framework.
Even though we currently only work and describe the hardware devices evaluated in the section Hardware Selection~\ref{sec:hardware-selection}, other devices or virtual machines could be used with \acrshort{nsak}\@.

The following list vaguely describes the minimum requirements for a device:
\begin{itemize}
    \item Processor architecture: ARM and x86 should work equally well, as the \acrshort{nsak} framework is written in python and the scenarios are \acrshort{oci} images/containers, which are built on the \acrshort{nsak} device.
    \item Capable of running a Linux-based operating system, such as Debian.
    \item Enough memory and compute resources to run multiple \acrshort{oci} containers.
    \item Ideally, multiple physical network ports and Wi-Fi for covering many scenarios and environments.
    \item Optionally, additional bulk storage for data collection, such as PCAPs via T-Shark.
\end{itemize}

\textbf{Provisioning a \acrshort{nsak} device} usually consists of the following tasks:
\begin{enumerate}
    \item Install and configure a Linux-based operating system
    \item Set up a minimal network configuration and SSH access
    \item Install system dependencies required for \acrshort{nsak}
    \item Install and configure \acrshort{nsak}
\end{enumerate}

After a device is provisioned, we refer to it as a \textbf{\acrshort{nsak} device}, which may or may not be prepared for an operation.

\subsection{Environments}\label{subsec:environments}
An \textbf{environment} is representing a specific network topology including infrastructure components, servers, clients and services.
Ideally, an environment describes a part or a subset of a network and system infrastructure like you would encounter in a real organization.

Examples of environments:
\begin{itemize}
    \item WLAN AP: Smartphone, WLAN AccessPoint, Router
    \item Client - server: Client, Server, Switch
    \item Home network: Router, WLAN, One Physical Network (Star Topology), Multiple Devices (Computers, Laptops, SmartPhones, SmartTVs)
    \item Business network: Firewall, Router, DC Server, Intranet, Multiple Subnets, Multiple WLAN Access points, Switches
\end{itemize}

\subsection{Drills}\label{subsec:drills}
A \textbf{drill}, initially called a module, is a sequence of actions with a specific goal.
This goal can be an active or passive attack, network discovery, monitoring, analysis, data extraction, a hook for manual intervention or a device configuration.

Examples of drills:
\begin{itemize}
    \item Network sniffing with TShark with a specific filter (http traffic)
    \item Data extraction on an internal bulk storage or external network file system
    \item Active or passive \acrshort{mitm} (man in the middle) attack with a transparent \acrshort{tcp} proxy
    \item \acrshort{arp} Spoofing
    \item WLAN SSID spoofing
    \item Network discovery with nmap or \acrshort{arp}-scan
    \item Network configuration, such as enabling IP-Forwarding or NAT
\end{itemize}

\subsection{Scenarios}\label{subsec:scenarios}
A \textbf{scenario} is designed for one or multiple environments, consists of a sequence of drills and describes a concrete use case for specific red or blue team activities.

Examples of Scenarios:
\begin{itemize}
    \item WLAN SSID Spoofing:
    \begin{itemize}
        \item Environment: WLAN AP
        \item Drills: Network configuration for DHCP, NAT, SSID Spoofing, Packet Sniffing
    \end{itemize}
    \item \acrshort{tcp} \acrshort{mitm} Attack:
    \begin{itemize}
        \item Environment: \acrshort{tcp} client - server
        \item Drills: Automatic network discovery and configuration, \acrshort{arp} Spoofing, Transparent \acrshort{tcp} Proxy, Packet manipulation
    \end{itemize}
\end{itemize}

\subsection{Operator}\label{subsec:operator}
For simplicity and consistency we use the term \textbf{operator} for the person or team, which is planning and executing operations with \acrshort{nsak}\@.
So an operator can refer to a single IT-specialist, a red, blue or purple team.

Examples of Operators:
\begin{itemize}
    \item A single IT-Specialized or Security researcher
    \item System and network engineering teams
    \item Usually, red, blue and purple teams, but potentially all teams in the InfoSec color wheel~\cite{cremen2018infoseccolourwheel}
\end{itemize}

\subsection{Operation}\label{subsec:operation}
An \textbf{operation} is the deployment of \acrshort{nsak} in a real network.

An operation explicitly excludes the development phase for scenarios, drills and environments, as these resources should be finalized and tested before being used in a real operation, otherwise the following conventions should be used:
\begin{itemize}
    \item \textbf{Simulated Operation:} Simulating an operation in a virtualized environment.
    \item \textbf{Test Operation:} Testing an operation in a physical lab network.
\end{itemize}

Preparing and planning an operation usually has the following sequence of tasks, assessed and executed by an operator:
\begin{enumerate}
    \item Provision a \acrshort{nsak} device.
    \item Select one or multiple environments which are relevant for the target network and system infrastructure.
    \item Configure and build all or a subset of scenarios which can be executed in the selected environments.
    \item Ideally, simulate and test the operation in a virtual or lab network infrastructure.
\end{enumerate}
