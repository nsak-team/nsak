\section{Sequence-Diagram}\label{sec:sequence-diagram}

\begin{figure}[H]
    \centering
    \includegraphics[width=0.9\linewidth]{../assets/figures/diagrams/concept-architecture-diagrams/sequence-diagram}
    \caption{Sequence of a Scenario loading and drill execution}
    \label{fig:sequence-diagram.drawio}
\end{figure}

\textbf{Figure~\ref{fig:sequence-diagram.drawio}} shows the interaction sequence for building,
loading, and executing a scenario within NSAK.
The diagram focuses on the main modularity concept and describes the orchestration flow between scenarios
and drills without interface details, error-handling, and drill or scenario clean-up mechanisms.

The process begins with the operator triggering the build command for a scenario container.
During this phase, the selected scenario is loaded and returned as a containerized representation.
In the execution phase, the scenario manager runs the container image and orchestrates the Drills order.
The Drill Manager executes the required drills.

A scenario may contain multiple drills; therefore, the * signalize various drills can be executed from a Drill Manager
in a one scenario.
Each drill is resolved and executed individually, while the scenario manager maintains complete control
over the scenario lifecycle.

Finally, each drill is intended to provide a clean-up procedure that can be triggered from the scenario manager.
All drill-specific artifacts should be removed from the host system to enable the sequential execution of multiple
scenarios.
The clean-up process is a planned extension of the framework.
