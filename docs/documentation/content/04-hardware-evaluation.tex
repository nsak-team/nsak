\subsection{Hardware Requirements}\label{subsec:requirements}

The following requirements were defined for the hardware platform used in this project:
\begin{itemize}
	\item At least two native Ethernet interfaces for inline packet sniffing
	\item Support for 2.5~GbE or higher
	\item Onboard Wi-Fi with access point (AP) and monitor mode support
	\item Low power consumption suitable for 24/7 operation
	\item Compact form factor for laboratory and prototype setups
	\item Strong community and software support
	\item Affordable cost (below 150 CHF)
\end{itemize}
\newpage

\subsection{Evaluated Boards}\label{subsec:evaluated-boards}

Several boards were considered as potential variants.
Their main specifications relevant to the project are listed in Table~\ref{tab:boardspecs}.

\begin{table}[H]
	\centering
	\caption{Comparison of Board Variants}
	\label{tab:boardspecs}
	\scriptsize % kleinere Schrift für den Inhalt
	\begin{tabularx}{\textwidth}{|X|X|X|X|X|X|}
		\hline
		\textbf{Board} & \textbf{SoC / CPU} & \textbf{RAM / Storage} & \textbf{Ethernet Ports} & \textbf{Power (typ.)} & \textbf{Wireless (onboard)} \\
		\hline
		Banana Pi R3 Mini & MT7986A, Quad-core ARM Cortex-A53 @ 1.3 GHz & 2 GB DDR4, 8 GB eMMC, microSD & 2 × 2.5 GbE & 5–7 W & MT7976C, Wi-Fi 6 (AP/Client/Monitor) \\
		\hline
		Banana Pi R3 & MT7986A, Quad-core ARM Cortex-A53 @ 1.3 GHz & 2–4 GB DDR4, eMMC, microSD & 1 × 1 GbE, 2 × 2.5 GbE, 4 × 1 GbE & 7–10 W & MT7976C, Wi-Fi 6 \\
		\hline
		Banana Pi R4 & MT7988A, Quad-core ARM Cortex-A73 & 4 GB DDR4, NVMe option & 4 × 2.5 GbE, 2 × 10 GbE (SFP+) & 10–15 W & None (M.2 Wi-Fi module required) \\
		\hline
		Banana Pi R5 & MT7988B, Quad-core ARM Cortex-A73 & 4 GB DDR4, NVMe option & 2 × 10 GbE, 2 × 2.5 GbE & 12–18 W & None (M.2 Wi-Fi module required) \\
		\hline
		Raspberry Pi 4 & BCM2711, Quad-core ARM Cortex-A72 @ 1.5 GHz & 2–8 GB LPDDR4, microSD & 1 × 1 GbE (second via USB dongle) & 6–8 W & Wi-Fi 5 (AP/Client only) \\
		\hline
		Raspberry Pi 5 & BCM2712, Quad-core ARM Cortex-A76 @ 2.4 GHz & 4–8 GB LPDDR4X, microSD & 1 × 1 GbE (second via PCIe card) & 8–12 W & Wi-Fi 5 (AP/Client only) \\
		\hline
		NanoPi R76S & Rockchip RK3588S, Octa-core (4× Cortex-A76 @ 2.4 GHz + 4× Cortex-A55 @ 1.8 GHz) & 16 GB LPDDR4X / LPDDR5, NVMe (option via M.2) & 3 × 2.5 GbE (RJ45) & 10–15 W & None (M.2 Wi-Fi 6E module recommended) \\
		\hline

	\end{tabularx}
\end{table}

\begin{table}[H]
	\centering
	\caption{Requirements Fulfillment by Candidate Boards}
	\label{tab:reqcheck}
	\resizebox{\textwidth}{!}{%
		\begin{tabular}{|p{6cm}|c|c|c|c|c|c|c}
			\hline
			\textbf{\large Requirement} & \textbf{\large R3 Mini} & \textbf{\large R3} & \textbf{\large R4} & \textbf{\large R5} & \textbf{\large RPi 4} & \textbf{\large RPi 5} & \textbf{NanoPi R76S}\\
			\hline
			\scriptsize ≥ 2 native Ethernet interfaces & \ding{51} & \ding{51} & \ding{51} & \ding{51} & \ding{55} & \ding{55} & \ding{51} \\
			\hline
			\scriptsize RAM > 4GB & \ding{55} & \ding{55} & \ding{51} &\ding{51} &
			\ding{55} & \ding{51} & \ding{51} \\
			\hline
			\scriptsize ≥ 2.5 GbE support & \ding{51} (2×) & \ding{51} (2×) & \ding{51} \ding{51} (4×) & \ding{51} (2×) & \ding{55} & \ding{55} & \ding{51} \\
			\hline
			\scriptsize Onboard Wi-Fi with AP \& Monitor mode & \ding{51} & \ding{51} & \ding{55} & \ding{55} & \ding{55} & \ding{55} & \ding{55} \\
			\hline
			\scriptsize Low power consumption (<10 W) & \ding{51} & \ding{51}/\ding{115} & \ding{55} & \ding{55} & \ding{51} & \ding{115} & \ding{51}\\
			\hline
			\scriptsize Compact form factor & \ding{51} & \ding{55} & \ding{55} & \ding{55} & \ding{51} & \ding{51} & \ding{51}\\
			\hline
			\scriptsize Strong community \& software support & \ding{51} & \ding{51} & \ding{115} & \ding{115} & \ding{51} (general) & \ding{51} (general) & \ding{51}\\
			\hline
			\scriptsize Suitable for inline packet sniffing & \ding{51} & \ding{51} (overkill) & \ding{115} (overkill) & \ding{115} (expensive) & \ding{55} & \ding{55} & \ding{51 }\\
			\hline
		\end{tabular}
	}
	\vspace{0.5em}
	\scriptsize
	\textbf{Legend:}~\ding{51} = Requirement fulfilled, \ding{55} = Requirement not fulfilled, \ding{115} = Partially fulfilled / limited
\end{table}

\subsection{Decision}\label{subsec:decision}

Based on the defined requirements and the evaluation of alternatives, the \textbf{Banana Pi R4} and the\textbf{NanoPI R76S} are the most suitable hardware platforms for this prototype implementation.

The Banana Pi R4 offers two native 2.5~GbE interfaces for inline sniffing the board is compact, affordable, and supported by a strong community.
In Addition, the two 10 GbE SFP+ ports provide flexibility for extensions as fiber-based packet capturing.
A drawback of the R4 is the weaker CPU and a larger size compared to the NanoPI R76S

The NanoPi R76S is more compact and provides up to 16GB of RAM, which is advantageous for memory-intensive processing and buffering tasks.
While it lacks built-in Wi-fi, it can be expanded via the M.2 Wi-Fi 6E module.
It cannot host both a Wi-Fi card and NVMe SSD simultaneously.
Consequently, data storage must be provided via microSD card or external USB SSD

Alternative boards such as the Banana Pi R3 Mini, R3 are limited overall performance.
Raspberry PI 4 or 5 offer higher single core performance but were ultimately discarded because they provide only a single native Ethernet interface, requiring external adapters that reduce performance for inline sniffing scenarios.
