\section{Conclusion}\label{sec:conclusion}

This thesis presents a conceptual design and implementation of the \acrfull{nsak}.
The \acrshort{nsak} framework is modular and scenario-based security software for controlled network security and
experimentation.


The proposed architecture combines container-based isolation with reusable drills.
These drills can be composed into high-level scenarios.
The implementation demonstrated the core design objectives of the framework, and the evaluation emphasizes the
modularity, reproducibility, and container-based isolation.
Independent drills were successfully orchestrated in two differential environments, in line, on the data link layer,
with an \acrshort{arp} spoofing \acrshort{mitm} scenario and over a Wi-Fi network interface to capture traffic as a rogue access point.

The results indicate that the \acrshort{nsak} framework can reduce setup complexity for network security experiments while maintaining
a high degree of flexibility.

At the same time, this work is merely a proof of concept with clear limitations.
Supporting more sophisticated attack scenarios in complex networks requires further refinement and extension.
Further improvements include advanced cleanup mechanisms, richer configuration models, and automated assessment of
defensive responses.

In conclusion, the \acrshort{nsak} framework provides a promising foundation stone for future extension
and open source contribution.
The independent core backend and the isolated framework structure are supporting developing encapsuled drills for embedded devices
without compromising other devices.
