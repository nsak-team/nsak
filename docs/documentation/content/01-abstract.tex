
We describe NSAK as an embedded, modular, open source, scenario-based network sniffing and security framework.
We provide an overview of the system design, emphasizing the functionality of a Swiss Army Knife in a networking context.
The attack scenario includes drills that configure or target a specific task to observe or open an attack vector in the system.
NSAK is based on a core back-end part that manages specific environments, scenarios, and drills.
The whole concept is based on containerization, where each container builds or prepares a scenario that can be triggered in a network environment.
In the analogy of the Swiss Army Knife, the right knife with the proper drills for the necessary task can be selected.
Modularity comes into play when a drill is selected across multiple scenarios.

Keywords
Network intrusion detection, Network Monitoring, Red/Blue Team


\todo{Mit Regeln von Abstract gegenchecken und querlesen}

\textbf{Rules to apply}

One-paragraph summary of the entire study – typically no more than 250 words in length (and in many cases it is well shorter than that), the Abstract provides an overview of the study.


Inhalt Abstract
– Hintergrundinformationen wie Ausgangslage, Relevanz, Forschungskontext in ein bis zwei Sätzen zusammenfassen.
– Fragestellung und Ziel explizit formulieren.
– Die wichtigsten Eckpunkte zum methodischen Vorgehen angeben, bei empirischen Studien auch Angaben zu den Daten wie
etwa die Charakteristika der Stichprobe.
– Im Hauptteil des Abstracts die relevanten Ergebnisse und deren Bedeutung mit wichtigen Kennzahlen aufführen (ca. zwei Drittel des Abstracts).
– Mit wichtigen Schlussfolgerungen oder Anwendungsmöglichkeiten das Abstract abrunden.
– Das Abstract enthält keine Quellenverweise
\newline
\newline
