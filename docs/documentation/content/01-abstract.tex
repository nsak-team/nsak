We describe \acrfull{nsak} as a modular, open-source, scenario-based network security framework,
which is designed to be deployed on embedded hardware.
We provide an overview of the system design, emphasizing the functionality of a Swiss Army Knife in a networking context.
A \acrshort{nsak} scenario describes a common red or blue team activity for specific network environments and
provides an automated and configurable implementation, based on multiple drills.
Drills are the modular and configurable technical building blocks that can be used across multiple scenarios to configure the \acrshort{nsak} device or execute a specific task for reconnaissance or exploitation in the target network.
In the analogy of the Swiss Army Knife, the right knife with the proper drills can be selected to solve a certain task.
The \acrshort{nsak} framework is based on a core backend that implements an API to manage the framework configuration and resources such as environments, scenarios, and drills.
The provided core-API is used by the CLI implementation to enable user interaction and could be used for other frontends and systems in the future.
Another important concept of the framework is containerization, where each scenario can be built as an \acrshort{oci} container and executed on a provisioned \acrshort{nsak} device in a network environment.
To prove that the concepts are working in the real world, we evaluated two different embedded ARM-based systems and executed experiments with the implemented scenarios in physical lab environments.
The results of these experiments indicate that \acrshort{nsak} is well suited for security experimentation and controlled attack simulations on
constrained systems.
On the other hand, it showed that the automation of such scenarios requires a lot of assumptions about the target environment, which potentially renders them very brittle.
The framework provides a practical foundation for network security research and can be extended to support automated
testing and future attack scenarios.

Keywords: Modular Security Automation Framework, Network Reconnaissance, Exploitation, Red/Blue Team
