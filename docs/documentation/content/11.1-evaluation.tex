\section{Evaluation}\label{sec: Evaluation}

\subsection{Generel Functionality of a Rogue Access Point}\label{subsec: General Functionality}
The Rogue Access Point scenario was executed with different combinations of drills to test the modularity and the
independent drill execution.
The expected outcome is the creation of an open Wi-fi access point broadcasting the configured SSID,
providing DHCP leases and forwarding traffic to the uplink interface.
Additionally, traffic will be captured in PCAP format.


During execution, the access point was successfully created, and client devices could connect using the assigned IP address.
via DHCP.
The Network traffic was captured in a PCAP file.

\subsection{Research Questions}\label{subsec: Research Questions}
\textbf{RQ1 – Modularity}\label{rq1}
RQ1: Can network security scenarios be composed of independent drills without modifying the core framework?


To ensure modularity, every drill should be able to work independently.
When setting the DISABLE-DRILL environment variable, remaining drills should execute independently without failure.
For example, if everything is disabled beside the ap-mode drill, the network should still be discoverable from clients.
In T4~\ref{tab:modular-execution}, IP forwarding, uplink, and capture should not be executed and therefore are not working.

During the process, each drill has a specific attack goal or functionality.
No implicit dependencies between drills were observed during execution.


\begin{table}[H]
    \centering
    \begin{tabularx}{\textwidth}{l X X}
        \toprule
        \textbf{Test} & \textbf{Description} & \textbf{Expectation} \\
        \midrule
        T1 & Execute Rogue AP scenario with full drill set & Scenario executes successfully \\
        T2 & Deactivate packet capture drill & AP still functions \\
        T3 & Deactivate dnsmasq drill & All drills are working, tshark cannot capture traffic \\
        T4 & Only the AP-mode drill is enabled & Client can see the SSID and connect \\
        \bottomrule
    \end{tabularx}
    \caption{Modular execution test cases}
    \label{tab:modular-execution}
\end{table}

\textbf{RQ2 – Reusability}\label{rq2}
RQ2: Can individual drills be reused across different scenarios with minimal configuration effort?


The hostapd drill was reused in two scenarios: a standalone access point and a rogue access point.
access point scenario.
In both cases, the drill implementation remained unchanged.
Scenario-specific behavior was controlled through environment variables such as SSID, channel, and country code.


These results confirm that drills are scenario-orchestrated and can be composed flexibly with minimal modification.

\textbf{RQ3 – Containerization}\label{rq3}
RQ3: Does container-based execution provide sufficient isolation while maintaining required network functionality?


The evaluation focuses on whether container-based execution restricts or enables low-level network operations,
required for the functionality of \acrshort{nsak}.


The Rogue Access Point scenario was executed inside a containerized \acrshort{nsak} environment on an embedded Linux device.
The container was granted only the necessary capabilities for network configuration and packet capture.


During scenario execution, all network services were started in privileged mode within the container environment.
During container-based execution, network interfaces, routing rules, and packet capture were successfully initialized.
The drills ap-mode, network setup, dnsmasq, t-shark left certain config files and rules in iptables and filepaths.

\textbf{RQ4 – Reproducibility}\label{rq4}
RQ4: Can scenarios be executed repeatedly without changing the system behavior?


With the same configuration, the scenario execution was triggered multiple times.
Devices could establish a connection with the access point.
Network connectivity was enabled in all runs.
And the \acrshort{nsak} framework was able to capture the traffic.


No deviations in system behavior were observed across repeated executions.
