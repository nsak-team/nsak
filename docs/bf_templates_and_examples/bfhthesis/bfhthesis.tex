%============================ MAIN DOCUMENT ================================
% define document class
\PassOptionsToPackage{table}{xcolor}
\documentclass[
  a4paper,
  BCOR=15mm,            % Binding correction
  twoside,
% openright,
%  headings=openright,
  bibliography=totoc,   % If enabled add bibliography to TOC
  listof=totoc,         % If enabled add lists to TOC
  monolingual,
% bilingual,
  invert-title,
]{bfhthesis}

\LoadBFHModule{listings,terminal,boxes}
%---------------------------------------------------------------------------
% Documents paths
%---------------------------------------------------------------------------
\makeatletter
\def\input@path{{content/}}
%or: \def\input@path{{/path/to/folder/}{/path/to/other/folder/}}
\makeatother
%-----------------  Base packages     --------------------------------------
% Include Packages
\usepackage[french,ngerman,main=english]{babel}  % https://www.namsu.de/Extra/pakete/Babel.html
%% To disable the french list setting you can add -- see https://gitlab.ti.bfh.ch/bfh-latex/bfh-ci/-/issues/166
%\frenchsetup{StandardLists=true}

\usepackage{amsmath}          % various features to facilitate writing math formulas
\usepackage{amsthm}           % enhanced version of latex's newtheorem
\usepackage{amsfonts}         % set of miscellaneous TeX fonts that augment the standard CM
\usepackage{amssymb}          % mathematical special characters

\usepackage{siunitx}

\usepackage{graphicx}         % integration of images
\usepackage{float}            % floating objects

\usepackage{caption}          % for captions of figures and tables
\usepackage{subcaption}       % for subcaptions in subfigures
\usepackage{cite}             % use bibtex
\usepackage{wrapfig}

\usepackage{exscale}          % mathematical size corresponds to textsize
\usepackage{multirow}         % multirow emables combining rows in tables
\usepackage{multicol}

\usepackage{longtable}

\usepackage{parskip}

%---------------------------------------------------------------------------
% Graphics paths
%---------------------------------------------------------------------------
\graphicspath{{pictures/}{figures/}}
%---------------------------------------------------------------------------
% Blind text -> for dummy text
%---------------------------------------------------------------------------
\usepackage{blindtext}    
\usepackage{letltxmacro}   
\LetLtxMacro{\blindtextblindtext}{\blindtext}

\RenewDocumentCommand{\blindtext}{O{\value{blindtext}}}{
	\begingroup\color{BFH-Gray}\blindtextblindtext[#1]\endgroup
}
%---------------------------------------------------------------------------
% Glossary Package
%---------------------------------------------------------------------------
% the glossaries package uses makeindex
% if you use TeXnicCenter do the following steps:
%  - Goto "Ausgabeprofile definieren" (ctrl + F7)
%  - Select the profile "LaTeX => PDF"
%  - Add in register "Nachbearbeitung" a new "Postprozessoren" point named Glossar
%  - Select makeindex.exe in the field "Anwendung" ( ..\MiKTeX x.x\miktex\bin\makeindex.exe )
%  - Add this [ -s "%tm.ist" -t "%tmx.glg" -o "%tm.gls" "%tm.glo" ] in the field "Argumente"
%
% for futher informations go to http://ewus.de/tipp-1029.html
%---------------------------------------------------------------------------
\usepackage[nonumberlist]{glossaries-extra}
\makeglossaries
\newacronym{arp}{ARP}{Address Resolution Protocol}

\newglossaryentry{BlueTeam}{
    name=Blue Team,
    description={The blue team is responsible for protective measures within an organization; during an exercise, they detect and defend against red team engagements.}
}

\newacronym{cli}{CLI}{Command Line Interface}

\newacronym{dhcp}{DHCP}{Dynamic Host Configuration Protocol}

\newacronym{dns}{DNS}{Domain Name System}

\newglossaryentry{GreenTeam}{
    name=Green Team,
    description={The green team represents the cooperation between the blue and yellow teams to close knowledge gaps in both areas.}
}

\newacronym{ids}{IDS}{Intrusion Detection System}

\newacronym{ips}{IPS}{Intrusion Prevention System}

\newacronym{mitm}{MITM}{Man-in-the-Middle}

\newglossaryentry{ProxyServer}{
  name=Proxy Server,
  description={An intermediary service between a client and a server.}
}

\newacronym{nsak}{NSAK}{Network Swiss Army Knife}

\newacronym{oci}{OCI}{Open Container Initiative}

\newglossaryentry{OrangeTeam}{
    name=Orange Team,
    description={The orange team represents the cooperation between the red and yellow teams, primarily for sharing insights, knowledge, and education.}
}

\newglossaryentry{PurpleTeam}{
    name=purple team,
    description={Purple teaming describes the cooperation between red and blue teams to enhance mutual understanding and defensive capabilities.}
}

\newglossaryentry{RedTeam}{
    name=red team,
    description={The red team consists of ethical hackers who simulate attacks against systems, networks, and software to test defensive measures.}
}

\newacronym{rogueap}{RAP}{Rogue Access Point}

\newacronym{siem}{SIEM}{Security Information and Event Management}

\newacronym{soc}{SOC}{Security Operations Center}

\newacronym{ssid}{SSID}{Service Set Identifier}

\newacronym{tcp}{TCP}{Transmission Control Protocol}

\newacronym{udp}{UDP}{User Datagram Protocol}

\newglossaryentry{WhiteTeam}{
    name=White Team,
    description={The white team consists of non-technical and technical stakeholders who provide oversight, compliance guidance, and organizational requirements during an exercise.}
}

\newglossaryentry{YellowTeam}{
    name=Yellow Team,
    description={The yellow team consists of system, network, and software architects and engineers who design and maintain the infrastructure.}
}

\newacronym{api}{API}{Application Programming Interface}

\newacronym{rest}{REST}{Representational State Transfer}

\newacronym{yaml}{YAML}{YAML Ain't Markup Language}

\newacronym{git}{GIT}{Git is a free and open source distributed version control system designed to handle everything from small to very large projects with speed and efficiency}

\newacronym{mac}{MAC}{Media Access Control}

%---------------------------------------------------------------------------
% Makeindex Package
%---------------------------------------------------------------------------
\usepackage{makeidx}
\makeindex
%\usepackage{imakeidx}          % To produce index
%\makeindex[columns=2,intoc]    % Index-Initialisation
%\makeindex[columns=3,columnseprule,columnsep,intoc]
%---------------------------------------------------------------------------
% Hyperref Package (Create links in a pdf)
%---------------------------------------------------------------------------
\usepackage[
	,bookmarks
	,plainpages=false
	,pdfpagelabels
        ,pdfusetitle
	,backref = {false}          % No index backreference
	,colorlinks = {true}        % Color links in a PDF
	,hypertexnames = {true}     % no failures "same page(i)"
	,bookmarksopen = {true}     % opens the bar on the left side
	,bookmarksopenlevel = {0}   % depth of opened bookmarks
	,linkcolor=.
	,filecolor=.
	,urlcolor=.
	,citecolor=.
]{hyperref}
%---------------------------------------------------------------------------

%% %% Customize Footer and Headers in Document
%% \KOMAoptions{headsepline,plainheadsepline,footsepline,plainfootsepline}%
%% \setkomafont{headsepline}{\color{BFH-DarkBlue}}% BFH-DarkBlue required bfhcolors
%% \setkomafont{footsepline}{\color{BFH-DarkBlue}}%
%% \lehead*{lehead} % the * character does replace the header on the first chapter page as well
%% \cehead*{cehead}
%% \rehead*{rehead}
%% \lohead*{lohead}
%% \cohead*{cohead}
%% \rohead*{rohead}

%% \lefoot*{lefoot}
%% \cefoot*{cefoot}
%% \refoot*{refoot}
%% \lofoot*{lofoot}
%% \cofoot*{cofoot}
%% \rofoot*{rofoot}
%---------------------------------------------------------------------------
\begin{document}

%% Snippet do redefine German babel translations -- see FAQ latex.ti.bfh.ch for further information or
%% the package documentation https://www.ctan.org/pkg/translations
%% \redefinetranslation{German}{Advisor}{Dozentin}
%% \redefinetranslation{German}{advisor}{Dozentin} %  just for consistency
%% \redefinetranslation{German}{Author}{Autorin}
%% \redefinetranslation{German}{author}{Autorin} %  just for consistency
%% \redefinetranslation{German}{Co-advisor}{Mitbetreuerin}
%% \redefinetranslation{German}{co-advisor}{Mitbetreuerin} %  just for consistency
%% \redefinetranslation{German}{Expert}{Expertin}
%% \redefinetranslation{German}{expert}{Expertin} %  just for consistency
%% \redefinetranslation{German}{Project partner}{Projektpartnerin}
%% \redefinetranslation{German}{project partner}{Projektpartnerin} %  just for consistency

%------------ START FRONT PART ------------
\frontmatter

\title{Bachelor's Thesis}
\subtitle{Thesis subject}
\author{Anton Muster \and Cindy Example}
\institution{Bern University of Applied Sciences}
\department{Technik und Informatik}
\institute{Mikro- und Medizintechnik}
\version{1.0}
\titlegraphic*{\includegraphics{somePicture}}
\advisor{Prof. Dr. Super Smart}
\coadvisor{PhD A. Smart}
\projectpartner{proj partner}
\expert{Some expert}
\degreeprogram{Bachelor of Science in Computer Science}
\setupSignature{
	A. Muster={\includegraphics[width=.5\linewidth]{sig_muster}},
	C. Example={\includegraphics[width=.5\linewidth]{sig_example}}
}


%----------------  BFH tile page   -----------------------------------------
\maketitle
%------------ ABSTRACT        ----------------
\addchap{Abstract}
\input{content/abstract}

%------------ TABLEOFCONTENTS ----------------
\tableofcontents

%------------ START MAIN PART ------------
\mainmatter

\chapter{First Thesis Chapter}
\section{Introduction}

What is the topic and why is it worth studying? – the first major section of text in the paper, the Introduction commonly describes the topic under investigation, summarizes or discusses relevant prior research (for related details, please see the Writing Literature Reviews section of this website), identifies unresolved issues that the current research will address, and provides an overview of the research that is to be described in greater detail in the sections to follow.

\section{Methods}
What did you do? – a section which details how the research was performed.  It typically features a description of the participants/subjects that were involved, the study design, the materials that were used, and the study procedure.  If there were multiple experiments, then each experiment may require a separate Methods section.  A rule of thumb is that the Methods section should be sufficiently detailed for another researcher to duplicate your research.

\section{Results}
What did you find? – a section which describes the data that was collected and the results of any statistical tests that were performed.  It may also be prefaced by a description of the analysis procedure that was used. If there were multiple experiments, then each experiment may require a separate Results section.


\chapter{Second Thesis Chapter}
\input{content/expl_fragments}
\section{Discussion}
What is the significance of your results? – the final major section of text in the paper.  The Discussion commonly features a summary of the results that were obtained in the study, describes how those results address the topic under investigation and/or the issues that the research was designed to address, and may expand upon the implications of those findings.  Limitations and directions for future research are also commonly addressed.


%------------ Authorship declaration translated to main language ------------
\declarationOfAuthorship

%----------- Bibliography ----------------
\clearpage
\bibliographystyle{unsrt}
\bibliography{project}      % the project.bib file gets loaded

%------------ List of Figures ------------
\listoffigures
 
%------------ List of Tables -------------
\listoftables
 
%------------ List of Listings -----------
\lstlistoflistings 
 
%------------ Glossary -------------------
\printglossary

%------------ Index ----------------------
\clearpage
\printindex
%------------ Appendix ----------------	
\appendix
\chapter{First Appendix Chapter}

\end{document}
