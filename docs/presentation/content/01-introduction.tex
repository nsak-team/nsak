
\begin{frame}{Motivation}
	\framesubtitle{Subtitle}
\begin{itemize}
	\item \textbf{Cyberangriffe nehmen zu}
	\small Angriffsbarrieren sinken, Automatisierung steigt
	\vspace{0.3em}

	\item \textbf{Sicherheit erfordert Sichtbarkeit}
	\small Netzwerkangriffe sind oft nur auf Layer 2–4 erkennbar
	\vspace{0.3em}

	\item \textbf{Bestehende Frameworks sind komplex}
	\small Hoher Konfigurations- und Betriebsaufwand
	\vspace{0.3em}

	\item \textbf{Unser Ansatz (PoC)}
	\small Modularer, kontrollierter Network-Sniffer für Angriffsszenarien


\end{itemize}
	Eher alles als Bilder
\end{frame}

\begin{frame}{Warum Network Sniffing}
	Color Wheel?

\end{frame}
\begin{frame}{Hardware Selektion}
	\framesubtitle{Was braucht so ein Board}
	\begin{itemize}
		\item At least two native Ethernet interfaces for inline packet sniffing
		\item Support for 2.5~GbE or higher
		\item Onboard Wi-Fi with access point (AP) and monitor mode support
		\item Low power consumption suitable for 24/7 operation
		\item Compact form factor for laboratory and prototype setups
		\item Strong community and software support
		\item Affordable cost (below 150 CHF)
	\end{itemize}
	 Bild R4 und Nano PI
\end{frame}
\begin{frame}{Was ist der Swiss Army Network Sniffer}
	\framesubtitle{NSAK Concepts}

		\begin{itemize}
			\item[\faServer] \textbf{Devices}
			\small Physical machines used as attack and target hosts

			\vspace{0.4em}


			\item[\faNetworkWired] \textbf{Environments}
			\small Network infrastructure and topology

			\vspace{0.4em}
					\item[\faLayerGroup] \textbf{Scenarios}
			\small Sequence of drills (e.g.\ ARP spoofing, Packet Capture)

			\vspace{0.4em}

			\item[\faCogs] \textbf{Drills}
			\small Individual attack or observation steps

			\vspace{0.4em}


			\item[\faUserShield] \textbf{Operator}
			\small Red / Blue team

			\vspace{0.4em}

		\end{itemize}

\end{frame}
