
\begin{frame}{Future Work and Conclusion}
	\framesubtitle{Key Insights and Takeaways}

	\begin{columns}[T]
		\begin{column}{0.5\textwidth}
			Future Work
			\vspace{0.4em}
			\begin{itemize}
				\item[\faCloud] REST API: Integration of frontends or systems
				\vspace{0.4em}
				\item[\faReact] GUI: Better UX and lower entry barrier
				\vspace{0.4em}
				\item[\faBroom] Redesign cleanup management
				\vspace{0.4em}
				\item[\faCogs] Implement configuration management
				\vspace{0.4em}
				\item[\faBug] Test coverage: Correctness and maintainability
				\vspace{0.4em}
				\item[\faShieldVirus] Security Concept: NSAK must be secure
				\vspace{0.4em}
			\end{itemize}
		\end{column}
		\begin{column}{0.5\textwidth}
			Conclusion\\
			\vspace{0.4em}
			\small We could show the feasibility of a PoC realization of a Network Swiss Army Knife (NSAK) and that the modular drills can be combined to effectively build scenarios, which can be executed on constraint hardware in simulated or real-world network environments.\\
			\vspace{0.4em}
			\small The key difficulty is that the automation of scenarios requires a lot of assumptions about the target environment, which potentially renders them very brittle.


%			\begin{itemize}
%				\item Design and implementation of the framework
%				\item Evaluated and configured devices
%				\item Implemented environments, scenarios, drills
%				\item Putting everything together
%				\item Run scenarios in physical and simulated environments
%			\end{itemize}
		\end{column}
	\end{columns}
	\centering
\end{frame}


\section[Access Point Demo: Can you spot the rougue AP? / Open questions?]{Access Point Demo: Can you spot the rogue AP?\\~\\ Open questions?}\label{sec:demo-access-point-and-questions}
