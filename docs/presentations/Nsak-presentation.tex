\documentclass[
	ngerman,%globale Übergabe der Hauptsprache
%	logofile=example-image, %Falls die Logo Dateien nicht vorliegen
	authorontitle=true,
	]{bfhbeamer}

%\usepackage[main=ngerman]{babel}

% Der folgende Block ist nur bei pdfTeX auf Versionen vor April 2018 notwendig
\usepackage{iftex}
\ifPDFTeX
\usepackage[utf8]{inputenc}%kompatibilität mit TeX Versionen vor April 2018
\fi
\usepackage{bytefield}
\usepackage{pifont}
\usepackage{xcolor}
\usepackage{fontawesome5}


%Makros für Formatierungen der Doku
%Im Allgemeinen nicht notwendig!
\let\code\texttt

\title{Swiss Army Knife Network Sniffer (NSAK) }
\subtitle{Version 0.4}
\author[]{Lukas von Allmen (vonal3) and Frank Gauss (gausf1)}
\institute{Bern University of Applied Sciences}
\titlegraphic*{\includegraphics{{assets/swiss-army-network-sniffer}}}%is only used with BFH-graphic and BFH-fullgraphic

%Activate the output of a frame number:
\setbeamertemplate{section page}[BFH-ruled]
\setbeamertemplate{page number in head/foot}[framenumber]
\AtBeginSection{\sectionpage}

\begin{document}

%\maketitle

%second title variant
%\setbeamertemplate{title page}[BFH-Orange]

%\maketitle

\setbeamertemplate{title page}[BFH-graphic]
\maketitle

%\setbeamertemplate{title page}[BFH-fullgraphic]
%\maketitle

\section{Introduction}\label{sec:introduction}

\begin{frame}{Motivation}
	\framesubtitle{Subtitle}
\begin{itemize}
	\item \textbf{Cyberangriffe nehmen zu}
	\small Angriffsbarrieren sinken, Automatisierung steigt
	\vspace{0.3em}

	\item \textbf{Sicherheit erfordert Sichtbarkeit}
	\small Netzwerkangriffe sind oft nur auf Layer 2–4 erkennbar
	\vspace{0.3em}

	\item \textbf{Bestehende Frameworks sind komplex}
	\small Hoher Konfigurations- und Betriebsaufwand
	\vspace{0.3em}

	\item \textbf{Unser Ansatz (PoC)}
	\small Modularer, kontrollierter Network-Sniffer für Angriffsszenarien


\end{itemize}
	Eher alles als Bilder
\end{frame}

\begin{frame}{Warum Network Sniffing}
	Color Wheel?
\end{frame}

\section{Design and Architecture}\label{sec:design-and-architecture}

\begin{frame}{Was ist der Swiss Army Network Sniffer}
	\framesubtitle{NSAK Concepts}

	\begin{itemize}
		\item[\faServer] \textbf{Devices}
		\small Physical machines used as attack and target hosts

		\vspace{0.4em}

		\item[\faNetworkWired] \textbf{Environments}
		\small Network infrastructure and topology

		\vspace{0.4em}
		\item[\faLayerGroup] \textbf{Scenarios}
		\small Sequence of drills (e.g.\ ARP spoofing, Packet Capture)

		\vspace{0.4em}

		\item[\faCogs] \textbf{Drills}
		\small Individual attack or observation steps

		\vspace{0.4em}


		\item[\faUserShield] \textbf{Operator}
		\small Red / Blue team

		\vspace{0.4em}

	\end{itemize}

\end{frame}

\section{Hardware Evaluation}\label{sec:hardware-evaluation}

\begin{frame}{Hardware Selection}
	\framesubtitle{Was braucht so ein Board}
	\begin{itemize}
		\item At least two native Ethernet interfaces for inline packet sniffing
		\item Support for 2.5~GbE or higher
		\item Onboard Wi-Fi with access point (AP) and monitor mode support
		\item Low power consumption suitable for 24/7 operation
		\item Compact form factor for laboratory and prototype setups
		\item Strong community and software support
		\item Affordable cost (below 150 CHF)
	\end{itemize}
	 Bild R4 und Nano PI
\end{frame}

\section{Implementation}\label{sec:implementation}

\begin{frame}{Use Cases / Demo-Szenario}
	\framesubtitle{MITM ARP-spoofing / Transparent TCP Proxy}
	\begin{itemize}
		\item Idee und Konzept auffrischen
		\item umsetzung erklären
		\item Demo
	\end{itemize}
	\centering
\end{frame}

\begin{frame}{Use Cases / Demo-Szenario}
	\framesubtitle{Rogue AP}
	\begin{itemize}
		\item Idee und Konzept auffrischen
		\item umsetzung erklären
		\item Demo am Ende
	\end{itemize}
\end{frame}

\section{Evaluation and Discussion}\label{sec:evaluation_and_discussion}

\begin{frame}{Grenzen und Risiken Future Work}
	\framesubtitle{Subtitle}
	1 Folie
\end{frame}

\begin{frame}{Fazit und Takeaways}
	\framesubtitle{Subtitle}
	1 Folie
	Repo Link
\end{frame}

\begin{frame}{Demo Access Point}
	\framesubtitle{Subtitle}
	Keine Folien
\end{frame}


\begin{frame}{Blocks}
	\begin{block}{Block with a title}
		Content.
	\end{block}
	\begin{block}{}
		Without title
	\end{block}
\end{frame}

\begin{frame}{Block types}
	\begin{exampleblock}{Exampleblock}
		Content.
	\end{exampleblock}
	\begin{alertblock}{Alertblock}
		Content.
	\end{alertblock}
	\begin{example}[Example environment]
		Content.
	\end{example}
\end{frame}

\section{section pages}
%These can be automaticlly called by using \AtBeginSection{\sectionpage}

\frame{\sectionpage}

\setbeamertemplate{section page}[BFH-ruled]

\frame{\sectionpage}

%Change base color scheme (option can be added)

\setbeamercolor{BFH}{parent=BFH-Orange}

\frame{\sectionpage}

\setbeamertemplate{section page}[BFH]

\frame{\sectionpage}

\end{document}
